\subsection{Fuzzy Set and Fuzzy Relation}
\label{sec:FSAFR}
The mathematical modeling of fuzzy concepts was presented by Zadeh in 1965, and his approach is described here. His intention is to represent the meaning in natural language as a matter of degree. For instance, for a proposition such as ``Mary is young.'', it is not always possible to be asserted as either true or false. It depends on the definition of concept ``young'', if ``young'' is given as a set
\[Young = \{x|x \in [0,\infty), x<23\}\]
and Mary's age is given as 21, then the proposition ``Mary is young'' is true under the definition of ``young'' above. It seems that classical set theory solves the problem of representing the imprecise information such as `young' in the the example above. However, 18 and 20 year olds are young, but with different degrees: 18 is younger than 20. This suggests that the membership in `Young' concept should not be on a 0 or 1 basis, but rather on 0 to 1 scale, that is, the membership should be an element of the interval [0,1].

For representation of membership with degree, fuzzy set is defined in the same way as the definition of crisp set.

\begin{defin} \textbf{Crisp Set.} 
\label{def:CrispSet}
Let $U$ be the universal set, an ordinary subset $A$ of $U$ is determined by its \textbf{indicator function}, or \textbf{characteristic function} $\chi_A$ defined by
\[
  \chi_A(x) = \left\{ 
  \begin{array}{l l}
    1 & \quad \text{if $x \in A$}\\
    0 & \quad \text{if $x \notin A$}\\
  \end{array} \right.
\]
\end{defin}
The indicator function of a subset $A$ of the universal set $U$ specifies whether an element is in $A$ or not. Only two values can be taken for the characteristic function, $0$ and $1$. The characteristic function could be generalized as $\chi_A : U \rightarrow \{0,1\}$.

According to the definition above, crisp set only has $0$, $1$ as there membership degree. However, in fuzzy set, the degree value is a real number in $[0,1]$.  

\begin{defin} \textbf{Fuzzy Set.}
\label{def:FuzzySet}
Let $U$ be the universal set, which includes all the elements in our interested domain. A \textbf{fuzzy subset} $A$ of the universal set $U$ is a function $\mu_A : U \rightarrow [0,1]$. It is common to refer to a fuzzy subset simply as a \textbf{fuzzy set}.

\end{defin} 
The image of the fuzzy function $\mu_A$ contains two values $0$ and $1$, which correspond to the image of crisp subset of $U$. Therefore, crisp subsets are considered as special cases of fuzzy subsets.

It is customary in fuzzy literature to represent fuzzy set in both notations, $A$ and $\mu_A$. The first one is called `linguistic label', which indicates the concept of interested domain. For example, Let $A$ be a set of young people. Then, the $\mu_A : U \rightarrow [0,1]$ is called fuzzy function, representing the degree of youngness that has been assigned to each member in $U$.  

Suppose that all the numbers which are specified as ages belong to $U$, then a value in $[0,1]$ is assigned to each number in $U$. For example, $\mu_{young}(18)=0.85$, $\mu_{young}(20)=0.81$, then ``18 is younger then 20'' has its semantic meaning under $\mu_{young}$ or concept `young', which is $\mu_{young}(18) > \mu_{young}(20)$.

\subsubsection{Operations on fuzzy set}
\label{sec:OperationOnFuzzySet}
Following the definition above, a crisp subset $A$ of universal set $U$ can be represented by a function $\chi_A : U \rightarrow \{0,1\}$, while a fuzzy subset $\tilde{A}$ of universal set $U$ is indicated as a function $\mu_{\tilde{A}} : U \rightarrow [0,1]$. On the subsets of $U$, whether they are crisp or fuzzy, the operations of union, intersection, and complement are given as follows.

The original definition is given by the rules
\begin{center}
$A \cup B = \{x|x \in A$ or $x \in B\}$

$A \cap B = \{x|x \in A$ and $x \in B\}$

$A^{'} = \{x \in U | x \notin A\}$
\end{center}

From the point of view of function, the operations over crisp subsets are:
\[\chi_{A \cup B}(x) = max\{\chi_A(x),\chi_B(x)\} = \chi_A(x) \vee \chi_B(x)\] 
\[\chi_{A \cap B}(x) = min\{\chi_A(x),\chi_B(x)\} = \chi_A(x) \wedge \chi_B(x)\]
\[\chi_{A^{'}}(x) = 1 - \chi_A(x)\] 

Naturally, by the membership function, these operations over fuzzy subsets of $U$ are extended as,

\[\mu_{\tilde{A} \cup \tilde{B}}(x) = sup\{\mu_{\tilde{A}}(x),\mu_{\tilde{B}}(x)\}=\mu_{\tilde{A}}(x) \vee \mu_{\tilde{B}}(x)\]
\[\mu_{\tilde{A} \cap \tilde{B}}(x) = inf\{\mu_{\tilde{A}}(x),\mu_{\tilde{B}}(x)\}=\mu_{\tilde{A}}(x) \wedge \mu_{\tilde{B}}(x)\]
\[\mu_{\tilde{A^{'}}}(x) = 1 - \mu_{\tilde{A}}(x)\]

The syntax $\vee$ and $\wedge$ are the operations over sets, which are mentioned in section 2.1 Lattice. Note that they are different from the logic notation used for `or' and `and'. 

\subsubsection{Fuzzy Relation}
\label{sec:FuzzyRelation}
Relations, or associations among objects, are of fundamental importance in the analysis of real-world system. Since the real-world is full of vague and uncertain information, the fuzzy relation plays an important role to represent it.

The classical relation has been defined in the lattice section. In this subsection, the fuzzy relation will be generalized from the crisp one.

\begin{defin} \textbf{Fuzzy Relation.}
\label{def:FuzzyRelation}
An \textbf{n-ary fuzzy function relation in a set} $V = U_1 \times U_2 \times ... \times U_n$ is a fuzzy subset $R$ of $V$, that is, a function $R : V \rightarrow [0,1]$. If the sets $U_i$ are identical, say $U$, then $R$ is an \textbf{n-ary fuzzy relation on} U.
 
\end{defin}