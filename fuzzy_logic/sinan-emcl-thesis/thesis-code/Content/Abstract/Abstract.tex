\documentclass[Thesis.tex]{subfiles} 
\begin{document}

\chapter*{\centering Abstract}
\label{chap:Abstract}

Zadeh introduced Fuzzy Logic as an extension of classical logic capable of dealing with imprecise information. It assigns a real number in $[0,1]$ to a logical statement, instead of just $0$ (false), $1$ (true).  In this work, we restrict ourselves to two aspects interesting by themselves in Fuzzy Logic: Similarity and Quantification.

Similarity between two fuzzy concepts itself is indeed a fuzzy concept. The value of similarity is
a real number between $[0,1]$. $1$ represents that two comparing concepts are completely similar, we prefer to call them identical. $0$ represents that two comparing concepts are completely different or incomparable. 

Quantification is a fuzzy function taking fuzzy concepts as its arguments to evaluate the truth value of statement. It also can be considered as a modifier of the degree in which elements or individuals belong to a fuzzy set.

There are two measurements of obtaining similarity between two fuzzy predicates, Interpretation Based(IBM) and Structure Based(SBM). IBM is based on set theory and corresponding relation between fuzzy set/relation and Fuzzy Logic.  In this sense, similarity relation between fuzzy predicates is achieved by their interpretations. Whilst, in BMS, fuzzy predicates are represented as predicate trees according to fuzzy rules they are involved in. The similarity between two predicates is obtained by comparison and expansion, which are taken place alternatively layer by layer till the corresponding trees are completely built.

Quantification is defined as both second order and first order fuzzy predicate. With quantification, negation, fuzzy concept, and fuzzy rules, a complex query in Fuzzy Logic is generated to represent query and statement imprecisely in natural language. 

Our formalization of similarity and quantification in Fuzzy Logic increase the expressivity of Fuzzy Logic by allowing us to model much more problems in real world.\\ 


\end{document}