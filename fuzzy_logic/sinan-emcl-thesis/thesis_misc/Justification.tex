\documentclass[egilmezThesis.tex]{subfiles} 
\begin{document}
\chapter{The Need For a New  Proposal to Evaluate Similarity Between Fuzzy
Predicates Defined In a Fuzzy Program}
\label{chap:Justification}

In chapter \ref{chap:Introduction} we have mentioned about some existing works in the field of similarity concept in Fuzzy Logic, and also the \textit{SBM} method \cite{Lu} has been highlighted as an intriguing work in the field. In this chapter the \textit{SBM} method is observed in detail, and the layout of the section is as follows: 

In Structured Based Measurement,  a tree is built for each predicate, so comparing two predicates is indeed comparing two trees. In the light of that, section \ref{sec:PredicateTree} introduces the approach for building a tree for predicate. In section \ref{sec:Algorithm}, the algorithm to obtain similarity between two trees is described in detail. Then the shortcomings of the approach are displayed with their corresponding examples in section \ref{sec:Shortcomings}.

\documentclass[Thesis.tex]{subfiles} 
\begin{document}

\chapter{Similarity of fuzzy logic}
\label{chap:Similarity}
\subsection{overview of similarity}
\label{sec:overview}
%\section{overview of similarity}
%\label{sec:overview} Not in Thesis
Similarity appears and is applied almost everywhere in our daily life. Natural language gives a lot of vivid examples. The adjective vocabularies to describe a pretty girl, could be `good-looking', `beautiful', `nice', `attractive', and even more can be found in Shakespeare's work or Oxford dictionary. The phenomenon that there is the semantic similarity among vocabularies is called synonym in linguistics research area. The well-known linguistic project containing the similarity idea is WordNet \cite{Fel98}, in which, nouns, verbs, adjectives and adverbs are grouped into sets of cognitive synonyms, each expressing a distinct concept. 

Not only just for mechanical fun in linguistic research, similarity is also widely used for the cognitive processes in the human mind, visual and acoustic perception, motor movements and imprecise and uncertain knowledge representation and approximate reasoning and games.

Since natural language is ambiguous and vague, the keyword you search for in searching engines on web are mostly imprecise. For example, if `old furniture' is the keyword typed, `old' should be considered as a fuzzy concept, since we don't know how `old' should be. It may retrieve the `classical furniture', `antique furniture', or may obtain the `second hand furniture', which is quite different from `classical furniture' and `antique furniture'. However, `classical', `antique', and `second hand' have certain different similarity with the word `old'. The search engines returns `classical furniture', `antique furniture' and `second hand furniture', and other results which are similar to the keywords `old'. Therefore, by the semantic similarity, the concept-related result is returned, which makes the search process imprecise, but in fact returns the expected results, since in reality, the interesting information a human being requires is not always accurate, but approximate, which makes similarity important and necessary in both research area and industrial engineering. 

Fuzzy Logic is used to represent vague information in the real world. In order to draw the similarity between objects in real world, we introduce similarity between predicates into Fuzzy Logic. Similarity itself is a fuzzy concept, which makes fuzzy logic suitable and appropriate for representing similarity. 

In order to embed similarity into \textit{Fuzzy Logic}, we focus on the similarity between predicates. The reasons are,

\begin{enumerate}
\item In Fuzzy Logic, term is presented to be an object in the interested domain. While predicate represents the fuzzy concept and mostly is interpreted as property of terms. The similarity between two objects is achieved by comparing their properties. Thus, comparing predicates is actually comparing the properties of objects.

\item In Fuzzy Logic, the atomic query is an atom, which is a predicate with its arguments. Obtaining the similarity between predicates,  is one approach to fruit similar queries, which leads to similar answers.
\end{enumerate}

\subsection{Structure of the paper}
%In this Chapter,
In this paper,  we present two approaches to obtain similarity between predicates. One is \textit{Interpretation\_Based Measurement} in section \ref{sec:IBM}, and another is presented in section \ref{sec:SBM}, called \textit{Structured\_Based Measurement}. We compare these two measurements in section \ref{sec:CompTwoMeasurements}. The conclusion is drawn in the final section.

\documentclass[Thesis.tex]{subfiles} 
\begin{document}

\chapter{Similarity of fuzzy logic}
\label{chap:Similarity}
\subsection{overview of similarity}
\label{sec:overview}
%\section{overview of similarity}
%\label{sec:overview} Not in Thesis
Similarity appears and is applied almost everywhere in our daily life. Natural language gives a lot of vivid examples. The adjective vocabularies to describe a pretty girl, could be `good-looking', `beautiful', `nice', `attractive', and even more can be found in Shakespeare's work or Oxford dictionary. The phenomenon that there is the semantic similarity among vocabularies is called synonym in linguistics research area. The well-known linguistic project containing the similarity idea is WordNet \cite{Fel98}, in which, nouns, verbs, adjectives and adverbs are grouped into sets of cognitive synonyms, each expressing a distinct concept. 

Not only just for mechanical fun in linguistic research, similarity is also widely used for the cognitive processes in the human mind, visual and acoustic perception, motor movements and imprecise and uncertain knowledge representation and approximate reasoning and games.

Since natural language is ambiguous and vague, the keyword you search for in searching engines on web are mostly imprecise. For example, if `old furniture' is the keyword typed, `old' should be considered as a fuzzy concept, since we don't know how `old' should be. It may retrieve the `classical furniture', `antique furniture', or may obtain the `second hand furniture', which is quite different from `classical furniture' and `antique furniture'. However, `classical', `antique', and `second hand' have certain different similarity with the word `old'. The search engines returns `classical furniture', `antique furniture' and `second hand furniture', and other results which are similar to the keywords `old'. Therefore, by the semantic similarity, the concept-related result is returned, which makes the search process imprecise, but in fact returns the expected results, since in reality, the interesting information a human being requires is not always accurate, but approximate, which makes similarity important and necessary in both research area and industrial engineering. 

Fuzzy Logic is used to represent vague information in the real world. In order to draw the similarity between objects in real world, we introduce similarity between predicates into Fuzzy Logic. Similarity itself is a fuzzy concept, which makes fuzzy logic suitable and appropriate for representing similarity. 

In order to embed similarity into \textit{Fuzzy Logic}, we focus on the similarity between predicates. The reasons are,

\begin{enumerate}
\item In Fuzzy Logic, term is presented to be an object in the interested domain. While predicate represents the fuzzy concept and mostly is interpreted as property of terms. The similarity between two objects is achieved by comparing their properties. Thus, comparing predicates is actually comparing the properties of objects.

\item In Fuzzy Logic, the atomic query is an atom, which is a predicate with its arguments. Obtaining the similarity between predicates,  is one approach to fruit similar queries, which leads to similar answers.
\end{enumerate}

\subsection{Structure of the paper}
%In this Chapter,
In this paper,  we present two approaches to obtain similarity between predicates. One is \textit{Interpretation\_Based Measurement} in section \ref{sec:IBM}, and another is presented in section \ref{sec:SBM}, called \textit{Structured\_Based Measurement}. We compare these two measurements in section \ref{sec:CompTwoMeasurements}. The conclusion is drawn in the final section.

\documentclass[Thesis.tex]{subfiles} 
\begin{document}

\chapter{Similarity of fuzzy logic}
\label{chap:Similarity}
\input{Content/Similarity/Content/Introduction/introduction}

\input{Content/Similarity/Content/Interpretation_Base/similarity}

\input{Content/Similarity/Content/Structure_Base/similarity}

\input{Content/Similarity/Content/Comparison/comparison}
\end{document}

\documentclass[Thesis.tex]{subfiles} 
\begin{document}

\chapter{Similarity of fuzzy logic}
\label{chap:Similarity}
\input{Content/Similarity/Content/Introduction/introduction}

\input{Content/Similarity/Content/Interpretation_Base/similarity}

\input{Content/Similarity/Content/Structure_Base/similarity}

\input{Content/Similarity/Content/Comparison/comparison}
\end{document}

\section{Comparison of two measurements}
\label{sec:CompTwoMeasurements}
In this section, we compare two measurements which are Interpretation Based Measurement (IBM) in section \ref{sec:IBM}, and  Structure Based Measurement (SBM) in section \ref{sec:SBM}. The comparison is drawn from three points, the intuitions, the methods, and the results of IBM and SBM.

\begin{itemize}
\item Intuition 

The similarity between predicates in Fuzzy Logic is to represent the similarity between objects in the real world. Therefore, the intuition begins with  the similarity between objects. In Interpretation Based Measurement (IBM), objects are considered to be represented as a set of attributes it possesses, and similarity between two objects is the commonality of attributes of them. In Structure Based Measurement (SBM), the definition of objects is the key to obtain similarity. It works like mathematical proof, starting with the basic definition. And in SBM, the definition of objects  is represented with structured property.

\item Methotology

By Intuition of IBM, the problem is formalized as the similarity between sets. We start with discussing the similarity between crisp sets, and generalizing into similarity between fuzzy sets. According to the corresponding relation between fuzzy set and Fuzzy Logic presented in Chapter \ref{chap:CorrespondingRelation}, the similarity between fuzzy predicates is deduced from similarity between fuzzy sets. The idea of SBM is from an interesting topic in \textit{Foundations of Databases} \cite{AHV95}, which is achieving subsumption and equivalence between queries. The method is to find the isomorphism between atoms in the bodies of two comparing queries. In SBM, fuzzy rule with fuzzy predicate as head are considered as the ``defintion" of this predicate. To obtain similarity between two predicates, we start with their ``definitions" which are fuzzy rules by finding the similarity between the predicates in their bodies. Since the ``definition" of predicate is defined inductively. The basic one is the predicate which never appears in the head of any fuzzy rules. The predicate is formalized as predicate tree. The similarity between predicates is obtaining by comparing predicate trees. The Algorithm is defined inductively with promising complexity, since inductive is a property of tree.

\item Result

From IBM, both \textit{essential similarity} and \textit{surficial similarity} are obtained. From intuition of IBM, we actually assume the statement `` if two objects have more attributes in common, then they are more similar to each other" is true. But the true statement should be ``if two objects are more similar, then they have more attributes in common." 
\textit{Essential similarity} and \textit{surficial similarity} satisfy the first statement, because that is the intuition of IBM, used to obtain the result. However, \textit{essential similarity} satisfies the second statement, but \textit{surficial similarity} doesn't. That is why we name it ``surficial". The result from SBM avoids the disadvantage in IBM, but if some basic similarity is not defined, the ``essential similarity" will not be shown in the result. For example, comparing $p_1$ and $p_2$, there are two atomic predicates $q_1$ and $q_2$ in their ``definition" respectively, they may be similar to each other, but if this similarity between $q_1$ and $q_2$ is not defined, the similarity between $p_1$ and $p_2$ will be affected, like ``distortion". Also, only predicates with the same type and their fuzzy rules with same connective can be compared, which limits the result of the essential similarity between fuzzy concepts.

\end{itemize}



\end{document}

\documentclass[Thesis.tex]{subfiles} 
\begin{document}

\chapter{Similarity of fuzzy logic}
\label{chap:Similarity}
\subsection{overview of similarity}
\label{sec:overview}
%\section{overview of similarity}
%\label{sec:overview} Not in Thesis
Similarity appears and is applied almost everywhere in our daily life. Natural language gives a lot of vivid examples. The adjective vocabularies to describe a pretty girl, could be `good-looking', `beautiful', `nice', `attractive', and even more can be found in Shakespeare's work or Oxford dictionary. The phenomenon that there is the semantic similarity among vocabularies is called synonym in linguistics research area. The well-known linguistic project containing the similarity idea is WordNet \cite{Fel98}, in which, nouns, verbs, adjectives and adverbs are grouped into sets of cognitive synonyms, each expressing a distinct concept. 

Not only just for mechanical fun in linguistic research, similarity is also widely used for the cognitive processes in the human mind, visual and acoustic perception, motor movements and imprecise and uncertain knowledge representation and approximate reasoning and games.

Since natural language is ambiguous and vague, the keyword you search for in searching engines on web are mostly imprecise. For example, if `old furniture' is the keyword typed, `old' should be considered as a fuzzy concept, since we don't know how `old' should be. It may retrieve the `classical furniture', `antique furniture', or may obtain the `second hand furniture', which is quite different from `classical furniture' and `antique furniture'. However, `classical', `antique', and `second hand' have certain different similarity with the word `old'. The search engines returns `classical furniture', `antique furniture' and `second hand furniture', and other results which are similar to the keywords `old'. Therefore, by the semantic similarity, the concept-related result is returned, which makes the search process imprecise, but in fact returns the expected results, since in reality, the interesting information a human being requires is not always accurate, but approximate, which makes similarity important and necessary in both research area and industrial engineering. 

Fuzzy Logic is used to represent vague information in the real world. In order to draw the similarity between objects in real world, we introduce similarity between predicates into Fuzzy Logic. Similarity itself is a fuzzy concept, which makes fuzzy logic suitable and appropriate for representing similarity. 

In order to embed similarity into \textit{Fuzzy Logic}, we focus on the similarity between predicates. The reasons are,

\begin{enumerate}
\item In Fuzzy Logic, term is presented to be an object in the interested domain. While predicate represents the fuzzy concept and mostly is interpreted as property of terms. The similarity between two objects is achieved by comparing their properties. Thus, comparing predicates is actually comparing the properties of objects.

\item In Fuzzy Logic, the atomic query is an atom, which is a predicate with its arguments. Obtaining the similarity between predicates,  is one approach to fruit similar queries, which leads to similar answers.
\end{enumerate}

\subsection{Structure of the paper}
%In this Chapter,
In this paper,  we present two approaches to obtain similarity between predicates. One is \textit{Interpretation\_Based Measurement} in section \ref{sec:IBM}, and another is presented in section \ref{sec:SBM}, called \textit{Structured\_Based Measurement}. We compare these two measurements in section \ref{sec:CompTwoMeasurements}. The conclusion is drawn in the final section.

\documentclass[Thesis.tex]{subfiles} 
\begin{document}

\chapter{Similarity of fuzzy logic}
\label{chap:Similarity}
\input{Content/Similarity/Content/Introduction/introduction}

\input{Content/Similarity/Content/Interpretation_Base/similarity}

\input{Content/Similarity/Content/Structure_Base/similarity}

\input{Content/Similarity/Content/Comparison/comparison}
\end{document}

\documentclass[Thesis.tex]{subfiles} 
\begin{document}

\chapter{Similarity of fuzzy logic}
\label{chap:Similarity}
\input{Content/Similarity/Content/Introduction/introduction}

\input{Content/Similarity/Content/Interpretation_Base/similarity}

\input{Content/Similarity/Content/Structure_Base/similarity}

\input{Content/Similarity/Content/Comparison/comparison}
\end{document}

\section{Comparison of two measurements}
\label{sec:CompTwoMeasurements}
In this section, we compare two measurements which are Interpretation Based Measurement (IBM) in section \ref{sec:IBM}, and  Structure Based Measurement (SBM) in section \ref{sec:SBM}. The comparison is drawn from three points, the intuitions, the methods, and the results of IBM and SBM.

\begin{itemize}
\item Intuition 

The similarity between predicates in Fuzzy Logic is to represent the similarity between objects in the real world. Therefore, the intuition begins with  the similarity between objects. In Interpretation Based Measurement (IBM), objects are considered to be represented as a set of attributes it possesses, and similarity between two objects is the commonality of attributes of them. In Structure Based Measurement (SBM), the definition of objects is the key to obtain similarity. It works like mathematical proof, starting with the basic definition. And in SBM, the definition of objects  is represented with structured property.

\item Methotology

By Intuition of IBM, the problem is formalized as the similarity between sets. We start with discussing the similarity between crisp sets, and generalizing into similarity between fuzzy sets. According to the corresponding relation between fuzzy set and Fuzzy Logic presented in Chapter \ref{chap:CorrespondingRelation}, the similarity between fuzzy predicates is deduced from similarity between fuzzy sets. The idea of SBM is from an interesting topic in \textit{Foundations of Databases} \cite{AHV95}, which is achieving subsumption and equivalence between queries. The method is to find the isomorphism between atoms in the bodies of two comparing queries. In SBM, fuzzy rule with fuzzy predicate as head are considered as the ``defintion" of this predicate. To obtain similarity between two predicates, we start with their ``definitions" which are fuzzy rules by finding the similarity between the predicates in their bodies. Since the ``definition" of predicate is defined inductively. The basic one is the predicate which never appears in the head of any fuzzy rules. The predicate is formalized as predicate tree. The similarity between predicates is obtaining by comparing predicate trees. The Algorithm is defined inductively with promising complexity, since inductive is a property of tree.

\item Result

From IBM, both \textit{essential similarity} and \textit{surficial similarity} are obtained. From intuition of IBM, we actually assume the statement `` if two objects have more attributes in common, then they are more similar to each other" is true. But the true statement should be ``if two objects are more similar, then they have more attributes in common." 
\textit{Essential similarity} and \textit{surficial similarity} satisfy the first statement, because that is the intuition of IBM, used to obtain the result. However, \textit{essential similarity} satisfies the second statement, but \textit{surficial similarity} doesn't. That is why we name it ``surficial". The result from SBM avoids the disadvantage in IBM, but if some basic similarity is not defined, the ``essential similarity" will not be shown in the result. For example, comparing $p_1$ and $p_2$, there are two atomic predicates $q_1$ and $q_2$ in their ``definition" respectively, they may be similar to each other, but if this similarity between $q_1$ and $q_2$ is not defined, the similarity between $p_1$ and $p_2$ will be affected, like ``distortion". Also, only predicates with the same type and their fuzzy rules with same connective can be compared, which limits the result of the essential similarity between fuzzy concepts.

\end{itemize}



\end{document}

\section{Comparison of two measurements}
\label{sec:CompTwoMeasurements}
In this section, we compare two measurements which are Interpretation Based Measurement (IBM) in section \ref{sec:IBM}, and  Structure Based Measurement (SBM) in section \ref{sec:SBM}. The comparison is drawn from three points, the intuitions, the methods, and the results of IBM and SBM.

\begin{itemize}
\item Intuition 

The similarity between predicates in Fuzzy Logic is to represent the similarity between objects in the real world. Therefore, the intuition begins with  the similarity between objects. In Interpretation Based Measurement (IBM), objects are considered to be represented as a set of attributes it possesses, and similarity between two objects is the commonality of attributes of them. In Structure Based Measurement (SBM), the definition of objects is the key to obtain similarity. It works like mathematical proof, starting with the basic definition. And in SBM, the definition of objects  is represented with structured property.

\item Methotology

By Intuition of IBM, the problem is formalized as the similarity between sets. We start with discussing the similarity between crisp sets, and generalizing into similarity between fuzzy sets. According to the corresponding relation between fuzzy set and Fuzzy Logic presented in Chapter \ref{chap:CorrespondingRelation}, the similarity between fuzzy predicates is deduced from similarity between fuzzy sets. The idea of SBM is from an interesting topic in \textit{Foundations of Databases} \cite{AHV95}, which is achieving subsumption and equivalence between queries. The method is to find the isomorphism between atoms in the bodies of two comparing queries. In SBM, fuzzy rule with fuzzy predicate as head are considered as the ``defintion" of this predicate. To obtain similarity between two predicates, we start with their ``definitions" which are fuzzy rules by finding the similarity between the predicates in their bodies. Since the ``definition" of predicate is defined inductively. The basic one is the predicate which never appears in the head of any fuzzy rules. The predicate is formalized as predicate tree. The similarity between predicates is obtaining by comparing predicate trees. The Algorithm is defined inductively with promising complexity, since inductive is a property of tree.

\item Result

From IBM, both \textit{essential similarity} and \textit{surficial similarity} are obtained. From intuition of IBM, we actually assume the statement `` if two objects have more attributes in common, then they are more similar to each other" is true. But the true statement should be ``if two objects are more similar, then they have more attributes in common." 
\textit{Essential similarity} and \textit{surficial similarity} satisfy the first statement, because that is the intuition of IBM, used to obtain the result. However, \textit{essential similarity} satisfies the second statement, but \textit{surficial similarity} doesn't. That is why we name it ``surficial". The result from SBM avoids the disadvantage in IBM, but if some basic similarity is not defined, the ``essential similarity" will not be shown in the result. For example, comparing $p_1$ and $p_2$, there are two atomic predicates $q_1$ and $q_2$ in their ``definition" respectively, they may be similar to each other, but if this similarity between $q_1$ and $q_2$ is not defined, the similarity between $p_1$ and $p_2$ will be affected, like ``distortion". Also, only predicates with the same type and their fuzzy rules with same connective can be compared, which limits the result of the essential similarity between fuzzy concepts.

\end{itemize}



\end{document}



\end{document}