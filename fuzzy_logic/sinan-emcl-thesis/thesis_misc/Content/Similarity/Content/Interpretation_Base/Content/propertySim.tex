\subsection{Properties of similarity}
\label{sec:Property}
$U$ is the universe of discourse, and similarity is defined over two subsets of $U$, which is formalized as a function $Sim: 2^{U} \times 2^{U} \rightarrow [0,1]$, which is also considered as a binary relation. According to the formulas \eqref{CSFinite}, \eqref{CSInfinite}, \eqref{FSFinite}and \eqref{FSInfinite}, some classical properties of binary relation are presented as follow, where $A$,$B$ and $C$ are subsets of $U$.
\begin{itemize}
\item \textbf{reflexivity}
$Sim(A,A)=Sim(A,A)$, more important is $Sim(A,A)=1$.
\item \textbf{symmetry}
$Sim(A,B)=Sim(B,A)$
\item \textbf{triangle inequality}
Triangle inequality holds, which is $Sim(A,C)<=Sim(A,B)+Sim(B,C)$. In other words, if $Sim(A,B)=v$ and $Sim(B,C)=u$, it implies $Sim(A,C)<=u+v$. 
\end{itemize}
