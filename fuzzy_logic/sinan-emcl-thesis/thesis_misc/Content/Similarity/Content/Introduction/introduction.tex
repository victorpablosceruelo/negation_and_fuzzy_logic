Similarity appears and is applied almost everywhere in our daily life. Natural language gives a lot of vivid examples. The adjective vocabularies to describe a pretty girl, could be `good-looking', `beautiful', `nice', `attractive', and even more can be found in Shakespeare's work or Oxford dictionary. The phenomenon that there is the semantic similarity among vocabularies is called synonym in linguistics research area. The well-known linguistic project containing the similarity idea is WordNet \cite{Fel98}, in which, nouns, verbs, adjectives and adverbs are grouped into sets of cognitive synonyms, each expressing a distinct concept. 

Not only just for mechanical fun in linguistic research, similarity is also widely used for the cognitive processes in the human mind, visual and acoustic perception, motor movements and imprecise and uncertain knowledge representation and approximate reasoning and games.

Since natural language is ambiguous and vague, the keyword you search for in searching engines on web are mostly imprecise. For example, if `old furniture' is the keyword typed, `old' should be considered as a fuzzy concept, since we don't know how `old' should be. It may retrieve the `classical furniture', `antique furniture', or may obtain the `second hand furniture', which is quite different from `classical furniture' and `antique furniture'. However, `classical', `antique', and `second hand' have certain different similarity with the word `old'. The search engines returns `classical furniture', `antique furniture' and `second hand furniture', and other results which are similar to the keywords `old'. Therefore, by the semantic similarity, the concept-related result is returned, which makes the search process imprecise, but in fact returns the expected results, since in reality, the interesting information a human being requires is not always accurate, but approximate, which makes similarity important and necessary in both research area and industrial engineering. 

In order to embed similarity into \textit{Fuzzy Logic}, we focus on the similarity between predicates. The reasons are,

\begin{enumerate}
\item In Fuzzy Logic, term is presented to be an object in the interested domain. While predicate represents the fuzzy concept and mostly is interpreted as property of terms. The similarity between two objects is achieved by comparing their properties. Thus, comparing predicates is actually comparing the properties of objects.

\item In Fuzzy Logic, the atomic query is an atom, which is a predicate with its arguments. Obtaining the similarity between predicates,  is one approach to fruit similar queries, which leads to similar answers.
\end{enumerate}

In this Chapter, we present two approaches to obtain similarity between predicates. One is \textit{Interpretation\_Based Measurement} in section \ref{sec:IBM}, and another is presented in section \ref{sec:SBM}, called \textit{Structured\_Based Measurement}.