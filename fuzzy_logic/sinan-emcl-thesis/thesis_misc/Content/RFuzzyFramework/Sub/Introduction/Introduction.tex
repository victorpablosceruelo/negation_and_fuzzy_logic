\subsection{Introduction}
Logic programming \cite{Llo87} has been successfully used in knowledge representation and reasoning for decades. Indeed, world data in not always perceived in a crisp way. Information that we gather might be imperfect, uncertain, or fuzzy in some other way. Hence the management of uncertainty and fuzziness is very important in knowledge representation.

Introducing Fuzzy Logic into Logic Programming has provided the development of several fuzzy systems over Prolog. These systems replace its inference mechanism, SLD-resolution, with a fuzzy variant that is able to handle partial truth. Most of these systems implement the fuzzy resolution introduced by Lee in \cite{Lee72}, as the Prolog-Elf system, the FRIL Prolog system \cite{IK85} and the F-Prolog language \cite{LL90}. However, there is no common method for fuzzifying Prolog, as noted in \cite{SDM89}.

One of the most promising fuzzy tools for Prolog is RFuzzy Framework. The most important advantages against the other approaches are:
\begin{enumerate}
\item A truth value is represented as a real number in unit interval $[0,1]$, satisfying certain constrains. 
\item A truth value is propagated through the rules by means of an aggregation operator. The definition of this aggregation operator is general and is subsumes conjunctive operators (triangular norms \cite{KPMRPE00} like min, prod, etc.), disjunctive operators \cite{TCC95} (triangular co-norms, like max, sum, etc.),
average operators (average as arithmetic average, quasi-linear average, etc.) and hybrid operators (combinations of the above operators \cite{PTC02}). 
\item Crisp and fuzzy reasoning are consistently combined
\item It provides some interesting improvements with respect to FLOPER \cite{MM08a,Mor06}: default values, partial default values, typed predicates and useful syntactic sugar (for presenting facts, rules and functions).
\end{enumerate}