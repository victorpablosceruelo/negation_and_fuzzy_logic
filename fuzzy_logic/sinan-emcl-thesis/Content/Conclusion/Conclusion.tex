\documentclass[Thesis.tex]{subfiles} 
\begin{document}

\chapter{Conclusions}
\label{chap:Conclusions}

Our main contribution in this work is a proposal of syntax and semantics for similarity and quantification in Fuzzy Logic.
% In general, similarity problem and quantification problem
Similarity between two objects is widely used in lots of research areas, such as Pattern Recognition, Bio-information. The expressivity of quantifiers of natural language is always a hot topic in formal logic. Fuzzy Logic considered as a formalization of vague information in real world is a tailored platform for the representation of \textit{similarity} and \textit{quantification}. 
% Similarity problem

In similarity, we present two measurement of achieving it between two fuzzy predicates, which are Interpretation Based and Structure Based. Interpretation Based measurement (IBM) obtains the similarity between fuzzy predicates by the comparison between their interpretations. 
It follows the intuition that `` if two objects have more attributes in common, then they are more similar to each other" and generates both \textit{essential similarity} and \textit{surficial similarity}.
But in fact, the truth is ``If two objects are more similar, then they have more attributes in common." 
\textit{Essential similarity} satisfies this statement, but \textit{surficial similarity} doesn't. SBM avoids the disadvantage in IBM. It considers the structure of fuzzy predicates, that is, the way that fuzzy rules defines fuzzy concepts. Besides, the algorithm of SBM not only has a promising complexity, and also could be used to solve the other problems based on similarity, which have structure property inside, such as gene similarity, chemistry similarity, ontology similarity.

% Quantification problem
Fuzzy quantification is used to express quantifiers of natural language, such as ``some", ``all", ``almost", ``extremely". It could be formalized as both second order and first order fuzzy predicate in Fuzzy Logic. Our proposal allows to apply quantification to Fuzzy Logic, increasing its expressivity. 
With respect to the implementation of quantification in RFuzzy Framework,  since there are difficulties when dealing with second order fuzzy predicate, we propose to reduce it to first order.  For this, we generate new fuzzy quantifier for each fuzzy concept which is an argument of original fuzzy predicate. It makes the new fuzzy quantifier be defined as first order fuzzy predicate, whose function is $[0,1] \rightarrow [0,1]$. This implementation offers simplicity to generate simple fuzzy queries such as ``extremely strict professor", ``not very expensive laptop". With the fuzzy rule to combine several simple fuzzy queries, a complex fuzzy query is generated. It is considered as an application of quantification.

% In general, similarity and quantification
Similarity and quantification are concepts used in many research areas such as Computational Linguistics, Similar Searching, Vision Process, Pattern Recognition. The proposal demonstrated in Fuzzy Logic also can be applied in those areas. Back to intuition in real world, the answers to most of questions are not just `Yes' or `No'. Similarity and quantification as two characteristics of Fuzzy Logic simulate imprecise information in real world in some sense, and achieve humanized results, that is, approximately close to human being's demand. 

\end{document}