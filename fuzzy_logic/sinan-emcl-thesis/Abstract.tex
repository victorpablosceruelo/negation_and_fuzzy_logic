\documentclass[egilmezThesis.tex]{subfiles} 
\begin{document}


\chapter*{\centering \LARGE{Abstract}}
\label{chap:Abstract}

\small{Real world applications often come with imprecise, uncertain or incomplete information. Fuzzy logic was born as a consequence of  the incapability of classical two-valued logic systems when dealing with such domains.}

\small{The concept of similarity adds another layer to the field, as it introduces a fuzzy domain not only for the concepts, but also concerning the relations between concepts. This feature results into many means for flexible querying on any type of knowledge bases. For that reason, there has been a renewed interest in \textit{Similarity-Based Logic Programming} over the last decade which has proved to be fruitful.}

\small{In this research we propose a new methodology concerning evaluating the similarity proximities in fuzzy logic domain, which eliminates all shortcomings of the preceding approaches. Moreover we introduce a framework for approximating the precision of our method by presenting a fully automized algorithm.  We demonstrate the value of our work via displaying the accuracy of the method on real-world scenarios.}



\end{document}