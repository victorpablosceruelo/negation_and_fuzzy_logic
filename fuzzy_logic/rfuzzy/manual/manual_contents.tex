%%%%%%%%%%%%%%%%%%%%%%%
%% CONTENIDO_MEMORIA %%
%%%%%%%%%%%%%%%%%%%%%%%

% Ejemplos de comentarios.
\excludecomment{versionfull}
% \includecomment{versionwhitepaper}
% \excludecomment{versionquestionnaire}
% \excludecomment{versionjavarules}

\newcommand{\baseuri}[0]{https://babel.ls.fi.upm.es/~vpablos/others/packages}
\newcommand{\ciaoversion}[0]{1.13+r11293}

\section{About Rfuzzy}

Rfuzzy is a preprocessor package hosted at
\url{http://babel.ls.fi.upm.es/software/rfuzzy/}
that has been tested for Ciao Prolog \ciaoversion.

More information can be found there, although for semantics we recommend
you the following papers:
%
\cite{MunozHernandez2010}
\cite{jisbd:2010:vpcafdsmh}
\cite{hardandsoft:2010:afdvpcsmh}
\cite{DBLP:conf/eusflat/StrassMC09}
\cite{DBLP:conf/iwann/Munoz-HernandezCS09}
\cite{5156427}
\cite{victor:susana:2008:wlpe}

\section{Install Ciao Prolog}
\subsection{Easy Installation (Debian and Ubuntu)}

Take a look at 
\url{\baseuri /readme.html}. 
Since this version of Rfuzzy has been tested for Ciao Prolog \ciaoversion,
we recommend you to install this and only this version of Ciao Prolog.

\subsection{Not so easy installation}

\subsubsection{Install cowbuilder}
Use apt-get, aptitude or synaptic to install cowbuilder.
\subsubsection{Add to pbuilderrc the mirrors you have in sources.list}
Simple edit '/etc/pbuilderrc' and add the mirrors
you have enabled at your '/etc/apt/sources.list' file in the 'OTHERMIRROR' option. 
We have used the following lines to build ubuntu gutsy i386 precompiled packages:

\textbox{/etc/pbuilderrc OTHERMIRROR option for ubuntu Gutsy}{
  OTHERMIRROR="deb http://archive.ubuntu.com/ubuntu/ gutsy restricted universe multiverse $ \mid $ deb-src http://archive.ubuntu.com/ubuntu/ gutsy restricted universe multiverse $ \mid $ deb http://archive.canonical.com/ubuntu gutsy partner $ \mid $ deb-src http://archive.canonical.com/ubuntu gutsy partner $ \mid $ deb http://security.ubuntu.com/ubuntu/ gutsy-security main restricted universe multiverse $ \mid $ deb-src http://security.ubuntu.com/ubuntu/ gutsy-security main restricted universe multiverse $ \mid $ deb http://archive.ubuntu.com/ubuntu/ gutsy-updates main restricted universe multiverse $ \mid $ deb-src http://archive.ubuntu.com/ubuntu/ gutsy-updates main restricted universe multiverse" 
}

\subsubsection{Download ciao-prolog sources}
Download ciao-prolog sources from 
\url{\baseuri/sources/}. 
The files you need are listed below:
\textbox{Ciao prolog sources' files}{
ciao-prolog\_1.13+svn20080125-1.diff.gz   ... 10.4K \\
ciao-prolog\_1.13+svn20080125-1.dsc       ...  0.6K \\
ciao-prolog\_1.13+svn20080125.orig.tar.gz ... 30.9M \\
}

\subsubsection{Initialize cowbuilder environment, update it and build}
Initialize cowbuilder environment and update it to the last version. 
After that you can do your precompiled packages with the option --build:

\textbox{Cowbuilder commands to make precompiled files}{
  cowbuilder --create \\
  cowbuilder --update \\
  cowbuilder --build ciao-prolog\_1.13+svn20080125-1.dsc \\
}

\subsubsection{Results from compilation}
Finally you'll obtain at /var/cache/pbuilder/result/ all the files you need to 
install ciao prolog. Use dpkg or other tool to install them and you're done.

\section{Installing the rfuzzy library}

Uncompress the \textbf{.tgz} file and copy the {\it rfuzzy} subfolder 
into the ciao library installation folder.
It is usually {\it /usr/lib/ciao/ciao-1.13/library/} or
{\it /usr/local/lib/ciao/ciao-1.13/library/}. 

Simply drag and drop the rfuzzy subfolder there. That's all. 
Be sure all users can read those files and folder 
(use {\it chmod 644 filename} for the files 
and {\it chmod 755 filename} for the folder). 

For debugging purposes we recommend you to install too the 
debugging package, which is included in the \textbf{.tgz} file.
For that simply copy the {\it debugger\_pkg} subfolder 
into the ciao library installation folder.

\section{Examples}

At subfolder examples you have some examples that show you how to use the library.
They are:

\textbox{Examples included}{
  good\_player.pl \\
  human\_development.pl \\
  jobs.pl \\
  restaurant.pl \\
  teams.pl \\
  which\_row.pl \\
}

\section{Rfuzzy beginning}

To start using the rfuzzy package you need to include the following line
in your program source file: 

\textbox{Rfuzzy package beginning}{
:- module(file\_name,\_,[rfuzzy, clpr]).
}

If you want to see what is being executed by the Ciao compiler, 
the easiest way is to install the {\it debugger\_pkg} and 
include the following line (instead of the previous one).

\textbox{Rfuzzy package beginning}{
:- module(file\_name,\_,[rfuzzy, clpr, debugger\_pkg]).
}

The following example shows how this is done:

\line(1,0){300}
\verbatiminput{../examples/good_player.pl}
\line(1,0){300}

Enjoy !!!

% Estos comentarios que hay debajo permiten que herramientas como el
% emacs usen meta-x flyspell-mode para corregir ortografia y para que
% además compilen los archivos desde cualquiera de los incluidos.
% NO borrar este comentario.

%%% Local Variables: 
%%% mode: latex
%%% TeX-master: "manual"
%%% End: 
