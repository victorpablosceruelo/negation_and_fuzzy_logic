\section{Comparison of works}
\label{sec:ComparisonQuantification}
Our definition of fuzzy quantification is based on \textit{fuzzy determiner} in definition \ref{def:FuzzyDeterminer}, which is a second order fuzzy relation. We start with reducing the arguments in determiner according to the demand in the real world. The quantification after simplification is $Q : \mathcal{F}(U)\rightarrow [0,1]$, which is still defined in second order. We degrade this definition into first order by generating new quantifier according to the fuzzy concept it describes. In this process, the expressivity of quantifier is not limited, but the definition is in first order fuzzy predicate, which is easier to implement. However, comparing with the definition of \textit{fuzzy determiner}, the expressivity of our quantification is less, since it only takes one fuzzy set as argument. On the other hand, it gives us simplicity to generate simple query with negation and quantifier. By combining those simple queries using fuzzy rules, complex query can be answered in RFuzzy Framework. It is considered as an application of quantification, which is described in the next section.