\documentclass{beamer}
\usepackage[latin1]{inputenc}
\usetheme{Warsaw}
\title[Similarity and Quantification]{Similarity and Quantification \\ in Fuzzy Logic}
\author{Lu JingWei}
\institute{Facultad of Inform\'atica\\
  Universidad Polit\'ecnica de Madrid \\}
\date{February 11, 2011}
\begin{document}

\begin{frame}
\titlepage
\end{frame}

\begin{frame}{Overview}
\tableofcontents
\end{frame}

\AtBeginSubsection[]
{
  \begin{frame}<beamer>
    \frametitle{Where we are $\dots$}
    \tableofcontents[currentsection,currentsubsection]
  \end{frame}
}

\AtBeginSection[]
{
  \begin{frame}<beamer>
    \frametitle{Where we are $\dots$}
    \tableofcontents[currentsection]
  \end{frame}
}

%\begin{frame}{Overview}
%New Slide
%\end{frame}

\section{Introduction}

\begin{frame} {Overview of Fuzzy Logic}
\begin{itemize}
\item What is Fuzzy Logic (FL) ?
\item Why do we need FL ?
\item ``Narrow" view of FL (Fuzzy Logic).
\item ``Wide" view of FL (Fuzzy Set/Relation).
\end{itemize}
\end{frame}

\section{Similarity}

\subsection{Statement of Problem}
\begin{frame}
\begin{itemize}
\item Real world is imprecise
\item Human demands is imprecise
\item Satisfying Result
\item Mind Reader System
\end{itemize}
\end{frame}

\subsection{Interpretation Based Measurement(IBM)}
\begin{frame}
\end{frame}

\subsection{Structure Based Measurement(SBM)}
\begin{frame}
\end{frame}

\subsection{Comparison between IBM and SBM}
\begin{frame}
\end{frame}

\section{Quantification}

\subsection{Statement of Problem}
\begin{frame}
\end{frame}

\subsection{Formalized in SOL Predicate}
\begin{frame}
\end{frame}

\subsection{Formalized in FOL Predicate}
\begin{frame}
\end{frame}

\subsection{Application of Quantification}
\begin{frame}
\end{frame}

\section{Conclusions}
\begin{frame}
\end{frame}











\end{document}