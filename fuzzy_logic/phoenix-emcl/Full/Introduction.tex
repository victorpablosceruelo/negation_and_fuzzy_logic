\documentclass[main.tex]{subfiles} 
\begin{document}
\section{Introduction}
\label{sec:Introduction}
% 1. Fuzzy Logic
Logic studies the notions of consequence. It expresses natural language in different formal logic, such as propositional logic, predicate logics, modal propositional/predicate logic, many-valued propositional/predicate logic. Logic deals with set of formulas and the relation of consequence among them. The task of formal logic is to represent all this by means of well-defined logical calculi admitting exact investigation. The intuitive distinction among different formal logics is their expressivity of natural language. In real world, most information is vague, most of which can not be represented by classical logic. For instance, we represent sentence``The patient is young." by several well-known classical logic. Propositional Logic can represent it as proposition $p$. Semantically, it only can express true or false to the whole statement, but not the truth value of ``young". First Order Logic is able to express it as $young(the\ patient)$, but only can assign true or false to vague information ``young". However, Fuzzy Logic is an answer to represent imprecise information. The sentence ``The patient is young." is true to some degree in unit interval $[0,1]$ - the lower the age of the patient, the more the sentence is true. Truth of Fuzzy Logic is a matter of degree that the element belongs to a fuzzy concept. The degree is in $[0,1]$, where $0$ means absolute ``False", $1$ means absolute ``True". Thus, the truth degrees is coded in real number between $0$ and $1$.

% 2. Fuzzy Logic Programming
Logic programming \cite{Llo87} has been successfully used in knowledge representation and reasoning for decades. Indeed, world data in not always perceived in a crisp way. Information that we gather might be imperfect, uncertain, or fuzzy in some other way. Hence the management of uncertainty and fuzziness is very important in knowledge representation.

Introducing Fuzzy Logic into Logic Programming has provided the development of several fuzzy systems over Prolog. These systems replace its inference mechanism, SLD-resolution, with a fuzzy variant that is able to handle partial truth. Most of these systems implement the fuzzy resolution introduced by Lee in \cite{Lee72}, as the Prolog-Elf system, the FRIL Prolog system \cite{IK85} and the F-Prolog language \cite{LL90}. However, there is no common method for fuzzifying Prolog, as noted in \cite{SDM89}.

One of the most promising fuzzy tools for Prolog is RFuzzy Framework. The most important advantages against the other approaches are:
\begin{enumerate}
\item A truth value is represented as a real number in unit interval $[0,1]$, satisfying certain constrains. 
\item A truth value is propagated through the rules by means of an aggregation operator. The definition of this aggregation operator is general and is subsumes conjunctive operators (triangular norms \cite{KPMRPE00} like min, prod, etc.), disjunctive operators \cite{TCC95} (triangular co-norms, like max, sum, etc.),
average operators (average as arithmetic average, quasi-linear average, etc.) and hybrid operators (combinations of the above operators \cite{PTC02}). 
\item Crisp and fuzzy reasoning are consistently combined
\item It provides some interesting improvements with respect to FLOPER \cite{MM08a,Mor06}: default values, partial default values, typed predicates and useful syntactic sugar (for presenting facts, rules and functions).
\end{enumerate}

%3. Fuzzy QA 
%Why do we need the webInterface
Fuzzy Query-Answering System (FQAS) is built on RFuzzy Framework, which inherits the advantages of RFuzzy Framework. Besides, \textit{quantification} expression is added into FQAS.
In order to query under RFuzzy Framework, we have to write a ciao prolog program, which requires the high level knowledge from users. FQAS has an user-friendly interface to overcome this disadvantage.

%What is the achievement of the project 
In Fuzzy Query-Answering System, it offers user to load crisp database, and to do configurations over crisp database, such as defining negations, quantifications, and fuzzy concepts. The simple or complex fuzzy queries can be generated by choosing negations, quantifications and fuzzy concepts which are defined during configuration. According to the functionality of FQAS, it is built into three modules, which are Loading File Module, Configuration Module and Query-Answering Module.
%4. The structure of Report

In this report, we first present RFuzzy Framework in section \ref{sec:Background}. In section \ref{sec:FuzzyQA}, Technical details of this project are shown, where three main modules Loading File Module, Configuration Module and Query-Answering Module are presented  in section \ref{sec:LoadingFileModule}, \ref{sec:ConfigurationModule} and \ref{sec:QAModule}, respectively. The Conclusions of Fuzzy Query-Answering System are given in the last section.



\end{document}