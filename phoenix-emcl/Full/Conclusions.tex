\documentclass[main.tex]{subfiles} 
\begin{document}
\newpage
\section{Conclusions}
\label{sec:Conclusions}
The goal of  Fuzzy Query-Answering System is providing a user-friendly webInterface to make fuzzy configuration and fuzzy query over crisp database. Fuzzy Configuration encapsulates crisp database into a virtual fuzzy database, where the information is described in an imprecise and human recognizable way. Fuzzy Query offers users to make queries in a subset of natural language, which includes negation and quantification. To achieve the reasoning of fuzzy information, FQAS interacts with RFuzzy Framework, where the reasoning works out. The result is sent back to FQAS from RFuzzy Framework.

\subsection{Disadvantages and Future work}
The Fuzzy Query-Answering system as a user-friendly web interface, there are other options can be improved to make the system more user-friendly.
\begin{enumerate}
\item Loading File Module (LFM)

At present, most database are relational database, they have their own standard to store tables and database schema. The parser in LFM can extend to deal with more different of formats.
 
\item Configuration Module (CM)

Adding or editing functions can be built in a graphic approach, which is more human recognizable and convenient comparing with choosing domain type and typing domain range and function expression.

\item Query-Answering Module (QAM)

The result also can be presented as a graph, which is easier to be analyzed and chosen by users. 

\end{enumerate}
\end{document}