\documentclass[12pt]{article}

\begin{document}
\begin{itemize}
\item poner sintaxis de intervalos de valores que se expanden en
  varios hechos. 
\item FRIL 
  \begin{itemize}
  \item Ejecutar Fril (Martes despues de comer)
  \item Seleccionar ejemplos para beanchmark. Idealmente que tengan el
  mismo comportamiento en los dos fuzzy.
  \item Realizar la comparaci�n. 
  \end{itemize}
\item Predicados Fuzzy b�sicos hasta ahora son con dominio reales
  continuos
  \begin{itemize} 
  \item Generalizar para discretos
  \item Dominios no numerios (Regtype) finitos e infinitos
  \end{itemize}
\item Bugs (conocidos y desconocidos)
  \begin{itemize}
  \item Comportamiento Modular ( declaraciones en itfs )
  \item Type Reduction (pasar de conjunto de intervalos a intervalos).
  \item (preguntar a Paco) meta calls en fuzzy.
  \end{itemize}
\item Prolog probabilistico (como extensi�n sint�ctica)
\item Articulos
  \begin{itemize}
  \item Corregir art�culo neg2.tex (Martes despues de comer)
  \item Fril (transformaciones e implementaci�n) conexion de
  implementaci�n con definici�n de semantica. Justificar la union de
  intervalos
  \item Te�rico. Justificar la union de intervalos. 
    \begin{itemize}
    \item Definir la teor�a (L-Loose sets) 
    \item Funciones de Agregaci�n: caso Algebra de Borel 
    \item Ejemplos que necesiten ser modelados con union de
    intervalos.
  \item Type Reduction.
    \end{itemize}
  \end{itemize}
\item Reunion con Manuel y Enrique 

\end{itemize}
\end{document}
