% -----------------------------------------------------------
% Brochure for Ciao 1.8 announcement (for SAS-LOPSTR-AGP 2002)
% -----------------------------------------------------------
\documentclass{article}

\usepackage{a4wide}
\usepackage{twocolumn}
\topmargin-1cm
\tolerance=10000

\makeatletter
\def\subsection{\@startsection{subsection}{2}{\z@}{-1.0ex plus -1ex minus 
 -.2ex}{0.7ex plus .2ex}{\normalsize\bf}}
\makeatother

\renewcommand{\baselinestretch}{0.95}
\newcommand{\myitems}{
\renewcommand{\baselinestretch}{0.5}
\parsep 0pt \parindent 0pt \itemsep -4pt \topsep -3pt }

%
% psfigTeX macros: combined TeX/LaTeX source
%
% Previous copyright below applies.  University of Maryland changes are
% Copyright (c) 1989 University of Maryland
% Department of Computer Science.  All rights reserved.
% Permission to copy these changes for any purpose is hereby granted
% so long as this copyright notice remains intact.
%
% psfig.tex and psfig.sty are derived automatically from this file,
% by including the appropriate one of either the TeX or LaTeX specific
% lines (marked with `%tex:' and `%latex').
%
% -------------------------------------------
%
% All software, documentation, and related files in this distribution of
% psfig/tex are Copyright (c) 1987 Trevor J. Darrell
%
% Permission is granted for use and non-profit distribution of psfig/tex
% providing that this notice be clearly maintained, but the right to
% distribute any portion of psfig/tex for profit or as part of any commercial
% product is specifically reserved for the author.
%
% Psfig/tex version 1.1
%
% $Header: /usr/src/local/tex/local/mctex/psfig/RCS/psfig.src,v 3.1 89/08/30 03:42:01 chris Exp $
% based on: psfig.tex,v 1.8 87/07/25 13:21:09 trevor Exp
%

\catcode`\@=11\relax
\newwrite\@unused
\def\typeout#1{{\let\protect\string\immediate\write\@unused{#1}}}

\typeout{psfig: version 1.1 (MC-TeX)}

\newdimen\ps@dim

%\def\psglobal#1{\ps@typeout
%  {psfig: including #1 globally}\immediate\special{ps:plotfile #1 global}}
%\def\psfiginit{\ps@typeout{psfiginit}\psglobal{/usr/lib/ps/figtex.pro}}

% @psdo control structure -- similar to Latex @for.
% I redefined these with different names so that psfig can
% be used with TeX as well as LaTeX, and so that it will not
% be vunerable to future changes in LaTeX's internal
% control structure.
%
\def\@nnil{\@nil}
\def\@empty{}
\def\@psdonoop#1\@@#2#3{}
\def\@psdo#1:=#2\do#3{\edef\@psdotmp{#2}\ifx\@psdotmp\@empty \else
\expandafter\@psdoloop#2,\@nil,\@nil\@@#1{#3}\fi}
\def\@psdoloop#1,#2,#3\@@#4#5{\def#4{#1}\ifx #4\@nnil \else
#5\def#4{#2}\ifx #4\@nnil \else#5\@ipsdoloop #3\@@#4{#5}\fi\fi}
\def\@ipsdoloop#1,#2\@@#3#4{\def#3{#1}\ifx #3\@nnil
\let\@nextwhile=\@psdonoop \else
#4\relax\let\@nextwhile=\@ipsdoloop\fi\@nextwhile#2\@@#3{#4}}

%
%

\def\psdraft{\def\@psdraft{0}}
\def\psfull{\def\@psdraft{100}}
\psfull

\def\ps@eat#1{}
\def\pssilent{\let\ps@typeout\ps@eat}
\def\psverbose{\let\ps@typeout\typeout}
\psverbose

\newif\if@prologfile
\newif\if@postlogfile
%%% These are for the option list.
%%% A specification of the form a = b maps to calling \@p@@sa{b}
\newif\if@bbllx
\newif\if@bblly
\newif\if@bburx
\newif\if@bbury
\newif\if@height
\newif\if@width
\newif\if@rheight
\newif\if@rwidth
\newif\if@clip
\def\@p@@sclip#1{\@cliptrue}
\def\@p@@sfile#1{\def\@p@sfile{#1}}
\def\@p@@sfigure#1{\def\@p@sfile{#1}}
\def\@p@@sbbllx#1{%\ps@typeout{bbllx is #1}
	\@bbllxtrue
	\ps@dim=#1
	\edef\@p@sbbllx{\number\ps@dim}
}
\def\@p@@sbblly#1{
	%\ps@typeout{bblly is #1}
	\@bbllytrue
	\ps@dim=#1
	\edef\@p@sbblly{\number\ps@dim}
}
\def\@p@@sbburx#1{
	%\ps@typeout{bburx is #1}
	\@bburxtrue
	\ps@dim=#1
	\edef\@p@sbburx{\number\ps@dim}
}
\def\@p@@sbbury#1{
	%\ps@typeout{bbury is #1}
	\@bburytrue
	\ps@dim=#1
	\edef\@p@sbbury{\number\ps@dim}
}
\def\@p@@sheight#1{
	\@heighttrue
	\ps@dim=#1
	\edef\@p@sheight{\number\ps@dim}
	%\ps@typeout{Height is \@p@sheight}
}
\def\@p@@swidth#1{
	%\ps@typeout{Width is #1}
	\@widthtrue
	\ps@dim=#1
	\edef\@p@swidth{\number\ps@dim}
}
\def\@p@@srheight#1{
	%\ps@typeout{Reserved height is #1}
	\@rheighttrue
	\ps@dim=#1
	\edef\@p@srheight{\number\ps@dim}
}
\def\@p@@srwidth#1{
	%\ps@typeout{Reserved width is #1}
	\@rwidthtrue
	\ps@dim=#1
	\edef\@p@srwidth{\number\ps@dim}
}
\def\@p@@sprolog#1{\@prologfiletrue\def\@prologfileval{#1}}
\def\@p@@spostlog#1{\@postlogfiletrue\def\@postlogfileval{#1}}
\def\@p@@ssilent#1{\pssilent}

\def\@cs@name#1{\csname #1\endcsname}
\def\@setparms#1=#2,{\@cs@name{@p@@s#1}{#2}}
%
% initialize the defaults (size the size of the figure)
%
\def\ps@init@parms{
	\@bbllxfalse \@bbllyfalse
	\@bburxfalse \@bburyfalse
	\@heightfalse \@widthfalse
	\@rheightfalse \@rwidthfalse
	\def\@p@sbbllx{}\def\@p@sbblly{}
	\def\@p@sbburx{}\def\@p@sbbury{}
	\def\@p@sheight{}\def\@p@swidth{}
	\def\@p@srheight{}\def\@p@srwidth{}
	\def\@p@sfile{}
	\def\@p@scost{10}
	\def\@sc{}
	\@prologfilefalse
	\@postlogfilefalse
	\@clipfalse
}
%
% Go through the options setting things up.
%
\def\parse@ps@parms#1{\@psdo\@psfiga:=#1\do{\expandafter\@setparms\@psfiga,}}

%
% Compute bb height and width
%
\newif\ifno@bb
\newif\ifnot@eof
\newread\ps@stream
\def\bb@missing{
	\ps@typeout{psfig: searching \@p@sfile \space  for bounding box}
	\openin\ps@stream=\@p@sfile
	\no@bbtrue
	\not@eoftrue
	\catcode`\%=12
	\loop
		\read\ps@stream to \line@in
		\global\toks200=\expandafter{\line@in}
		\ifeof\ps@stream \not@eoffalse \fi
		%\ps@typeout{looking at :: \the\toks200}
		\@bbtest{\toks200}
		\if@bbmatch\not@eoffalse\expandafter\bb@cull\the\toks200\fi
	\ifnot@eof \repeat
	\catcode`\%=14
}
\catcode`\%=12
\newif\if@bbmatch
\def\@bbtest#1{\expandafter\@a@\the#1%%BoundingBox:\@bbtest\@a@}
\long\def\@a@#1%%BoundingBox:#2#3\@a@{\ifx\@bbtest#2\@bbmatchfalse\else\@bbmatchtrue\fi}
\long\def\bb@cull#1 #2 #3 #4 #5 {
	\ps@dim=#2 bp\edef\@p@sbbllx{\number\ps@dim}
	\ps@dim=#3 bp\edef\@p@sbblly{\number\ps@dim}
	\ps@dim=#4 bp\edef\@p@sbburx{\number\ps@dim}
	\ps@dim=#5 bp\edef\@p@sbbury{\number\ps@dim}
	\no@bbfalse
}
\catcode`\%=14
%
\def\compute@bb{
	\no@bbfalse
	\if@bbllx \else \no@bbtrue \fi
	\if@bblly \else \no@bbtrue \fi
	\if@bburx \else \no@bbtrue \fi
	\if@bbury \else \no@bbtrue \fi
	\ifno@bb \bb@missing \fi
	\ifno@bb \typeout{FATAL ERROR: no bb supplied or found}
		\no-bb-error
	\fi
	%
	\count203=\@p@sbburx
	\count204=\@p@sbbury
	\advance\count203 by -\@p@sbbllx
	\advance\count204 by -\@p@sbblly
	\edef\@bbw{\number\count203}
	\edef\@bbh{\number\count204}
	%\ps@typeout{bbh = \@bbh, bbw = \@bbw}
}
%
% \in@hundreds performs #1 * (#2 / #3) correct to the hundreds,
%	then leaves the result in @result
%
\def\in@hundreds#1#2#3{\count240=#2 \count241=#3
	\count100=\count240	% 100 is first digit #2/#3
	\divide\count100 by \count241
	\count101=\count100
	\multiply\count101 by \count241
	\advance\count240 by -\count101
	\multiply\count240 by 10
	\count101=\count240	%101 is second digit of #2/#3
	\divide\count101 by \count241
	\count102=\count101
	\multiply\count102 by \count241
	\advance\count240 by -\count102
	\multiply\count240 by 10
	\count102=\count240	% 102 is the third digit
	\divide\count102 by \count241
	\count200=#1\count205=0
	\count201=\count200
		\multiply\count201 by \count100
	 	\advance\count205 by \count201
	\count201=\count200
		\divide\count201 by 10
		\multiply\count201 by \count101
		\advance\count205 by \count201
		%
	\count201=\count200
		\divide\count201 by 100
		\multiply\count201 by \count102
		\advance\count205 by \count201
		%
	\edef\@result{\number\count205}
}
\def\compute@wfromh{
	% computing : width = height * (bbw / bbh)
	\in@hundreds{\@p@sheight}{\@bbw}{\@bbh}
	%\ps@typeout{ \@p@sheight * \@bbw / \@bbh, = \@result }
	\edef\@p@swidth{\@result}
	%\ps@typeout{w from h: width is \@p@swidth}
}
\def\compute@hfromw{
	% computing : height = width * (bbh / bbw)
	\in@hundreds{\@p@swidth}{\@bbh}{\@bbw}
	%\ps@typeout{ \@p@swidth * \@bbh / \@bbw = \@result }
	\edef\@p@sheight{\@result}
	%\ps@typeout{h from w: height is \@p@sheight}
}
\def\compute@handw{
	\if@height \if@width \else \compute@wfromh \fi
	\else \if@width \compute@hfromw \else
		\edef\@p@sheight{\@bbh}
		\edef\@p@swidth{\@bbw} \fi
	\fi
}
\def\compute@resv{
	\if@rheight \else \edef\@p@srheight{\@p@sheight} \fi
	\if@rwidth \else \edef\@p@srwidth{\@p@swidth} \fi
}
%
% Compute any missing values
\def\compute@sizes{
	\compute@bb
	\compute@handw
	\compute@resv
}
%
% \psfig
% usage : \psfig{file=, height=, width=, bbllx=, bblly=, bburx=, bbury=,
%			rheight=, rwidth=, clip=}
%
% "clip=" is a switch and takes no value, but the `=' must be preset.
\def\psfig#1{\vbox{
    \ps@init@parms
    \parse@ps@parms{#1}
    \compute@sizes
    %
    \ifnum\@p@scost<\@psdraft
	\ps@typeout{psfig: including \@p@sfile}
	%
	\special{ps::[begin] \@p@swidth \space \@p@sheight \space
		\@p@sbbllx \space \@p@sbblly \space
		\@p@sbburx \space \@p@sbbury \space
		startTexFig \space}
	\if@clip
		\ps@typeout{(clip)}
		\special{ps:: \@p@sbbllx \space \@p@sbblly \space
			\@p@sbburx \space \@p@sbbury \space
			doclip \space}
	\fi
	\if@prologfile \special{ps: plotfile \@prologfileval \space} \fi
	\special{ps: plotfile \@p@sfile \space}
	\if@postlogfile \special{ps: plotfile \@postlogfileval \space} \fi
	\special{ps::[end] endTexFig \space}
	% Create the vbox to reserve the space for the figure
	\vbox to\@p@srheight true sp{\hbox to\@p@srwidth true sp{\hfil}\vfil}
    \else
	% draft figure, just reserve the space and print the
	% path name.
	\vbox to\@p@srheight true sp{\vss
	    \hbox to\@p@srwidth true sp{\hss\@p@sfile\hss}\vss}
    \fi
}}

\catcode`\@=12\relax


\input{/home/clip/Papers/clip_description/logo.tex}
\newcommand{\ciao}{\psfig{figure=ciao_s.ps,width=2.5cm}}

\begin{document}

\pagestyle{empty}

\twocolumn[
\noindent
\hbox{
\begin{Huge}
{\bf The 
\begin{minipage}[c]{0.3\textwidth}
\ciao
\end{minipage}
1.8 Prolog System}\\
\end{Huge}
%\engcliplogo
}
\vspace{0.5\baselineskip}
]

%\twocolumn
%%GP Reworded
%% Ciao is a \emph{public domain}, \emph{next generation} logic
%% programming environment.
Ciao is a \emph{next generation} logic programming environment. Ciao is
\emph{free software} distributed under GNU licenses.
%%EndGP

Ciao is a complete Prolog system, supporting \emph{ISO-Prolog}, but
its novel modular design allows both \emph{restricting} and
\emph{extending} the language. This makes it possible to work with
\emph{fully declarative subsets} of Prolog and also to \emph{extend}
these subsets (or ISO-Prolog) both syntactically and semantically.
Most importantly, these restrictions and extensions can be activated
separately on each program module so that several extensions can 
coexist in the same application for different modules.

Ciao currently includes extensions for feature terms (records),
functional logic programming, higher-order (with predicate
abstractions), constraints, object-oriented programming, persistence,
several control rules (breadth-first search, iterative deepening,
...), concurrency (threads/engines), distributed execution (agents),
concurrent built-in database, and parallel execution.  Libraries also
support WWW programming, sockets, external interfaces (C, Java, TclTk,
relational databases, etc.), etc.

\emph{Programming in the large} is supported thanks to the robust
module/class system, module-based automatic incremental compilation
(with no need for makefiles), an assertion language for declaring
(\emph{optional}) program properties (including types, modes,
determinacy, non-failure, cost, ...), automatic static inference and
static/dynamic checking of such assertions, program optimization and
parallelization, and powerful automatic documentation generation (the
latter tasks performed by the \emph{CiaoPP preprocessor} and the
\emph{LPdoc autodocumenter}).  It is the novel modular design of Ciao
which enables modular program development, effective global program
analysis, and static debugging and optimization via source to source
program transformation.

\emph{Programming in the small} is supported by having reduced size
executables, which only include those builtins and libraries used by
the program, and by supporting
% \emph{scripts} written in Prolog.
Prolog \emph{scripts}.

The compiler generates several forms of architecture-independent and
stand-alone executables. Program modules can be compiled into compact
bytecode and linked statically, dynamically, or autoloaded. The
executables generated are very competitive in both performance and
size with all current commercial and academic Prolog systems.

The programming environment also offers a rich \texttt{emacs}
interface (with direct access to top-level/debugger, preprocessor,
and autodocumenter), embeddable source-level debugger with
breakpoints, and some execution visualization tools. 

\subsection*{Main Features:}
\begin{itemize}
\myitems
% \item Stand-alone executables available for UNIX-like o.s. 
\item Multi-architecture executables
\item Emacs interface \& source-level debugger.
% \item Emacs environment improved, added menus for Ciaopp and LPDoc.
\item Debugger embeddable in executables.
%jcf? \item Improved installation/deinstallation.
%jcf \item Many improvements to autodocumenter.
\item Source code autodocumenter (with a menu-based interface).
%jcf \item Threads now available in Win32.
%jcf \item Many improvements to threads.
\item New generation, robust module system.
\item Modular clp(R) / clp(Q).
\item Libraries for and-fair breadth-first and iterative deepening.
%jcf \item Improved syntax for pred.~abstractions.
%jcf \item Support for {\em predicate abstractions}.
\item Higher-order syntax for predicate abstractions.
%jcf
\item Library of higher-order list predicates.
%jcf \item Better code expansion facilities (macros).
\item Extensive, built in, and modular code expansion facilities
  (macros). 
%jcf \item Other new libraries. 
%item \item New  (\&Prolog-like) concurrency / multiengine
%primitives. 
\item (\&Prolog-like) concurrency / multiengine primitives. 
\item Full thread support in Linux / Unix / Mac OS X / Win32.
%jcf \item New delay predicates (when/2).
\item Delay predicates (when/2).
\item Delayed goals with freeze restored.
\item Rich set of system libraries. 
%\item New version of O'Ciao objects, 
%      with improved performance. 
\item Object oriented extension.
\item C, Java, SQL, TclTk interfaces.
\item Web/Internet programming: the \emph{PiLLoW} library.
\item Ciao CGI excutables under IIS.
\item Persistent logic database.
%jcf \item The size of atoms is now unbound.
\item Unbound atom size.
\item Fast creation of new unique atoms.
\item \#~of clauses/pred.~essentially unbound.
%jcf \item Much faster fast write/read. 
\item Fast writing/reading (marshalling and unmarshalling) of terms.
\item Compressed object code/executables. % on demand. 
%jcf \item Faster compilation and startup.
\item Fast compilation and startup.
\item Incremental stand-alone compiler.
\item Automatic (re)compilation of foreign files.
\item Extensive, up to date documentation.
% \item Complete documentation.
%jcf? \item Precedence of importations changed: last one is now higher. 
%jcf? \item Modules can implicitly export all preds.
%jcf? \item Improved documentation.

\end{itemize}

\subsection*{Contact / download info:}

%%GP
%% \texttt{http://www.clip.dia.fi.upm.es}\\ 
%% \texttt{http://www.clip.dia.fi.upm.es/Software}\\ 
%% \texttt{clip@clip.dia.fi.upm.es}\\ 
%% Manuel Hermenegildo / The CLIP Group\\
%% Facultad de Inform\'{a}tica -- UPM\\
%% E-28660 Boadilla del Monte, Madrid, SPAIN
\texttt{http://cliplab.org}\\ 
\texttt{http://ciaoprolog.org}\\ 
\texttt{ciao@clip.dia.fi.upm.es}\\ 
The CLIP Group\\
Facultad de Inform\'{a}tica -- UPM\\
E-28660 Boadilla del Monte, Madrid, SPAIN
%%EndGP

\end{document}
