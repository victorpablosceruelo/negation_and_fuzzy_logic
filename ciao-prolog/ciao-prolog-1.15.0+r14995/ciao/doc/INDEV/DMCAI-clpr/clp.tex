% -*-texinfo-*-
\raggedbottom
\input texinfo

@comment %**start of header (This is for running Texinfo on a region.)
@setfilename clp.info
@settitle DMCAI CLP 1.2
@c synindex tp fn
@c footnotestyle separate
@c paragraphindent 0
@c setchapternewpage odd
@comment %**end of header (This is for running Texinfo on a region.)
@c @smallbook

@c say that this is not a draft
@iftex
@finalout
@end iftex

@ifinfo
This file documents a variant of SICStus Prolog, featuring extensible
unification. 

Copyright @copyright{} 1992 DMCAI.  All rights reserved.
@end ifinfo

@tex
% LaTeX "sloppy"
\global\tolerance=10000
\global\hfuzz=.5pt
\global\vfuzz=.5pt

% Adjust \hoffset for A4 paper

\global\vsize 25.6cm
\global\pageheight=\vsize
\global\hoffset-0.125cm
@end tex

@titlepage
@c  use the new format for titles

@title DMCAI CLP 1.2
@subtitle This Manual documents a variant of SICStus Prolog version 2.1,
@subtitle featuring extensible unification
@subtitle
@subtitle Edition 1.2
@subtitle March 1992

@author by Christian Holzbaur

@comment   Include the Distribution inside the titlepage so
@c that headings are turned off.

@page
@vskip 0pt plus 1filll
Copyright @copyright{} 1992 DMCAI

Department of Medical Cybernetics and Artificial Intelligence @*
University of Vienna @*
Freyung 6 @*
A-1010 Vienna, Austria


Permission is granted to make and distribute verbatim copies of
this manual provided the copyright notice and this permission notice
are preserved on all copies.

Permission is granted to copy and distribute modified versions of this
manual under the conditions for verbatim copying, provided that the entire
resulting derived work is distributed under the terms of a permission
notice identical to this one.

Permission is granted to copy and distribute translations of this manual
into another language, under the above conditions for modified versions,
except that this permission notice may be stated in a translation approved
by the DMCAI.
@end titlepage
@page

@node Top,            Intro,         , (DIR)
@comment  node-name,  next,  previous,  up

@node Intro, Basic Predicates, Top, Top
@comment  node-name,  next,  previous,  up
@chapter Introduction
@cindex Unification

Within the version of SICStus Prolog @cite{[Carlsson & Widen 90]}
described in this manual, the unification mechanism has been changed in
such a way that the user (you) may introduce interpreted terms and
specify their unification through Prolog predicates.
@cindex Constraint Logic Programming (CLP)
@cindex CLP
Extensible unification in turn, aims at the implementation of instances
of the general CLP scheme @cite{[Jaffar & Michaylov 87]}.

The basic mechanism that was utilized to provide extensible unification
in the SICStus Prolog implementation are so-called @dfn{attributed
variables}.
@cindex Attributed Variables
@quotation
@cindex Meta-structures
If the reader is familiar with @dfn{meta-structures} @cite{[Neumerkel
90]}, that likewise aim at the provision of extensible unification, s/he
is referred to @xref{Meta-structures}, where we briefly compare the two
approaches.
@end quotation

Attributed variables are variables with an associated attribute, which
is a term.  Attributes are attached to variables, and attributes are
referred to, through built-in predicates. As far as the rest of a given
Prolog implementation is concerned, attributed variables behave like
variables --- they can be considered as a subtype of type variable. The
indexing mechanism treats variables and attributed variables the same
way.  Built-in predicates observe attributed variables as if they were
ordinary variables. Special treatment for attributed variables applies
in the following situations:
@itemize @bullet

@item 
During unification. This is the most important difference! When an attributed
variable is to be unified with another attributed variable or some
other non-variable term, user-defined predicates specify how this unification
has to be performed.

@item 
When printed via @code{print/1}, a user-supplied predicate gets a chance
to print the attributed variable in some customized fashion.

@item 
During memory management, i.e, during garbage collection, @dfn{early
reset} and @dfn{variable shunting} are performed on attributed (and
other) variables.
@end itemize

SICStus2.1 did not provide attributed variables directly, but within the
kernel this data type was used to implement @code{freeze/2} and
@code{dif/2} and other coroutining facilities. Although the data type
@dfn{attributed variable} (variables with tag @code{CVA} for kernel
hackers) could be utilized directly, the @dfn{wake mechanism} had to be
changed for the new purpose. The wake mechanism is concerned with the
treatment of attributed variables during unifications. The mechanism
built into the original SICStus2.1 release is sufficient for the
implementation of the mentioned coroutining mechanisms. For our CLP
applications however, it had to be changed somewhat. If you are mainly
interested in CLP programming, you don't want to know what exactly the
changes are (you can always look at the code). An important consequence
of the modifications is that the original coroutining mechanisms do not
work in our SICStus clone. This is only half as bad as it sounds because
explicit Prolog versions of comparable mechanisms, implemented via the
new CLP scheme, will be provided.  The @strong{explicit} encoding of,
say @code{freeze/2}, is not to be understood as a mere substitute for
the original C code. The true motivation for explicit encodings is that
it enables the user to combine, say @code{freeze/2}, with other
instances of the general CLP scheme like CLP(R).

@node Basic predicates, Meta-structures, Intro, Top    
@comment  node-name,  next,  previous,  up
@chapter Manipulation of attributed variables
The following predicates provide for the introduction, detection, and
manipulation of attributed variables.

@table @code
@item get_attribute(@var{X},@var{C})
@findex get_attribute/2
If @var{X} is an attributed variable, unify the corresponding attribute 
with @var{C}, otherwise the predicate fails.


@item attach_attribute(@var{X},@var{C})
@findex attach_attribute/2
Turn the free variable @var{X} into an attributed with attribute @var{C}.
@var{C} must @strong{not} be a variable. Note that attaching an attribute
to variable changes the identity of the variable. Example:

@example
| ?- write(X),nl,attach_attribute(X,prime),write(X),nl.
_40
_219
@end example

@noindent
In terms of Logic, this change of identity does no harm as it takes place
consistently. That is, all occurrences of the affected variable get the new
identity. If some application depends on the ordering of terms, and variables
in particular, attaching attributes to variables might invalidate the ordering
of ordered data structures.


@item detach_attribute(@var{X})
@findex detach_attribute/1
Remove the attribute from an attributed variable, turning it into a free
variable. Note that detaching the attribute from a variable changes the
identity of the variable. Example:

@example
| ?- attach_attribute(X,prime),write(X),nl,
     detach_attribute(X),write(X),nl.
_328
_337
@end example

@noindent
In terms of Logic, this change of identity does no harm as it takes
place consistently. That is, all occurrences of the affected variable get
the new identity. If some application depends on the ordering of terms,
and variables in particular, detaching attributes from variables might
invalidate the ordering of ordered data structures.


@item update_attribute(@var{X},@var{C})
@findex update_attribute/2
Change the attribute of the attributed variable @var{X} to @var{C}.
Acts as if defined below, but might be more (memory) efficient. @var{C}
must @strong{not} be a variable. Note that updating the attribute of an
attributed variable changes the identity of the variable.

@example
update_attribute( X, C) :-
  detach_attribute( X),
  attach_attribute( X, C).
@end example
@end table

@node Meta-structures, Source transformations, Basic predicates, Top
@comment  node-name,  next,  previous,  up
@chapter Attributed variables vs. Meta-structures
@cindex Meta-structures
In @cite{[Holzbaur 90]} meta-structures on top of C-Prolog were applied
for the specification and implementation of a variety of instances of
the CLP scheme. The quality and the availability of SICStus Prolog
motivated the migration to SICStus Prolog, entailing a compiler, garbage
collector, and better arithmetics. In order to keep the implementation
efforts within reasonable bounds, we decided to provide extensible
unification on top of SICStus Prolog through the use of attributed
variables as described above.  Although attributed variables and
meta-structures share some properties, there are differences.

@quotation
Meta-structures are ordinary, non-variable Prolog terms with the sole
difference that they can be detected as members of this special sort.
Meta-structures are introduced by a declaration @code{:- meta_functor
N/A}, where @code{N/A} denotes any functor.  The behavior of
meta-structures during unifications can be specified precisely through a
Prolog meta interpreter which makes unification explicit, and makes some
further conventions integral parts of the specification @cite{[Holzbaur
90]}.

The implementations of many CLP instances require the functionality that
is achieved with destructive updates in traditional, procedural
realizations.  In logic programming, this functionality is provided in a
sound fashion by either copying or @dfn{modifying by variable
substitution}.  The latter option can be applied to meta-structures
through the convention that one particular argument of the structure is
a free variable, which will eventually be bound to another
meta-structure, obeying the same convention.  Therefore, sequences of
modifications lead to meta-structure chains. The current 'value' of a
meta-structure is to be found at the end of the chain.  Traversing this
chains could of course be left to the user, but it is so common a
pattern, that is has been made part of the specification.
@end quotation

@noindent
Please compare the implementation of @code{freeze/2} via
meta-structures with the one via attributed variables:

@example
:- meta_functor( frozen/2). 
   
freeze( frozen(_,Goal), Goal).
   
meta_term_unify( frozen(Value,Goal), Value) :- 
  call( Goal).

meta_meta_unify( frozen(V,G1), frozen(V,G2)) :- 
  V = frozen(_,(G1,G2)).
@end example

@noindent
The above encoding of @code{freeze/2} with meta-structures assumes the
specification from @cite{[Holzbaur 90]} being in force.
Within the framework documented in this manual, @code{freeze/2} can be
coded like this:

@example
freeze( X, Goal) :-
  attach_attribute( V, frozen(V,Goal)),
  X = V.

verify_attribute( frozen(Var,Goal), Value) :-
  detach_attribute( Var),
  Var = Value, 
  call(Goal).

combine_attributes( frozen(V1,G1), frozen(V2,G2)) :-
  detach_attribute( V1),
  detach_attribute( V2),
  V1 = V2,
  attach_attribute( V1, frozen(V1,(G1,G2))).
@end example

@node Source transformations,  Unification of attributed variables , Meta-structures, Top    
@comment  node-name,  next,  previous,  up
@section Source transformations
The most significant difference between meta-structures and attributed
variables is that interpreted terms can be represented directly through
meta-structures. When we realize interpreted terms via attributed
variables, we have to replace interpreted terms through attributed
variables with the interpreted terms as their attributes, say.
For a given CLP instance with a given set of interpreted functors, one
would typically use the @code{term_expansion/2} mechanism to perform
this transformation. The same transformation applies to queries.
In order to enable the user to toggle between a Prolog system which
recognizes interpreted functors and one that does not, we provide an
alternative toplevel, @pxref{Toplevel}.

@node Unification of attributed variables, User-defined predicates, Source transformations, Top   
@comment  node-name,  next,  previous,  up
@chapter Unification in the presence of attributed variables
Once attributed variables have been created, they have to be dealt with
during unification. The following procedural description lists the 
sequence of events from the point of view of the @dfn{WAM}. 

@itemize @bullet
@item
A unification between an unbound variable and an attributed variable
binds the unbound variable to the attributed variable.

@item 
When an @emph{attributed variable} is about to be bound during
unification because it is equated with a non-variable Prolog term or
another attributed variable, the attributed variable and the value it
should be bound to are recorded in some internal data structure.

@item 
If there is more than one binding event for attributed variables between
two inference steps, a list of attributed variable-value pairs is
collected in some internal data structure.

@item 
At the next inference step, the abstract machine takes measures to feed
the attributed variable-value pairs to user-supplied predicates. The
data structures for the representation of the list of variable-value
pairs can be reclaimed at this point.
@end itemize

@node User-defined predicates, Unification, Unification of attributed variables, Top    
@comment  node-name,  next,  previous,  up
@section User-defined predicates
The following predicates have to be supplied by the user. They
specify the behavior, i.e., the meaning of attributed variables during
unification.


@node Unification, Printing, User-defined predicates, Top   
@comment  node-name,  next,  previous,  up
@subsection Specifying unification for interpreted terms
@table @code
@item verify_attribute(@var{C},@var{T})
@findex verify_attribute/2
This predicate is called when an attributed
variable with attribute @var{C} is about to be unified with the
non-variable term @var{T}.

@item combine_attributes(@var{C1},@var{C2})
@findex combine_attributes/2
This predicate is called when two attributed variables with attributes
@var{C1,C2} are about to be unified.
@end table

Note that the two predicates are are not called with the attributed
variables involved, but with the corresponding attributes instead.  The
reasons are:

@itemize -
@item 
There are simple applications which only refer to the attributes.

@item 
If the application wants to refer to the attributed variables
themselves, they can be made part the attribute term.  The
implementation of @code{freeze/2} (@xref{Meta-structures}) utilizes this
technique.  Note that this does not lead to cyclic structures, as the
connection between an attributed variable and it's attribute is
invisible to the pure parts of the Prolog implementation.
    
@item 
If attributed variables were passed as arguments, the user's code would
have to refer to the attributes through an extra call to
@code{get_attribute/2}.

@item 
As the/one attribute is the first argument to each of the two
predicates, indexing applies. Note that attributed variables themselves
look like variables to the indexing mechanism.
@end itemize

@node Printing, Dumping, Unification, User-defined predicates    
@comment  node-name,  next,  previous,  up
@subsection Printing
In order to gain control over the printing of attributed variables, the
user may define the predicate:

@table @code
@item portray_attribute(@var{A},@var{V})
@findex portray_attribute/2
The predicate should either print something based on @var{A} or @var{V},
or do nothing and fail. In the latter case, the default printer
(@code{write/1}) will print the attributed variable like an unbound
variable, e.g. @samp{_673}.  @var{A} is the attribute of the attributed
variable @var{V}.
@end table

@node Dumping, , Printing, User-defined predicates    
@comment  node-name,  next,  previous,  up
@subsection Dumping
Also related to the presentation of the results of computations is the 
user-supplied predicate:

@table @code
@item dump(@var{Term},@var{Copy},@var{Constraints})
@findex dump/3
@code{dump/3} is intended to do almost the same as @code{copy_term/2} (a
SICStus builtin predicate), with the difference that besides a copy a
@strong{list} with some representation of the constraints associated
with the variables in the copy is produced. @var{Copy} and
@var{Constraints} must not contain any attributed variables. Speaking in
global terms, this predicate should produce a @dfn{domain specific
representation} of the constraints.
@cindex Domain specific representation
Because @var{Copy} and @var{Constraints} do not contain attributed
variables, these terms are @strong{pure} Prolog objects and can dealt
with correspondingly. You might for example want to assert or retract
constraints or process them in some other way with @strong{pure} Prolog
programs that don't know about attributed variables.

The new toplevel supplied with this patch tries to call @code{dump/3} to
present the answer constraints. @xref{Toplevel}. If it fails, some
domain independent representation of the constraints, i.e. attributed
variables, gets printed instead.
@end table


@node Toplevel, Random extensions, User-defined predicates, Top    
@comment  node-name,  next,  previous,  up
@chapter New toplevel
@cindex Toplevel interaction
The new toplevel is needed because the original one does not know about
the new uses of attributed variables. Example:

@example
| ?- attach_attribute(X,prime).

prolog:prime
@end example

@noindent
The printout produced by the original toplevel is pretty useless.
The new one shows us the relation between @var{X} and its attribute:

@example
top.
[Clp] ?- attach_attribute(X,prime).

attach_attribute(X,prime)
@end example

@noindent
If not customized further via @code{dump/3} (@xref{Dumping}.), this is
nothing but the description for the reconstruction of the answer
constraints. This is all we can expect in the general case.  A suitable
definition of @code{dump/3} will be able to produce the following
'theory' specific result:

@example
top.
[Clp] ?- attach_attribute(X,prime).

prime(X)
@end example

@table @code
@item top
@findex top/0
Calling @code{top/0} enters a toplevel that differs from the original
one in the following ways:

@itemize -
@item
The prompt @code{[Clp] ?-} indicates that the new toplevel is executing.
In fact, the prompt @var{P} can be set via
@code{unset(prompt(_),set(prompt(@var{P}))}. @xref{Flags}.

@item
@cindex Answer constraints
Answer substitutions are presented in the usual way. @dfn{Answer
constraints} are extracted via @code{dump/3} and printed.

@item 
Once entered, the flag @code{expansion(clp)} (@xref{Flags}.) is set and
remains set until the toplevel is left when @code{end_of_file} is read
from the current input stream. This flag enables @code{term_expansion/2}
to 'know' whether the expansion has to deal with interpreted functors or
not.
@end itemize
@end table

@noindent 
The new toplevel catches exceptions. When left via @code{abort},
the flag indicating the expansion mode is not cleared. In this
hopefully rare cases, the flag has to be cleared manually with
@code{unset(expansion(_))}.

@node Random extensions, Flags, Toplevel, Top    
@comment  node-name,  next,  previous,  up
@chapter Other additions to SICStus2.1
Although not directly related to the general CLP scheme provided with 
this patch, some utility predicates have been added also.

@node Flags, Numerical functions, , Random extensions
@comment  node-name,  next,  previous,  up
@section Global flags
A simple mechanism for @dfn{global flags} is provided:

@table @code
@item set(@var{Flag})
Sets @var{Flag} which can be any Prolog term.
@findex set/1

@item unset(@var{Flag})
Clears all flags that unify with @var{Flag}.
@findex unset/1

@item setting(@var{Flag})
@findex setting/1
Nondeterministically unifies @var{Flag} with any flag set via
@code{set/1}.
@end table

@node Numerical functions, Type recognition, Flags, Random extensions    
@comment  node-name,  next,  previous,  up
@section Numerical functions

@table @code
@item gcd(@var{A},@var{B})
@findex gcd/2
The numerical function @code{gcd/2} has been added to the set of
functions recognized by @code{is/2} and the predicates that perform
numerical comparisons. The function computes the greatest common divisor
of the two integer arguments @var{A} and @var{B}. Non-integer arguments
rise a condition. Example:

@example
| ?-  X is gcd( 2376446732, -23764).

X = 4 
@end example
@end table

@noindent
The motivation for the existence of this predicate is the absence of
arithmetic functions and comparison predicates for rational numbers in
SICStus2.1. Writing a rational evaluator in Prolog is easy, but
efficiency is awful if @code{gcd/2} has to be coded in Prolog. Once
SICStus reaches the numerical maturity of Commonlisp, patches like this
one will be obsolete.

@node Type recognition, , Numerical functions, Random extensions
@comment  node-name,  next,  previous,  up
@section Type recognition predicate
@table @code
@item type(@var{X},@var{Type})
@findex type/2
The meta-logical predicate determines the type of @var{X} and unifies
@var{Type} with one of the atoms @code{var}, @code{cva}, @code{float},
@code{integer}, @code{structure}, @code{atom}, @code{list}. @code{cva}
means that @var{X} is an attributed variable.  The remaining type names
should be self explaining.  The motivation for the existence of this
predicate is that indexing in SICStus2.1 is not exhaustive on this list
of types.  With the help of this predicate we can write code like below,
in order to get the benefits of indexing.

@example
sample( X) :-
  type( X, Type),
  sample( Type, X).

  sample( var,     X) :- format( "~p is a variable", [X]).
  sample( atom,    X) :- format( "~p is an atom", [X]).
  sample( integer, X) :- format( "~p is an integer", [X]).
  sample( list,    X) :- format( "~p is a list", [X]).
@end example
@end table

@node References
@comment  node-name,  next,  previous,  up
@unnumbered References

@table @cite
@item [Carlsson & Widen 90]
Carlsson M., Widen J.: Sicstus Prolog Users Manual, Swedish Institute of
Computer Science, SICS/R-88/88007C, 1990.

@item [Holzbaur 90]
Holzbaur C.: Specification of Constraint Based Inference Mechanisms through
Extended Unification, Dept. of Medical Cybernetics & AI, University of Vienna,
Dissertation, 1990.

@item [Jaffar & Michaylov 87]
Jaffar J., Michaylov S.: Methodology and Implementation of a CLP System, in
Lassez J.L.(ed.), Logic Programming - Proceedings of the 4th
International Conference - Volume 1, MIT Press, Cambridge, MA, 1987.
 
@item [Neumerkel 90]
Neumerkel U.: Extensible Unification by Metastructures, Proc. META90, 1990.
@end table

@node A complete example    
@comment  node-name,  next,  previous,  up
@appendix complete example
This example illustrates the realization of a very simple type system
with the DMCAI CLP package. The types we want to deal with are
@code{odd} and @code{even}. They partition the Prolog type @code{integer} 
into two disjoint sets. The syntax for associating a type with a term
shall be @code{odd(X)} and @code{even(X)} respectively. Therefore,
@code{odd/1} and @code{even/1} are our interpreted functors.

@node Sample session    
@comment  node-name,  next,  previous,  up
@appendixsec Sample session

@example
SICStus 2.1 #4 [DMCAI Clp]: Mon Mar 23 18:27:18 MET 1992
| ?- use_module(types).
{compiling types.pl...}
{compiling settings.pl...}
{settings.pl compiled in module settings, 117 msec 10544 bytes}
{compiling clp.pl...}
{compiling dump.pl...}
{dump.pl compiled in module clp, 1117 msec 5616 bytes}
{compiling toplevel.pl...}
{toplevel.pl compiled in module clp, 917 msec 5888 bytes}
{clp.pl compiled in module clp, 2200 msec 20320 bytes}
{types.pl compiled, 3884 msec 40720 bytes}

yes
| ?- top.
[Clp] ?- f(odd(X),even(Y)) = Z.

Z = f(X,Y),
even(Y),
odd(X) ? ;

no
[Clp] ?- f(odd(X),even(Y)) = Z, Z = f(13,_).

X = 13,
Z = f(13,Y),
even(Y)
@end example

@node Necessary definitions    
@comment  node-name,  next,  previous,  up
@appendixsec Necessary definitions
We start the example by importing the new toplevel (@xref{Toplevel}).
@example
     1 %
     2 % Simple type theory to demonstrate some CLP basics
     3 %
     4	     
     5 :- use_module( settings).
     6 :- use_module( clp).
@end example

@noindent
Next, we specify the source transformations that replace interpreted
terms with attributed variables (@xref{Source transformations}.):

@example
     1	term_expansion( A, B) :-
     2	  setting( expansion( clp)),
     3	  our_term_expansion( A, B).
     4	       
     5	our_term_expansion( X,       X) :- var( X), !.
     6	our_term_expansion( odd(X),  New) :- !, 
     7	  attach_attribute( New, odd(New)),  
     8	  New = X.
     9	our_term_expansion( even(X), New) :- !, 
    10	  attach_attribute( New, even(New)), 
    11	  New = X.
    12	our_term_expansion( Old,     New) :- 
    13	  functor( Old, N, A),
    14	  functor( New, N, A),
    15	  our_term_expansion( A, Old, New).
    16	
    17	our_term_expansion( 0, _,   _) :- !.
    18	our_term_expansion( N, Old, New) :-
    19	  arg( N, Old, Oa),
    20	  arg( N, New, Na),
    21	  our_term_expansion( Oa, Na),
    22	  N1 is N-1,
    23	  our_term_expansion( N1, Old, New).
@end example

@noindent
In line 2 above, we test whether our interpreted functors should be
interpreted indeed. The actual transformation takes place in lines 6 to
11. You might wonder why the code in lines 6--8 is not just:

@example
     1	our_term_expansion( odd(X),  X) :- !, 
     2	  attach_attribute( X, odd(X)).
@end example

@noindent
Well, this is because we want the term @code{odd(33)}, which is
equivalent to @code{33} in our 'theory', to be translated correctly.
Therefore we introduce a fresh attributed variable @var{New} (lines 7
and 10) and unify it with @var{X} (lines 8 and 11). Note that in our
example the transformation of @code{odd(33)} yields @code{33} after the
call to @code{verify_attribute/2} (@xref{Unification}), triggered by the
unification in line 8, has verified the 'oddness' of 33. The rest of the
transformation is straightforward.

@noindent
Now we specify how the interpreted terms are to be unified with the pure
Prolog data types (lines 1 to 10) and with each other (lines 12 to 17).

@example
     1	verify_attribute( odd(X), Term) :-
     2	  integer( Term),
     3	  0 =\= Term mod 2,
     4	  detach_attribute( X),
     5	  X = Term.   
     6	verify_attribute( even(X), Term) :-
     7	  integer( Term),
     8	  0 =:= Term mod 2,
     9	  detach_attribute( X),
    10	  X = Term.
    11	
    12	combine_attributes( even(X), even(Y)) :-
    13	  detach_attribute( X),
    14	  X = Y.   
    15	combine_attributes( odd(X),  odd(Y)) :-
    16	  detach_attribute( X),
    17	  X = Y.
@end example

@noindent
The only Prolog data type compatible with our types are the integer
numbers (lines 2 and 7). @code{Odd} and @code{even} have the usual
semantics (lines 3 and 8). If an integer is compatible with the type of
@var{X}, we detach the attribute and unify @var{X} which is a free
variable with the integer @var{Term} (lines 5 and 10).  All other
attempted unifications between @code{odd} and @code{even} typed
variables and the remaining Prolog data types fail as they are not
covered by @code{verify_attribute/2}.

@noindent
Unifications between typed variables are dealt with by
@code{combine_attributes}. Both variables must be of the same type
(lines 12 and 15). Besides verifying compatibility, we must not forget
that the attributed variables are to be the same --- because of this
very unification. We detach the attribute of one of the two attributed
variables involved (lines 13 and 16). The choice is arbitrary.  Then the
variables can be unified without rising another call to
@code{combine_attributes/2} (lines 14 and 17).

@noindent
With the definitions up to this point, we would get the following
interaction:

@example
[Clp] ?- f(odd(X),even(Y)) = Z.

Z = f(X,Y),
attach_attribute(Y,even(Y)),
attach_attribute(X,odd(X)) 

[Clp] ?- f(odd(X),even(Y)) = Z, Z = f(13,_).

X = 13,
Z = f(13,Y),
attach_attribute(Y,even(Y)) ? 
@end example

@noindent
In order to get the more appealing output that was shown in the session
transcript (@xref{Sample session}), @code{dump/3} has to be coded,
@pxref{Dumping}.

@noindent
The following encoding of @code{dump/3} uses a binary tree as dictionary
to represent the association between the (attributed) variables in the original
@var{Term} and the variables in the @var{Copy} and the list of @var{Constraints}.
The code as presented gets fooled by cyclic structures. It is a trivial matter
to make it cycle-proof with the very same dictionary that is used for the 
variables (left as an exercise).

Once the copy has been produced (line 2), the dictionary is traversed to
collect the attributes associated with the variables in the copy via
@code{collect_constraints/3} (lines 39--45), which is formulated as a DCG.

@example
     1	dump( Term, Copy, Constraints) :-
     2	  copy( Term, Copy, Dict),
     3	  collect_constraints( Dict, Constraints, []).
     4	
     5	copy( Term, Copy, Dict) :-
     6	  type( Term, Tt), 
     7	  copy( Tt, Term, Copy, Dict).
     8	
     9	copy( cva,       Cva,    Copy,     Dict) :- 
    10	  get_attribute( Cva, Attrib), 
    11	  copy_cva( Attrib, Copy, Dict).
    12	copy( var,       V,      Copy,     Dict) :- 
    13	  dict_insert( Dict, v(V,Copy), _).
    14	copy( integer,   I,      I,        _). 
    15	copy( float,     F,      F,        _).
    16	copy( atom,      A,      A,        _).
    17	copy( list,      [X|Xt], [Xc|Xct], Dict) :- 
    18	  type( X, Xtype),   copy( Xtype, X, Xc, Dict), 
    19	  type( Xt, Xttype), copy( Xttype, Xt, Xct, Dict).
    20	copy( structure, Term,   Copy,     Dict) :-
    21	  functor( Term, N, A),
    22	  functor( Copy, N, A),
    23	  copy_arg( A, Copy, Dict, Term).
    24	
    25	copy_arg( 0, _,    _,    _) :- !.
    26	copy_arg( N, Copy, Dict, Term) :- 
    27	  N1 is N-1, 
    28	  arg( N, Copy, Ac),
    29	  arg( N, Term, At),
    30	  type( At, Att), 
    31	  copy( Att, At, Ac, Dict),
    32	  copy_arg( N1, Copy, Dict, Term). 
    33	
    34	copy_cva( odd(X), Copy, Dict) :-
    35	  dict_insert( Dict, cva(X,Copy,odd(Copy)), _).  
    36	copy_cva( even(X), Copy, Dict) :-
    37	  dict_insert( Dict, cva(X,Copy,even(Copy)), _).
    38	
    39	collect_constraints( Dict)        --> {var(Dict)}, !.
    40	collect_constraints( t(L,Key,R) ) --> 
    41	  ( { Key = v(_,_) }
    42	  ; { Key = cva(_,_,Pred) }, [ Pred ]
    43	  ),
    44	  collect_constraints( L),
    45	  collect_constraints( R).
@end example

@noindent
The dictionary operations on the binary tree can be performed with
the following predicate:

@example
     1	% insert a F(Key,...) elem into the a binary tree
     2	%
     3	dict_insert( Tree, Key, Occ) :- var(Tree), !,
     4	  Tree = t(_,Key,_), 
     5	  Occ = new.
     6	dict_insert( t(L,Key0,R), Key, Occ) :- 
     7	  arg( 1, Key, Ak),
     8	  arg( 1, Key0, Ak0),
     9	  compare( Rel, Ak, Ak0),
    10	  ( Rel = =, Occ = old, Key = Key0
    11	  ; Rel = <, dict_insert( L, Key, Occ)
    12	  ; Rel = >, dict_insert( R, Key, Occ)
    13	  ). 
@end example

@noindent
What is still missing is how to print our attributed variables that
stand for interpreted terms, @pxref{Printing}. In general it is a good
idea to use @code{dump/3} for that purpose. In @strong{this} simple case
however, it is not useful to apply the full dump machinery. Therefore:

@example
     1	portray_attribute( odd(X),  _) :- write( odd(X)).
     2	portray_attribute( even(X), _) :- write( even(X)).
@end example

@noindent
Note the use of @code{write/1}! If @code{print/1} is used instead, nothing
will be printed (guess why).

@noindent
Taken together, these code fragments implement our type 'theory'.  

@noindent
@cindex Common mistake
Warning: 
@itemize @bullet
@item
Because this code contains interpreted terms (@code{odd/1} and
@code{even/1} functors occur in some predicates), it @strong{must} be
loaded from the original SICStus Prolog toplevel. Only after consulting
or compiling the file, you would enter the new toplevel via
@code{top/0}. If your programs behave strange, it is quite likely that a
'pure' program has been loaded into a context with interpreted terms, or
vice versa.
@end itemize

@node How to produce the DMCAI CLP clone    
@comment  node-name,  next,  previous,  up
@appendix How to produce the DMCAI CLP clone
The following procedure assumes that you have got SICStus2.1 at patchlevel
#3 or later. We will try to keep up with future versions of SICStus2.1.

@enumerate
@item
Produce a copy of your original SICStus2.1 directory tree.  Assuming
that that the original resides in the directory @file{sicstus2.1}, and
that you want to call the clone @file{sicstus2.1.clp}, you might use the
commands:

@example                                                                 
mkdir sicstus2.1.clp
(cd sicstus2.1; tar cf - .) | (cd sicstus2.1.clp; tar xf -)
@end example

@item     
@code{cd sicstus2.1.clp}
      
@item
execute the patch script 

@example       
patch < clp.2.1.6.patch
@end example
    
The last digit in @file{clp.2.1.6.patch} identifies the SICStus2.1 
patchlevel to which the patch applies.
If you get messages about failed hunks (most likely for Makefiles), the
corresponding patches must be applied by hand. You can check for failed
patches with

@example      
find . -name '*.rej' -print 
@end example
      
@item
@code{cd Compiler; make} @* 
This step assumes that the executable of your original SICStus2.1 
is installed as @code{sicstus}. If this is not so, 
type @code{make SICSTUS=someothername}
      
@item
@code{cd Library; make}
      
@item
@code{cd Emulator; make clean}
      
@item
@code{cd} to the root of the copy (@file{sicstus2.1.clp}), ensure that the 
@var{OPTIONS} in the Makefile @emph{do not} include @code{-DBDD},
and type:

@example
setenv SP_PATH `pwd`
make all SICSTUS=$SP_PATH/sicstus
@end example

@end enumerate
@noindent
Please let us know about how your installation went. Address correspondence
to:

@display
    Christian Holzbaur
    Dept. of Medical Cybernetics & Artificial Intelligence (DMCAI)
    University of Vienna
    Freyung 6
    A-1010 Vienna
    Austria
    Email: christian@@ai.univie.ac.at
@end display

@node Predicate Index
@comment  node-name,  next,  previous,  up
@unnumbered Predicate Index
@printindex fn

@node Concept Index
@comment  node-name,  next,  previous,  up
@unnumbered Concept Index
@printindex cp

@contents

@bye
