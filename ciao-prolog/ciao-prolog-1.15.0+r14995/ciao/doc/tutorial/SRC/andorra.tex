
\section{Andorra Prolog: The Determinate-first Principle}

In Prolog there is an advice known as the first-fail principle:
provide for goals which are more likely to fail to be executed
first. This is because the sooner they fail more useless work is
saved, if they are going to fail anyway.

In Andorra Prolog this is generalized to the determinate-first
principle: execute first goals which have less alternatives (which are
determinate). A goal is said to be {\bf determinate} if there is only
one clause that matches it. Note that such a goal is more ``close'' to
failing than other goals that may have more alternatives. In any case,
such goals save backtracking. They are thus executed first, regardless
of their position in the program.
