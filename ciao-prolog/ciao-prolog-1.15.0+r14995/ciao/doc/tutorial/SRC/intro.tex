
This manual, ``the Ciao tutorial'', is a tutorial overview of the Ciao
Prolog language. It won't tell you how to program in (Ciao) Prolog,
but what to write in a program in order to use some of the language
features. Thus, it is not a programming manual. It is a quick
introduction to the most relevant language features. It is
complementary to and complemented with ``the Ciao reference manual'':
{\em The Ciao Prolog System Reference Manual}~\cite{ciao-reference-manual-tr}.


The Ciao tutorial assumes familiarity with Prolog.
However, this document includes a brief explanation of the Prolog
language (see Section~\ref{sec:pure}). If you are a beginner to
Prolog, you would probably want to start with that introduction before
going through the rest of the sections. 

The current status of this document is that of a ``preliminary
version''. Currently, the relevant features of Ciao that are explained 
are the module system, libraries, packages, and code
expansions. Future releases will include: concurrent and parallel 
execution, object and distributed programming, constraint programming,
fuzzy prolog, and other (logic) programming styles supported by Ciao.

%% If you want to start reading
%% from the very beginning, here is a very short introduction in order to
%% set up the basic concepts.
%% 
%% A (Ciao) Prolog program is a set of procedure (or predicate)
%% definitions. Predicates/pro-cedures are identified by a {\bf predicate
%%   spec}, which is of the form \verb+p/n+, where \verb+p+ is the
%% predicate name, and \verb+n+ its {\bf arity}: the number of its
%% arguments. 
%% 
%% Procedure headers and procedure calls are written as {\bf atoms}: the
%% name of the predicate followed by the arguments enclosed in
%% parentheses and separated by commas. A predicate is defined by one or
%% several clauses, as in:
%% \begin{quote}
%% \begin{verbatim}
%% append([],X,X).
%% append([X|Xs],Ys,[X|Zs]):- append(Xs,Ys,Zs).
%% \end{verbatim}
%% \end{quote}
%% %
%% where the first clause is a {\bf fact}, which consists only of the
%% procedure header, and the second one is a {\bf rule}, which consists
%% of the procedure header (left to \verb+:-+) and a body, which is a
%% sequence of procedure calls, called {\bf goals}, separated by commas.
%% 
%% Execution of a procedure call (or goal) proceeds by first selecting
%% one of the clauses defining the procedure, and then executing each of
%% the goals in the body of the clause. Parameter passing is performed
%% via unification. Clause selection is decided by resolution and
%% backtracking. (See Section~\ref{sec:pure}.) 
%% 
%% Procedure arguments are terms: constants, variables, or data
%% structures (see Section~\ref{sec:pure}). In what follows, procedure
%% arguments are constants if starting with lowercase and variables if
%% starting with uppercase. 
