
Hi Manuel,

I'd like to suggest a wording along the following lines, to give SICS the 
right to bundle PilLLoW with SICStus distribution:

Facultad de Informatica, Universidad Politecnica de Madrid (UPM), 
28660-Boadilla del Monte, MADRID, SPAIN hereby grants the Swedish Institute 
of Computer Science (SICS) the right to distribute the PiLLoW software to 
the licensees of the SICStus Prolog software as a part of SICStus Prolog 
distribution.

SICS undertakes to properly acknowledge that UPM is the 
author/owner/copyright holder (delete words that do not apply) of the 
PiLLoW software and to provide UPM with the users' feedback.

UPM undertakes that to the best of its knowledge the PiLLoW software is 
original to UPM and/or that the use of the PiLLoW software does not 
constitute a violation of any third party's intellectual property.
(the above sentence aims to let SICS assure SICStus' licensees that their 
use of PiLLoW will not bring them to court. Some of the elephants 
(Ericsson, and alike) are very reluctant to commercially use any code 
without such a assurance from a supplier).

Please comment on the above.

------------

X-Coding-System: nil
Return-Path: <boris@lml.ls.fi.upm.es>
From: Manuel Carro <boris@lml.ls.fi.upm.es>
MIME-Version: 1.0
Date: Wed, 4 Oct 2000 13:41:30 +0200 (MEST)
To: herme@fi.upm.es
Cc: clip-small
Subject: Re: [janusz@sics.se: Re: PiLLoW for SICStus 3.8 available!]
In-Reply-To: <200010041039.MAA10967@clip.dia.fi.upm.es>
References: <200010041039.MAA10967@clip.dia.fi.upm.es>
Reply-To: boris@lml.ls.fi.upm.es
X-My-Url: http://lml.ls.dia.fi.upm.es/~boris


    Me parece que la propuesta de SICS es razonable.  Lo �nico es que
dice que UPM es el propietario de copyright.  Sin embargo se puede
argumentar que Pillow fue desarrollado por Daniel Cabeza cuando no
ten�a contrato de profesor con la UPM, as� que posiblemente no se
pueda aplicar los puntos que a los dem�s.  Por otro lado, quiz� no
queramos firmar algo que diga que un producto es de la UPM, sino mejor
de los integrantes del grupo CLIP (�o eso es muy vago?).  Y en cuanto
a cambiar el status de Ciao, que os parece algo como:

    In return, we would like to change the status of a previous
agreement between SICS and UPM, which stated that UPM had permission
to use SICStus 0.5, 0.6, and 0.7 in order to develop &-Prolog and its
derivatives.  Our current system is termed Ciao Prolog, and, although
we clearly see it as a derivative of &-Prolog, we do not want to
create confusion around a change of names.  We would prefer this
agreement to be reformulated as

"SICS hereby grants the components of the CLIP Group the right to
freely use SICStus Prolog 0.5, 0.6 and 0.7 in order to develop
&-Prolog, Ciao Prolog, and its derivatives (all three termed jointly
as the System) at no charge.  Non-exclusive, perpetual permission is
granted to the aforementioned CLIP Group to licence, sublicence, and
distribute the System under any terms they deem appropriate."

    �o suena demasiado fuerte?  Es que lo del lenguaje legal es un
poco agresivo...

        MCL

________________________________________________________________
[...] put Windows back into its place as an overpriced Nintendo.

X-Coding-System: nil
Return-Path: <herme@clip.dia.fi.upm.es>
Date: Wed, 4 Oct 2000 20:32:09 +0200
From: Manuel Hermenegildo <herme>
To: boris@lml.ls.fi.upm.es
CC: clip-small
In-reply-to: <14811.5994.530595.521711@salmon.ls.fi.upm.es> (message from
        Manuel Carro on Wed, 4 Oct 2000 13:41:30 +0200 (MEST))
Subject: Re: [janusz@sics.se: Re: PiLLoW for SICStus 3.8 available!]
Reply-to: herme@fi.upm.es
References: <200010041039.MAA10967@clip.dia.fi.upm.es> <14811.5994.530595.521711@salmon.ls.fi.upm.es>


 > argumentar que Pillow fue desarrollado por Daniel Cabeza cuando no
 > ten�a contrato de profesor con la UPM, as� que posiblemente no se
 > pueda aplicar los puntos que a los dem�s.  Por otro lado, quiz� no

Humm, lo que pasa es que si que estaba contratado todo el rato por la
UPM de una manera u otra (pero no a tiempo completo, claro). 

 > queramos firmar algo que diga que un producto es de la UPM, sino mejor
 > de los integrantes del grupo CLIP (�o eso es muy vago?).  Y en cuanto
 > a cambiar el status de Ciao, que os parece algo como:

A mi esto me gusta mas (pero necesitamos el disclaimer de la UPM).

Sobre el parrafo de SICS, que tal:

Actually, we would also like to ask something from SICS. You may
recall that at some point you gave us a written note saying that we
had permission to freely distribute the parts of SICStus 0.5, 0.6, and
0.7 which were included in &-Prolog and its derivatives.  Our current
derivative of &-Prolog (the Ciao system, which we use as our
experimentation platform, and which we distribute with a GPL license).
The GPL license requires us to get from you a disclaimer to any part
of the system.  Your original paper says this essentially, but, just
to conform exactly to what the GPL asks us, and to make the GNU guys
happy, we would appreciate if you can sign this disclaimer in the
exact wording that they like, i.e.:

... y aqui habria que meter la licencia esa que dice Stallman que hay
que conseguir de 'your contractor', a nuestro favor.

Man

-- 

-----------------------------------------------------------------------------
herme@fi.upm.es                      | Manuel Hermenegildo                 
+34-91-336-7435 (Work)               | Facultad de Informatica, UPM
+34-91-352-4819 or 336-7412 (FAX)    | Universidad Politecnica de Madrid   
http://www.clip.dia.fi.upm.es/~herme | 28660-Boadilla del Monte, MADRID SPAIN
-----------------------------------------------------------------------------


>>  > ... y aqui habria que meter la licencia esa que dice Stallman que hay
>>  > que conseguir de 'your contractor', a nuestro favor.
>> 
>> Alguien puede localizar el texto este? Viene con las instrucciones
>> sobre como hacer alfo GNU, etc. Man

  Yoyodyne, Inc., hereby disclaims all copyright interest in the program
  `Gnomovision' (which makes passes at compilers) written by James Hacker.

  <signature of Ty Coon>, 1 April 1989
  Ty Coon, President of Vice

        MCL

____________________________________
I like beer for breakfast sometimes.
To: herme, german, bardo, boris, pedro
Subject: Pillow, SICS, Ciao, CLIP y la UPM


Este es el "contrato" de cesion de Pillow a SICS que se me ocurre que
podemos usar para, de paso, hacernos con la propiedad. A ver que os
parece. 

Paco
------------------------------------------------------------------------

Universidad Politecnica de Madrid (UPM), 
<direccion social UPM>, MADRID, SPAIN
hereby grants the Swedish Institute of Computer Science (SICS) the
right to distribute the PiLLoW software to the licensees of the
SICStus Prolog software as a part of SICStus Prolog distribution.

UPM hereby disclaims all copyright interest in the Pillow software,
written by 
the CLIP group in the Artificial Intelligence Dept. of the Facultad de
Inform�tica of UPM (CLIP). 

SICS undertakes to properly acknowledge that CLIP at UPM
is the author
<quitamos owner/copyright holder (delete words that do not apply)>
of the PiLLoW software and to provide CLIP with the users' feedback.

CLIP undertakes that to the best of its knowledge the PiLLoW software is 
original to them and/or that the use of the PiLLoW software does not 
constitute a violation of any third party's intellectual property.

Xxxx              Javier Uceda                     Manuel Hermenegildo
Vice-Y of SICS    Vice-Rector for Research, UPM    Head of CLIP

