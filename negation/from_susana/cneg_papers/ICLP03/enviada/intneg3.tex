%%%%%%%%%%%%%%%%%%%%%%%%%%%%%%%%%%%%%%%%%%%%%%%%%%%%%%%%%%%
% INTENSIONAL NEGATION. ICLP 2003
%%%%%%%%%%%%%%%%%%%%%%%%%%%%%%%%%%%%%%%%%%%%%%%%%%%%%%%%%%%

\documentclass[9pt]{llncs}

\usepackage{isolatin1}
\usepackage{amsmath} %% for functions defined using different parts
\usepackage{amssymb}
\usepackage{latexsym}
\usepackage{times}
\usepackage{mathptmx}
\renewcommand{\ttdefault}{cmtt}
\usepackage{theorem}
\newtheorem{Def}{Definition}
\newtheorem{Prop}{Proposition}
\newtheorem{Theo}{Theorem}
\theoremstyle{change}


%\newcommand{\tab}{\hspace{2em}}
%\newcommand{\N}{I\!\!N}
\newcommand{\N}{\Bbb{N}}
\newcommand{\neck}{\leftarrow}
\newcommand{\where}{[\!]} %% OJO!!
\newcommand{\df}{\mathit{def}}
\newcommand{\cdf}{\mathit{cdef}}
\newcommand{\FS}{\mathit{FS}}
\newcommand{\PS}{\mathit{PS}}
\newcommand{\n}{\mathit{neg}}
\newcommand{\Term}{\mathit{Term}}
\newcommand{\comp}{\mathit{Comp}}
\newcommand{\entails}{\models}
\newcommand{\vecx}{\overline{x}}
\newcommand{\vecy}{\overline{y}}
\newcommand{\vecz}{\overline{z}}
\newcommand{\vecb}{\overline{b}}
\newcommand{\vecB}{\overline{B}}
\newcommand{\true}{\underline{\mathrm{t}}}
\newcommand{\false}{\underline{\mathrm{f}}}
\newcommand{\CET}{\mathit{CET}}
\newcommand{\even}{\mathit{even}}
\newcommand{\then}{\rightarrow}
\newcommand{\els}{;}
\newcommand{\onlythen}{\twoheadrightarrow}
\newcommand{\parels}{\parallel}
\newcommand{\CD}{\mathit{CD}}
\newcommand{\NNF}{\mathit{NNF}}
\newcommand{\negp}{\mathit{neg\_p}}
\newcommand{\vect}{\overline{t}}
\newcommand{\negrhs}{\mathit{negate\_rhs}}
\newcommand{\paratodo}{\mathit{forall}}

%% TITULO %%%%%%%%%%%%%%%%%%%%%%%%%%%%%%%%%%%%%%%%%%%%%%%%%%%%%%%%%%%%%%%%

%% \title{Constructive Intensional Negation: \\ 
%%       Program Transformation to obtain Logic Negation} 

%% Provisional 
%%\title{A Practical Implementation of Intensional Negation}

\author{Susana Mu�oz \and Julio Mari�o \and Juan Jos� Moreno-Navarro}
\title{Constructive Intensional Negation:
       a Practical Implementation \\
       (Extended abstract} 

\institute{Universidad Polit�cnica de Madrid
\thanks{Dpto.\ LSIIS -- Facultad de Inform�tica.
        Campus de Montegancedo s/n,
        28660, Madrid, SPAIN.
        email:\texttt{\{susana,jmarino,jjmoreno\}@fi.upm.es},
        voice: +34-91-336-7455, fax: +34-91-336-6595. This
                research was partly supported by the Spanish MCYT project 
                TIC2000-1632.
    }
}


\newenvironment{mytabbing}
   {\vspace{0.4em}\begin{small}\begin{tabbing}}
   {\end{tabbing}\end{small}\vspace{0.4em}}

\newenvironment{prolog}
   {\vspace{0.4em}\begin{small}\begin{tt}\begin{tabbing}}
   {\end{tabbing}\end{tt}\end{small}\vspace{0.4em}}

%\theoremstyle{change}

\begin{document}
%\pagestyle{empty}

\maketitle

%% ABSTRACT %%%%%%%%%%%%%%%%%%%%%%%%%%%%%%%%%%%%%%%%%%%%%%%%%%%%%%%%%%%%%%%
\begin{abstract}
  Negation is arguably the most important feature of first order logic
  not present in Logic Programming (LP).  While the integration of
  negation into LP has been actively studied from a semantic
  perspective, the negation capabilities in actual Prolog compilers
  are rather limited.  One of the most promising techniques in the
  literature is intensional negation, following a transformational
  approach: for each positive predicate $p$ its negative counterpart
  $intneg(p)$ is generated.  However, from the practical point of view
  intensional negation cannot be considered a successful approach
  because no implementation is given. The reason is that neither
  universally quantified goals can be computed nor many practical
  issues are addressed.  In this paper, we describe our variant of the
  transformation of the intensional negation, called
  \emph{Constructive Intensional Negation}
%% �de d�nde sale este palabro?
  providing some formal results as well as discussing a concrete
  implementation.

\noindent

{\bf Keywords:} Negation, Constraint Logic Programming, Program
Transformation, Logic Programming Implementation, Constructive
Negation.

\end{abstract}

%%%%%%%%%%%%%%%%%%%%%%%%%%%%%%%%%%%%%%%%%%%%%%%%%%%%%%%%%%%%%%%%%%%%%%%%%%%

%%%%%%%%%%%%%%%%%%%%%%%%%%%%%%%%%%%%%%%%%%%%%%%%%%%%%%%%%%
%%%%%%%%%%%%%%%%%%% INTRODUCTION %%%%%%%%%%%%%%%%%%%%%%%%%
%%%%%%%%%%%%%%%%%%%%%%%%%%%%%%%%%%%%%%%%%%%%%%%%%%%%%%%%%%

\section{Introduction}
Kowalski and Colmerauer's decision on the elements of first order logic
supported in LP was based on efficiency considerations.
% the logical components that can be implemented ``efficiently''. 
Among those important aspects not included from the beginning we can
mention \emph{evaluable functions}, \emph{negation} and \emph{higher
order features}.  All of them have revealed themselves as important
for the expressiveness of Prolog as a programming language, but, while
in the case of evaluable functions and higher order features
considerable effort has been invested both in the semantic
characterization and its efficient implementation we cannot say the
same of negation.  Many research papers do propose semantics to
understand and incorporate negation into logic programming, but only a
small subset of these ideas have their corresponding implementation
counterpart.
%%, forgetting that for
%%LP "inventors" it was a pre-requisite for inclusion.  
In fact, the negation capabilities incorporated by current Prolog
compilers are rather limited, namely: the (unsound) negation as
failure rule, and the sound (but incomplete) delay technique of the
language G\"odel \cite{Goedel}, or Nu-Prolog \cite{Naish} (having the
risk of floundering.) The constructive negation of Eclipse, which was
announced in earlier versions has been removed from recent releases
due to implementation errors.

The authors have been involved in a project
\cite{SusanaPADL2000,SusanaLPAR01} to incorporate negation in a real
Prolog system (Ciao, \cite{CIAO}.)  This allows us to keep the very
advanced Prolog current implementations based on WAM technology as
well as to reuse thousands of Prolog code lines.  The basis of our
work is to combine existing techniques to make them useful for
practical application. Furthermore, we try to use the simplest
technique as possible in any particular case.  To help on this
distinction, we need to use some tests to characterize the situation
for efficiency reasons.  To avoid the execution of dynamic tests, we
suggest to use the results of a global analysis of the source code.
For instance, the primary technique is the built-in \emph{negation as
failure} that can be used if a groundness analysis tells that every
negative literal will be ground at call time \cite{SusanaPRODE00}.
%% Esto puede ir en otra parte a menos que quede clara la relaci�n
%% entre delays y negaci�n por fallo
%% Other program analyses includes elimination of delays, and the
%% determination of the finiteness of the number of solutions.

In order to handle non-ground literals, a number of alternatives to
the negation-as-failure rule have been proposed under the generic name
of \emph{constructive negation}: Chan's \emph{constructive negation}%
~\cite{Chan1,Chan2}, \emph{intensional negation}%
~\cite{Barbuti1,Barbuti2,Bruscoli},  
fail substitutions, fail answers, etc. 
From a theoretical viewpoint Chan's approach is enough
%% esto no est� nada claro
%% (the only with completeness results) 
but it is quite difficult to implement and expensive in terms of 
execution resources.
On the other hand, intensional negation
%% Esto no es verdad
%% , that precludes completeness
%% because the use of universally quantified goals, 
uses a transformational approach, so most of the work is performed at
compile time and then a variant of the standard SLD resolution is used
to execute the resulting program, so a significant gain in efficiency
is expected.  There are, however, some problems when transforming
certain classes of programs.

In this paper we concentrate on the study of the efficient
implementation of intensional negation.  As it is formulated in the
original papers, it is not possible to derive an efficient
transformation. On one hand, universally quantified goals are
generated hard to be managed. On the other hand the operational
behavior of the original program is modified computing infinitely many
answers instead of compact results. Furthermore, many significant
details were missing. We propose a way of implementing the universal
quantification. It is based on the properties of the domain instead of
the code of the program as in \cite{Bruscoli}.

%% Esto no lo proponemos nosotros
%% In fact we propose to reformulate the technique in terms of Constraint
%% LP to handle the first problem. The idea is also present in the
%% so-called {\em Compilative Constructive Negation} \cite{Bruscoli}
%% which solves some of the problems and provide interesting formal
%% results. However, it is again not formulated thinking on a concrete
%% implementation (important implementation issues are not present nor
%% can be easily derived) and universal quantification remains in the
%% resulting code.

%% Esto no debe ir aqu�
%% The main ideas of our {\em Constructive Intensional Negation}
%% transformational approach were presented in the seminal paper
%% \cite{SusanaPADL2000}.  The paper is devoted to supply some practical
%% details needed for the complete implementation as well as to present
%% some formal results proving the correctness of our approach. In
%% particular, the complete transformation is fully specified
%% algorithmically, ready to be included in a compiler. In fact, it has
%% been done in Ciao Prolog and we include some experimental executions
%% and results.

The rest of the paper is organized as follows. 
% We first present (Section
% \ref{previous}) the
% Related Work we base on: Intensional Negation and Compilative
% Constructive Negation. 
Section \ref{preliminaries} introduces basic syntactic and
semantic concepts needed to understand our method.
Section
\ref{techniquecin} presents more details on the Constructive
Intensional Negation technique to transform programs. Section
\ref{transformation} formally presents the transformation algorithm
while some formal results are established (Section \ref{formal}). The
implementation issues are studied in Section \ref{implementation}.
Then we show some execution examples (Section \ref{examples}).
Finally, we conclude and discuss some future work (Section
\ref{strategy}).


%%%%%%%%%%%%%%%%%%%%%%%%%%%%%%%%%%%%%%%%%%%%%%%%%%%%%%%%%%%%%%%%%%
%%%%%%%%%%%%%%  INTENSIONAL NEGATION TECHNIQUE  %%%%%%%%%%%%%%%%%%
%%%%%%%%%%%%%%%%%%%%%%%%%%%%%%%%%%%%%%%%%%%%%%%%%%%%%%%%%%%%%%%%%%

% \section{Previous Work}
% \label{previous}

% \subsection{Intensional Negation}
% \label{intensional}

% In Intensional negation \cite{Barbuti1,Barbuti2} a program
% transformation technique is used to add new predicates to the program
% in order to express the negative information.  Informally, the {\em
%   complement} of head terms of the positive clauses are computed and
% they are used later as the head of the negated predicate. Instead of
% generating so many negated predicates as predicates we want to negate,
% we will use a predicate $intneg/1$ that, in fact, is the negation
% whose implementation is explicit for any particular predicate that was
% going to be negated.

% For example, the transformation of a program (from \cite{Barbuti1}):
% %
% \begin{tt}
% \begin{mytabbing}
% ~~~~\= even(0) \\
%     \> even(s(s(X))) $\leftarrow$ even (X)
% \end{mytabbing}
% \end{tt}
% %
% yields a set of new clauses for the predicate $intneg/1$ that succeeds
% when its argument is a goal with $even$ that fails and vice versa
% (i.e.\ it is the logical negation $\neg$).
% %
% \begin{tt}
% \begin{mytabbing}
% ~~~~\= intneg(even(s(0))) \\
%     \> intneg(even(s(s(X)))) $\leftarrow$ intneg(even(X))
% \end{mytabbing}
% \end{tt}
% %
% Our transformation approach basically follows the ideas of
% \cite{Barbuti2}, but differs in some significant points.  For the
% detailed description of the transformation, Barbuti and his co-authors
% apply the transformation to a restricted class of programs. Then they
% show that any program can be translated into a program of that class.
% Despite the elegance and simplicity of the presentation, this
% description does not help too much for practical implementation.

% There are two problems with this technique. The first one is that in
% the presence of logical variables in the right hand side (rhs) of a
% clause, the new program needs to handle some kind of universal
% quantification construct. The second trouble affects the outcomes of
% the program: while the new program is semantically equivalent to the
% completed program, the operational behavior can differ. In the
% presence of logical variables, the new predicate can generate all the
% possible values one by one, even when a more general answer can be
% given. The predicate $p/2$ defined by the single clause $p(X, X)$ is
% negated by:
% %
% \begin{tt}
% \begin{mytabbing}
% ~~~~\= intneg(p(X, Y)) $\leftarrow$ intneg(eq(X, Y)) \\
% \\
% ~~~~\= intneg(eq(0, s(Y))) \\
%     \> intneg(eq(s(X), 0)) \\
%     \> intneg(eq(s(X), s(Y))) $\leftarrow$ intneg(eq(X, Y))
% \end{mytabbing}
% \end{tt}
% %
% \noindent
% if the program only works with natural numbers constructed with $0$
% and $succ/1$.  The query $intneg(p(X,Y))$ will generate infinitely
% many answers, instead of the more general constraint $X \neq Y$. An
% answer like $X \neq Y$ can only be replaced by an infinite number of
% equalities.


% \subsection{Compilative Constructive Negation}
% \label{techniquein}

% In order to cope with the second trouble it is possible to use
% explicitly the formulas in the program completion as rewrite rules
% with a suitable constraint extraction procedure. It forces us to
% understand the transformed program into a Constraint LP framework,
% working with equality and disequality constraints over the Herbrand
% domain.  The {\em Compilative Constructive Negation} of
% \cite{Bruscoli} follows the ideas of constructive negation but
% introducing constraints for predicate left hand side (lhs) and
% therefore for its negation. The description of the program
% transformation is in the following definition.

% \begin{definition} [Negation Predicate]

% Given a predicate $p/n$ defined by $m$ clauses:

% %\vspace{-15pt}
% \begin{mytabbing}
% ~~~~\=C$_1$:~~\=$p(\overline{t_1})$~ \=$ \leftarrow $ \=$A_1, B_1$. \\
%     \>~~~\ldots \\
%     \>C$_m$:  \>$p(\overline{t_m})$  \>$ \leftarrow $\> $A_m, B_m$. 
% \end{mytabbing}
% %\vspace{-12pt}

% \noindent
% where each $B_j = G'_{j,1}, \ldots, G'_{j, r_j}$ is a collection of
% literals and $A_j = G_{j,1}, \ldots, G_{j, k_j}$ is a collection of
% literals such that has no free variables (i.e. $var (A_j) \in var
% (\overline{t_j})$), its \emph{Negated Predicate} is
% \[\neg~p(\overline{X}) ~~\leftrightarrow~~ (\forall~\overline{Y_1}.(\neg c_1~
% \vee ~ \neg A_1 ~\vee~ \neg B_1) ~\wedge~ \ldots ~ \wedge~
% (\forall~\overline{Y_m}.(\neg c_m~ \vee ~ \neg A_m ~\vee~ \neg B_m) )
% \]
% %
% where $\overline{Y_j} = \mbox{\it free\_var (C$_j$)}$ and $c_j$ is the
% constraint resulting of unificating $\overline{X}$ with $\overline{t_j}$ (see below for details).
% \end{definition}


% As for each $j$ we have 
% $~~~F'_j = \forall~\overline{Y_j}.(\neg c_j~ \vee ~ \neg A_j ~\vee~ \neg B_j) ~~\equiv~~$ \\

% $ \forall~\overline{Y_j}.(\neg c_j~ \vee ~ \neg B_j) ~\vee~ \neg A_j ~~\equiv~~ \forall~\overline{Y_j}.(\neg c_j~) \vee ~ \forall~\overline{Y_j}.(\neg c_j~ \vee ~ \neg B_j)  ~\vee~ \neg A_j ~~\equiv~~$ \\

% $\forall~\overline{Y_j}.(\neg c_j) ~
% \vee ~ \forall~\overline{Y_j}.(\neg c_j ~ \vee ~\neg G'_{j,1} ~ \vee ~
% \ldots ~ \vee ~\neg G'_{j, r_j}) ~ \vee ~\neg G_{j,1} ~ \vee ~ \ldots ~ \vee ~\neg
% G_{j, k_j} $\\

% \noindent
% then we compute the disjunctive normal form $(F_1 ~ \vee ~ \ldots ~
% \vee ~ F_h)$ equivalent to the formula $(F'_1 ~ \wedge ~ \ldots ~ \wedge ~ F'_m)$
% and replace in $F_i$ the occurrences of $\neg q(\overline{s})$ by {\it
%   compneg} $(q(\overline{s}))$, obtaining $F^*_i$.

% The new program {\it CompNeg (P)} is:

% %\vspace{-15pt}
% \begin{mytabbing}
% ~~~~\=$compneg(p(\overline{X}))$~ \=$ \leftarrow $ \=$F*_1$. \\
%     \>~~~\ldots \\
%     \>$compneg(p(\overline{X}))$  \>$ \leftarrow $\> $F*_m$. 
% \end{mytabbing}
% %\vspace{-12pt}

% Again, the resulting program should contain universally quantified
% goals. The authors provide denotational semantics in terms of fixpoint
% of adequate immediate consequences operators as well as a description
% of the operational semantics, called {\sc sld$^{\forall}$}. All these
% semantics are proved to be correct and the correctness and
% completeness w.r.t.\ three-valued logic is established.

% Just to complete the presentation let us show the transformed program
% for our running example.
% %
% \begin{tt}
% \begin{mytabbing}
% ~~~~\= compneg(even(X)) $\leftarrow$ X $\neq$ 0 $\wedge$ $\forall$~ Y. X $\neq$ s(s(Y))) \\
%     \> compneg(even(X)) $\leftarrow$ X $\neq$ 0 $\wedge$ 
%                                      $\forall$~ Y.(X $\neq$ s(s(Y)) 
%                                                $\vee$ compneg(even(Y)))
% \end{mytabbing}
% \end{tt}
% %
% Although the paper contains many interesting ideas, it does not deal
% with universal quantification in a practical manner because it is
% evaluated at ru-rime in terms of satisfacibylity but not in a
% constructive way. Furthermore, the paper does not address the
% efficient management of constraints.

% As the authors are mainly concerned with formal semantics, the
% resulting program may contain useless clauses and by no means is
% optimal neither in the size of the program nor in the number of
% solutions.

%%%%%%%%%%%%%%%%%%%%%%%%%%%%%%%%%%%%%%%%%%%%%%%%%%%%%%%%%%%%%%%%%%%%%%%%%%%

\section{Preliminaries}
\label{preliminaries}
In this section the syntax of (constraint) logic programs and the
intended notion of correctness are introduced. Programs will be
constructed from a signature $\Sigma = \langle \FS_\Sigma, \PS_\Sigma
\rangle$ of function and predicate symbols. Provided a numerable set
of variables $V$ the set $\Term(\FS_\Sigma,V)$ of terms is constructed
in the usual way.

A \emph{constraint} is a first order formula whose atoms are taken
from $\Term(\FS_\Sigma,V)$ and where the only predicate symbol is the
binary equality operator $=_{/2}$. A formula $\neg(t_1 = t_2)$ will be
abbreviated $t_1 \neq t_2$. The constants $\true$ and $\false$ will
denote the neutral elements of conjunction and disjunction,
respectively. A tuple $(x_1,\ldots,x_n)$ will be abbreviated $\vecx$.
The concatenation of tuples $\vecx$ and $\vecy$ is denoted $\vecx
\cdot \vecy$.

A (constrained) Horn clause is a formula $h(\vecx) \neck
b_1(\vecy\cdot\vecz),\ldots,b_n(\vecy\cdot\vecz) \where c(\vecx \cdot
\vecy)$ where $\vecx$, $\vecy$ and $\vecz$ are tuples from disjoint
sets of variables. The variables in $\vecz$ are called the \emph{free}
variables in the clause. The symbols ``$,$'' and ``$\where$'' act here
as aliases of logical conjunction. The atom to the left of the symbol
``$\neck$'' is called the \emph{head} or \emph{left hand side} of the
clause.

\emph{Generalized} Horn clauses of the form $h(\vecx) \neck
B(\vecy\cdot\vecz)\where c(\vecx \cdot \vecy)$ where the body $B$ can
have arbitrary disjunctions, denoted by ``;'', and conjunctions of
atoms will be allowed as they can be easily translated into
``traditional'' ones.

A Prolog program (in $\Sigma$) is a set of clauses indexed by $p \in
\PS_\Sigma$:
%
\[\begin{array}{c}
p(\vecx) \neck B_1(\vecy_1\cdot\vecz_1) \where c_1(\vecx \cdot \vecy_1)\\
\vdots\\
p(\vecx) \neck B_m(\vecy_m\cdot\vecz_m) \where c_m(\vecx \cdot \vecy_m)\\
  \end{array}\]
%
The set of defining clauses for predicate symbol $p$ in program $P$ is
denoted $\df_P(p)$.
Without loss of generality we have assumed that the left hand sides in
$\df_P(p)$ share a single tuple of variables. 
Actual Prolog programs, however, will be written using a more
traditional syntax.

Assuming the normal form, let
$\df_P(p)=\{p(\vecx) \neck B_i(\vecy_i\cdot\vecz_i) 
                     \where c_i(\vecx \cdot \vecy_i) 
             | i \in 1 \ldots m\}$.
The \emph{completed definition} of $p$, $\cdf_P(p)$ is defined as the formula
%
\[ \forall \overline{x}.
   [p(\overline{x}) \iff \bigvee_{i=1}^m 
   \exists \vecy_i.\,~ c_i(\vecx \cdot \vecy_i) \wedge
                        \exists \vecz_i.B_i(\vecy_i\cdot\vecz_i)]
\]
%
The \emph{Clark's completion} of the program is the conjunction of the
completed definitions for all the predicate symbols in the program
along with the formulas that establish the standard interpretation for
the equality symbol, the so called \emph{Clark's Equality Theory} or
$\CET$. The completion of program $P$ will be denoted $\comp(P)$.
Throughout the paper, the standard meaning of logic programs will be
given by the three-valued interpretation of their completion -- i.e.\
its minimum 3-valued model.  
These 3 values will be denoted $\true$ (or \texttt{success}, or $\Box$),
$\false$ (or \texttt{fail}, or $\circleddash$) and
$\underline{\mathrm{u}}$ (or \texttt{unknown}, or $\bot$).

\begin{example}
Our sample program can be expressed as a set of normalized Horn
clauses in the following way:
%
\begin{tt} 
\begin{mytabbing}
even(X) $\neck$ $\where$ X=0.\\
even(X) $\neck$ even(N) $\where$ X=s(s(N)).
\end{mytabbing}
\end{tt}
%
Its completion is
%
\( \CET \wedge \forall x.\, [\even(x) \iff x=0 
   \vee \exists n.\, x=s(s(n)) \wedge even(n)]\,. \)
\end{example}


We are now able to specify intensional negation formally. 
Given a signature 
$\Sigma=\langle \FS_\Sigma, \PS_\Sigma \rangle$, let 
$\PS'_\Sigma \supset \PS_\Sigma$ be
such that for every $p \in \PS_\Sigma$ there exists a symbol 
$\n(p) \in \PS'_\Sigma \setminus \PS_\Sigma$. 
Let $\Sigma'= \langle \FS_\Sigma, \PS'_\Sigma \rangle$.
%
Given a program $P_\Sigma$, its intensional negation is a program
$P'_{\Sigma'}$ such that for every $p$ in $\PS_\Sigma$
the following holds:
%
\[ \forall \overline{x}. \comp(P) \entails_3 p(\vecx) \iff
                         \comp(P') \entails_3 \neg(\n(p)(\vecx)) \]


%%%%%%%%%%%%%%%%%%%%%%%%%%%%%%%%%%%%%%%%%%%%%%%%%%%%%%%%%%%%%%%%%%%%%%%%%%%
\section{Towards a Better Transformation}
\label{techniquecin}
Some of the problems with the previous methods are due to the
syntactical complexity of the negated program. 
A simplified translation can be obtained by taking into account some
properties that are often satisfied by the source program. 
For instance, it is usually the case that the clauses for a given
predicate are mutually exclusive.
The following definition formalizes this idea:

\begin{definition} A pair of constraints $c_1$ and $c_2$ is
  said to be \emph{incompatible} iff their conjunction $c_1 \wedge
  c_2$ is unsatisfiable. A set of constraints $\{c_i\}$ is
  \emph{exhaustive} iff $\bigvee_i c_i = \true$. A predicate
  definition
%
\(\df_P(p)=\{p(\vecx) \neck B_i(\vecy_i\cdot\vecz_i) 
                     \where c_i(\vecx \cdot \vecy_i) 
             | i \in 1 \ldots m\}
\)
%
is \emph{nonoverlapping} iff $\forall i,j \in 1\ldots m$ the
constraints $\exists \vecy_i.\, c_i(\vecx \cdot \vecy_i)$ and $\exists
\vecy_j.\, c_j(\vecx \cdot \vecy_j)$ are incompatible and it is
exhaustive if the set of constraints $\{\exists \vecy_i.\, c_i(\vecx
\cdot \vecy_i)\}$ is exhaustive.
\end{definition}

In the following, the symbols ``$\onlythen$'' and ``$\parels$'' will be
used as syntactic sugar for ``$\wedge$'' and ``$\vee$'' respectively, when
writing completed definitions from nonoverlapping and exhaustive sets
of clauses.
%
A nonoverlapping set of clauses can be made into a nonoverlapping and
exhaustive one by adding an extra ``default'' case:

\begin{lemma}\label{completion}
Let $p$ be such that its definition 
%
\(\df_P(p)=\{p(\vecx) \neck B_i(\vecy_i \cdot \vecz_i) 
                     \where c_i(\vecx \cdot \vecy_i) 
             | i \in 1 \ldots m\}
\)
is nonoverlapping. Then its completed definition is logically equivalent to
%
\[\begin{array}{rl}

\cdf_P(p) \equiv \forall \overline{x}.
   [p(\overline{x}) \iff &  
   \exists \vecy_1.\,~c_1(\vecx \cdot \vecy_1) 
     \onlythen \exists\vecz_1.B_1(\vecy_1\cdot\vecz_1) \parels \\
%   & \exists \vecy_2.\,~c_2(\vecx\cdot\vecy_2)
%     \onlythen \exists\vecz_2.B_2(\vecy_2\cdot\vecz_2) \parels \\
   & \vdots \\ 
   & \exists \vecy_m.\,~c_m(\vecx\cdot\vecy_m)
     \onlythen \exists\vecz_m.B_m(\vecy_m\cdot\vecz_m) \parels\\
   & \bigwedge_{i=1}^m 
     \neg\exists \vecy_i,\vecz_i.\,~c_i(\vecx\cdot\vecy_i)
     \onlythen \false \,]
  \end{array}\]
\end{lemma}

%% The completed definition of a predicate whose clauses are
%% nonoverlapping 
%% can be cast in a canonical, \emph{implicative}, form.
%% In the following, let $A \then B \els C$ be a shorthand for 
%% $(A \wedge B) \vee (\neg A \wedge C)$, and
%% $A \then B \els C \then D \els E$ a shorthand for 
%% $A \then B \els (C \then D \els E)$.

The interesting fact about this kind of definitions is captured by the
following lemma: 

\begin{lemma}\label{implicative-form2}
Given first order formulas $\{C_i\}$ and $\{B_i\}$ the
following holds: 
\[
\neg[\exists \vecy_1(C_1 \onlythen B_1) \parels \ldots \parels 
     \exists \vecy_n(C_n \onlythen B_n)]
\iff
\exists \vecy_1(C_1 \onlythen \neg B_1) \parels \ldots \parels 
\exists \vecy_n(C_n \onlythen \neg B_n)
\]
\end{lemma}

This means that the overall structure of the formula is preserved
after negation, which seems rather promising for our purposes. 



% \newcommand{\neglesser}{\mathit{neg\_lesser}}
% \begin{example}
% The following two completion formulas are logically equivalent:
% \[\begin{array}{rll}
%   \CET&\wedge~\cdf_P(lesser) \\
%       &\wedge \forall n,m.\, \neglesser(n,m) \iff
%       &\exists y. n=0 \wedge m=s(y) \onlythen \false \parels\\
%      &&\exists x,y. n=s(x) \wedge m=s(y) \onlythen \neg\lesser(x,y) \parels\\ 
%      &&m=0 \onlythen \true
%    \end{array}
% \]
% and
% \[\begin{array}{rll}
%   \CET&\wedge~\cdf_P(lesser) \\
%       &\wedge \forall n,m.\, \neglesser(n,m) \iff
%       &\exists y. n=0 \wedge m=s(y) \onlythen \false \parels\\
%      &&\exists x,y. n=s(x) \wedge m=s(y) \onlythen \neglesser(x,y) \parels\\ 
%      &&m=0 \onlythen \true
%    \end{array}
% \]
% \end{example}


%% Our approach to overcome these limitation is to formally define and
%% efficient transformational algorithm {\it CIntNeg} $(P)$, and to
%% provide effective and efficient implementation support either for
%% Herbrand constraints and universal quantification computations.

%% Before we specify the transformation let us show intuitively how our
%% technique works in the previous examples:

%% %\vspace{-5pt}
%% \begin{tt}
%% \begin{mytabbing}
%% ~~~~\= cintneg(p(X, Y)) $\leftarrow$ X $\neq$ Y \\
%%     \> cintneg(even(X)) $\leftarrow$ X $\neq$ 0 $\wedge$ $\forall$~ Y. X $\neq$ s(s(Y))) \\
%%     \> cintneg(even(s(s(X)))) $\leftarrow$ cintneg(even(X)) 
%% \end{mytabbing}
%% \end{tt}
%% %\vspace{-5pt}

%% Note that if the program domain only contains natural numbers, the first
%% clause is equivalent to {\tt cintneg(even(s(0)))}.

%% In Section \ref{implementation} we will treat implementation issues
%% related with the disequality constraints and the universal
%% quantification.

%% %%%%%%%%%%%%%%%%%%%%%%%%%%%%%%%%%%%%%%%%%%%%%%%%%%%%%%%%%%%%%%%%%%
%% %%%%%%%%%%%%%%%%%%% PROGRAM TRANSFORMATION  %%%%%%%%%%%%%%%%%%%%%%
%% %%%%%%%%%%%%%%%%%%%%%%%%%%%%%%%%%%%%%%%%%%%%%%%%%%%%%%%%%%%%%%%%%%

%% \subsection{Program Transformation Description}
%% \label{transformation}

%% In order to formally obtain the intensional negation of a goal {\tt G}
%% ({\tt cintneg(G)}) we proceed step by step. We firstly need to define
%% of the complement of a term $t$.  We will use constraints for this
%% purpose what provides a more compact representation.  Without the help
%% of constrained formulas the only way to represent the complement of a
%% term is a set of terms.

%% \begin{definition}
%% The {\tt Complement of a Term} $t$ (not using the variable $X$)
%% on the variable $X$ (in symbols $Comp(X,t)$) is a constraint
%% value for $X$, defined inductively as follows:
%% %\vspace{-5pt}
%% \begin{itemize}
%% %\addtolength{\itemsep}{-8pt}
%% \item $Comp (X,Y) = fail$
%% \item $Comp (X,c) = (X \neq c)$, with $c$  constant.
%% \item $Comp (X,c (t_1, \ldots, t_n)) =
%%          \forall~ \overline{Z} (X \neq c (t_1, \ldots, t_n))$,
%%          with $c$ a constructor and $\overline{Z}$ the variables
%%          of $t_1, \ldots, t_n$.
%% \end{itemize}
%% \end{definition}

%% \begin{definition}
%% The {\tt Complement of a Tuple} of terms $\overline{t}$ on the tuple of
%% variables $\overline{X}$ ( both n-tuples) is the following constraint:

%% \[ Comp(\overline{X},\overline{t}) = Comp(X_1,t_1) \vee ... \vee Comp(X_n,t_n)\]

%% Therefore, without loss of generality we can consider that all the
%% predicates has one argument, taking the tuple construction as a
%% constructor.
%% \end{definition}

%% \begin{definition}
%% Given a predicate $p/n$ defined by $m$ clauses:

%% %\vspace{-15pt}
%% \begin{mytabbing}
%% ~~~~\=C$_1$:~ \=$p(t^1)$~ \=$ \leftarrow $ \=$G_1 $. \\
%%     \>~~~\ldots \\
%%     \>C$_m$: \>$p(t^m)$ \>$ \leftarrow $\> $G_m $. 
%% \end{mytabbing}
%% %\vspace{-12pt}

%% \noindent
%% where $G^i$ is a conjunction of subgoals (with $i \in [1..m]$), we say
%% that $t$ is a {\tt Critical Value} of the predicate $p/n$ in a subset of its
%% clauses ($\{l_1, \ldots, l_r\} \subseteq \{1, \ldots, m\}$) if
%% %\vspace{-5pt}
%% $m.g.u. (t^{l_1}, \ldots, t^{l_r}) = \sigma \mbox{ and }
%% t = t^{l_j}\sigma, ~~~\forall~ j \in \{1, \ldots,r\}$
%% %\vspace{-15pt}$p/n$
%% \end{definition}

%% The definition is well known in Term Rewriting and, intuitively,
%% detects those terms for which there are more than one 
%% applicable clause. More formally:

%% \begin{Prop}
%%   For every critical value $t$ Let $p (s)$ be a goal and $t$ a
%%   critical value of $p$ in $\{l_1, \ldots, l_r\}$ such that $t$ and
%%   $s$ are unifiable.  Then for each $i=1..r$ there exists a
%%   SLD-derivation $~~p (s) ~\rightarrow_{C_{l_i}, \sigma_i} ~ G_i $ and
%%   for each SLD-derivation $~~p (s) ~\rightarrow_{C_j, \sigma} ~ G $
%%   then $j \in \{l_1, \ldots, l_r\}$

%% \end{Prop}


%% Notice that in case of the presence of critical pairs, all the $rhs$s
%% of the applicable clauses must be negated together.

%% \begin{definition}
%% Given a program $P$, and a predicate $p/n$ with $m$ clauses in $P$
%% with heads $p(t^1), \ldots, p(t^m)$, we define the {\tt Complement of
%%   a Predicate} $p/n$, {\tt cintneg(p(X))}, as the following set of
%% clauses:

%% \begin{enumerate}
%%         \item A clause of the form:

%% \[ cintneg(p(X)) \leftarrow Comp(X,t^1) \wedge ...  \wedge Comp(X,t^m)   \] 

%% assuming, by adequate renamings, that the terms do not share variables
%% (i.e. $var(t^i) \cap var(t^j) = \emptyset$ for $i \neq j$).  This
%% clause covers the cases where there is no definition for the predicate
%% in the original program, and it must be included in the new program. 
%% We will call it \emph{Complement Clause}.

%%         \item For each critical value $t$ of $p/n$, when $G_{l_1}, ... ,
%%         G_{l_r}$ are the bodies of the clauses whose head unifies with the
%%         critical value, a clause of the form:
%% \[ cintneg(p(t)) \leftarrow negate\_rhs(free\_var(G_{l_1} \vee ... \vee G_{l_r}), G_{l_1} \vee ... \vee G_{l_r})) \]

%% \noindent
%% where $free\_var(G)$ are the free variables of $G$ (variables
%% different from the ones in the head of the clause).

%%         \item For each clause $p(t^i) \leftarrow G_i$, where $G_i$ is not
%%   empty and $t^i$ does not unify with any critical value of $p/n$, a
%%   clause of the form:
%% \[ cintneg(p(t^i)) \leftarrow negate\_rhs(free\_var(G_i), G_i) \]

%%         \item For each clause $p(t^i) \leftarrow G_i$, where $G_i$ is not
%%   empty and $t^i$ unifies with a critical value $t$ of $p/n$ but they
%%   are not strictly identical terms, a clause of the form:
%% \[ cintneg(p(t^i)) \leftarrow t^i \neq t \wedge negate\_rhs(free\_var(G_i), G_i) \]
%% A clause can be implied in several critical values and we will add an
%% additional clause of the above form for each of them.

%% \end{enumerate}
%% \end{definition}

%% The effect of $negate\_rhs$ is easy to explain. It just
%% negates the rhs and introduces universal quantifications when
%% they are needed:

%% $$ 
%% negate\_rhs (\overline{V}, G) = 
%%   \begin{cases}
%%     cintneg(G)                        & \text{if $\overline{V} = \emptyset$} \\ 
%%     \forall~ \overline{V} .~ cintneg(G) & \mbox{otherwise} 
%%   \end{cases} 
%% $$

%% % %\vspace{-5pt}
%% % \begin{tabular}{lll}
%% % $\bullet$ &  $negate\_rhs (\overline{V}, G) = cintneg(G)$ & if $ \overline{V} = \emptyset $ \\
%% % $\bullet$ &  $negate\_rhs (\overline{V}, G) = \forall \overline{V}. cintneg(G)$ &  otherwise \\   
%% % \end{tabular}
%% % %\vspace{-5pt}

%% In Section \ref{strategy} we will discuss the advantages of defining
%% the function $negate\_rhs$ with a general negation $neg/1$ instead of
%% the intensional negation $cintneg/1$ as above.

%% It is possible to simplify the resulting expressions by extracting
%% outside the quantified formula the subgoals that have no free
%% variables. In a universally quantified expression $\forall
%% \overline{V}.~ cintneg(G_1 \vee ...\vee G_i \vee ... \vee G_n)$ if a
%% subgoal $G_i$ does not contain any free variable of $\overline{V}$
%% then the expression is equivalent to $cintneg(G_i) \wedge~ \forall
%% \overline{V}. cintneg(G_1 \vee ....\vee G_{i-1} \vee G_{i+1} \vee...
%% \vee G_n)$.

%% The transformation has some similarities with the transformation
%% proposed in \cite{Bruscoli}, but there are still some differences. The
%% transformation of \cite{Bruscoli} has a much more simple formulation
%% and some optimization covered by our detection of critical values are
%% not taken into account. The result is that much more universally
%% quantified goals are generated and the programs contain a lot of
%% trivial constraints (i.e. they are trivially true or false, as $X = a
%% ~\wedge~ X \neq a$, $X = a ~\vee~ X \neq a$).

\subsection{The Transformation}


The idea of the transformation to be defined below is to obtain a
program whose completion corresponds to the negation of the original
program, in particular to a representation of the completion where
negative literals have been eliminated. This can be formalized as the
successive application of several transformation steps:

\begin{enumerate}

\item For every completed definition of a predicate $p$ in
      $\PS_\Sigma$, $\forall \vecx.\, [p(\vecx) \iff D]$, add the
      completed definition of its negated predicate, 
      $\forall \vecx.\, [\negp(\vecx) \iff \neg D]$.

\item Move negations to the right of ``$\onlythen$'' using lemma
      \ref{implicative-form2}. 

\item If negation is applied to a existential quantification, replace 
      $\neg\exists\vecz. C$ by $\forall\vecz.\neg C$.

\item Replace $\neg C$ by its \emph{negated normal form} 
      $\NNF(\neg C)$.

\item Replace $\neg \true$ by $\false$, $\neg \false$ by $\true$ and,
      for every predicate symbol $p$, $\neg p(\vect)$ by
      $\negp(\vect)$.

\end{enumerate}


\begin{definition}The syntactic transformation $\negrhs$ is defined as
      follows:
\begin{align*}
\negrhs(P;Q) &= \negrhs(P),\negrhs(Q)\\
\negrhs(P,Q) &= \negrhs(P);\negrhs(Q)\\
\negrhs(\true) &= \false\\
\negrhs(\false) &= \true\\
\negrhs(p(\vect)) &= \negp(\vect)
\end{align*}
\end{definition}

\begin{definition}[Constructive Intensional Negation]
  For every predicate $p$ in the original program, assuming
  $\df_P(p)=\{p(\vecx) \neck B_i(\vecy_i\cdot\vecz_i) \where c_i(\vecx
  \cdot \vecy_i) | i \in 1 \ldots m\}$, the following clauses will be
  added to the negated program:
\begin{itemize}
\item If the set of constraints
$\{\exists\vecy_i.c_i(\vecx.\vecy_i)\}$ is not exhaustive, a clause
\[ \negp(\vecx) \neck \where \bigwedge_1^m \neg\exists\vecy_i.c_i(\vecx.\vecy_i)\]

\item If $\vecz_j$ is empty, the clause
\[ \negp(\vecx) \neck \negrhs(B_j(\vecy_j)) \where \exists\vecy_j.c_j(\vecx.\vecy_j)\]

\item If $\vecz_j$ is not empty, the clauses
\[ \negp(\vecx) \neck \paratodo([\vecz_j],p\_j(\vecy_j\cdot\vecz_j))
   \where \exists\vecy_j.c_j(\vecx.\vecy_j)\]
\[ p\_j(\vecy_j\cdot\vecz_j) \neck \negrhs(B_j(\vecy_j\cdot\vecz_j)) \]
\end{itemize}
\end{definition}


\subsection{Overlapping Definitions}
When the definition of some predicate has overlapping clauses the
simplicity of the transformation above is lost. Rather than defining a
different scheme for overlapping rules we will give a
translation of general sets of clauses into nonoverlapping ones:

\begin{lemma}\label{overlapping}Let $p$ be such that
\[\df_P(p)=\{p(\vecx) \neck B_i(\vecy_i\cdot\vecz_i) 
                     \where c_i(\vecx \cdot \vecy_i) 
             | i \in 1 \ldots m\}
\]
and there exist $j,k \in 1 \ldots m$ such that 
$\exists \vecy_j.\, c_j(\vecx\cdot\vecy_j)$ and 
$\exists \vecy_k.\, c_k(\vecx\cdot\vecy_k)$ are compatible.
Then the $j$-th and $k$-th clauses can be replaced by the clause
%
\[ p(\vecx) \neck B_j(\vecy_j\cdot\vecz_j);B_k(\vecy_k\cdot\vecz_k)
   \where c_j(\vecx \cdot \vecy_j) \wedge c_k(\vecx \cdot \vecy_k) \]
%
   and the following additional clauses in case their constraints were
   not equivalent to $\false$
\[ p(\vecx) \neck B_j(\vecy_j\cdot\vecz_j)
   \where c_j(\vecx \cdot \vecy_j) \wedge \neg c_k(\vecx \cdot \vecy_k) \]
and
\[ p(\vecx) \neck B_k(\vecy_k\cdot\vecz_k)
\where \neg c_j(\vecx \cdot \vecy_j) \wedge c_k(\vecx \cdot \vecy_k)
\] without changing the standard meaning of the program. The process
can be repeated a finite number of times if there are more than two
overlapping clauses.
\end{lemma}


\subsection{Examples}


Let us show some motivating examples:

\newcommand{\lesser}{\mathit{lesser}}

\begin{example}
The predicate \texttt{lesser/2} can be defined by the following
nonoverlapping set of clauses
%
\begin{prolog}
lesser(0,s(Y))\\
lesser(s(X),s(Y)) $\neck$ lesser(X,Y)
\end{prolog}
%
or, normalizing,
%
\begin{prolog}
lesser(N,M) $\neck$ $\where$ N=0, M=s(Y)\\
lesser(N,M) $\neck$ lesser(X,Y) $\where$ N=s(X), M=s(Y)
\end{prolog}
%
By lemma~\ref{completion}, the completed definition of \texttt{lesser}
is
%
\[\begin{array}{rl} 
   \cdf_P(lesser) \equiv \forall n,m.\, \lesser(n,m) \iff
   &\exists y. n=0 \wedge m=s(y) \onlythen \true \parels\\
   &\exists x,y. n=s(x) \wedge m=s(y) \onlythen \lesser(x,y) \parels\\ 
   &m=0 \onlythen \false
  \end{array}
\]
%
We have assumed that the constraint $\neg(\exists y. n=0 \wedge
m=s(y)) \wedge \neg(\exists x,y. n=s(x) \wedge m=s(y))$ has been
simplified to $m=0$ (if the only symbols are $0$ and $s/1$).

The constructive intensional negation of \texttt{lesser} is
%
\[\begin{array}{ll} 
   \cdf_P(neg\_lesser) \equiv \\
                      \forall n,m.\, \mathit{neg\_lesser}(n,m) \iff 
    &\exists y. n=0 \wedge m=s(y) \onlythen \false \parels\\ 
    &\exists x,y. n=s(x) \wedge m=s(y) \onlythen \mathit{neg\_lesser}(x,y) \parels\\ 
    &m=0 \onlythen \true
  \end{array}
\]
%
And the generated Prolog program is:
\begin{prolog}
neg\_lesser(N,M) $\neck$ $\where$ N=0, M=s(Y)\\
neg\_lesser(N,M) $\neck$ lesser(X,Y) $\where$ N=s(X), M=s(Y)
\end{prolog}

\end{example}

\newcommand{\ancestor}{\mathit{ancestor}}

\begin{example}


The second example, {\em family}, is also well known, and includes free
variables in the rhs of the definition of the predicate \texttt{ancestor/2}:

\vspace{-3pt}
\begin{tt}
\begin{mytabbing}
~~~~\=parent(john, mary) ~~~~\= ancestor(X, Y) $\leftarrow$ parent(X, Y) \\
    \>parent(john, peter) \> ancestor(X, Y) $\leftarrow$ \= parent(Z, Y) $\wedge$ \\ 
    \>parent(mary, joe)   \>                    \> ancestor(X, Z) \\
    \>parent(peter, susan) 
\end{mytabbing}
\end{tt}
\vspace{-3pt}
\noindent
%% The transformation of the predicate \emph{ancestor} has no complement
%% clause. The first and the second clause have an obvious critical
%% value $(X, Y)$. The associated clause is:

%% \vspace{-3pt}
%% \begin{tt}
%% \begin{mytabbing}
%% ~~~~\=cintneg(ancestor(X,Y)) $\leftarrow$ \=  $\forall$ Z.~(\= cintneg(parent(Z,Y) $\wedge$ ancestor(X,Z))) $\wedge$\\
%%     \>                          \> cintneg(parent(X,Y))
%% \end{mytabbing}
%% \end{tt}
%% \vspace{-3pt}


% In principle, we need to transform each of the clauses of the
% predicate, including
% the constraint $(X, Y) \neq(X, Y)$ in their bodies, but it
% is trivially unsatisfiable and we can omit the clauses.

%
or, normalizing,
%
\begin{prolog}
ancestor(N,M) $\neck$ parent(X,Y) $\where$ N=X, M=Y\\
ancestor(N,M) $\neck$ parent(Z,Y), ancestor(Y,X) $\where$ N=X, M=Y
\end{prolog}
%
By lemma~\ref{completion} we obtain a completed definition of
\texttt{ancestor} that we have replaced by lemma~\ref{overlapping}
obtaining
%
\[\begin{array}{ll} 
   \cdf_P(neg\_ancestor) \equiv \\
   \forall n,m.\, \mathit{neg\_ancestor}(n,m) \iff
   &\exists x,y. n=x \wedge m=y \onlythen \\ 
   &\mathit{neg\_parent}(n,m),forall([z],\ancestor-1(x,y,z))
  \end{array}
\]
%
%
\[\begin{array}{ll} 
   \cdf_P(ancestor-1) \equiv \\
   \forall n,m,r.\, \ancestor-1(n,m,r) \iff
   &\exists x,y,z. n=x \wedge m=y \wedge r=z\onlythen\\ 
                                                &    \mathit{neg\_parent}(z,y);
                                                    \mathit{neg\_ancestor}(x,z)
  \end{array}
\]

\end{example}


% Let us discuss the application of the method in a couple of
% examples. Consider the fragment of the program:

% %\vspace{-5pt}
% \begin{tt}
% \begin{mytabbing}
% ~~~~\=less(0, s(Y)) \\
%     \>less(s(X), s(Y)) $\leftarrow$ less(X, Y)
% \end{mytabbing}
% \end{tt}
% %\vspace{-5pt}

% First of all, we need to compute the complements of the terms in
% the lhs of the clauses, with respect to the variable pair
% $(W, Z)$. We have: 
% $$Comp((W, Z), (0, s(Y))) = Comp(W, 0) \vee Comp(Z, s(Y)) = $$
% $$(W \neq 0) ~ \vee ~ (\forall~ Y. ~Z \neq s(Y))$$
% $$Comp((W, Z), (s(X), s(Y))) = Comp(W, s(X)) \vee Comp(Z, s(Y)) = $$
% $$(\forall~ X.~ W \neq s(X))~ \vee~ ( \forall~ Y.~ Z \neq s(Y))$$
% %\begin{itemize}
% %\item $Comp (0, s(Y)) = (W \neq 0, \forall~ Y (Z \neq s(Y)))$
% %\item $Comp (s(X), s(Y)) = (\forall~ X (W \neq s(X)),
% %                              \forall~ Y (Z \neq s(Y)))$
% %\end{itemize}
% %
% %\begin{eqnarray*}
% % Comp(0, s(Y)) & = & (W \neq 0, \forall~ Y (Z \neq s(Y))) \\
% % Comp (s(X), s(Y)) & = & (\forall~ X (W \neq s(X)),
% % \forall~ Y (Z \neq s(Y)))   
% %\end{eqnarray*}
% The complement clause is:
% \vspace{-3pt}
% \begin{tt}
% \begin{mytabbing}
% ~~~~\=cintneg(less(W, Z),true) $\leftarrow$ \= $\forall$ J .~( (W, Z)  $\neq$ (0, s(fA(J))) ) $\wedge$ \\
%     \>                           \> $\forall$ X,Y .~( (W, Z)  $\neq$ (s(fA(X)), s(fA(Y))) ) \\
% \end{mytabbing}
% \end{tt}
% \vspace{-3pt}
% \noindent
% It can be simplified. There are no critical values and the first
% clause has no rhs, so the only transformed clause is for the second
% clause:

% \vspace{-3pt}
% \begin{tt}
% \begin{mytabbing}
% ~~~~\=cintneg(less(s(X), s(Y))) $\leftarrow$ cintneg(less(X, Y))
% \end{mytabbing}
% \end{tt}
% \vspace{-3pt}

\noindent

\subsection{Formal results}
\label{formal}

In order to establish the correctness of our transformation we can rely
on the formal properties of \cite{Bruscoli}. The main result is that our
transformation is essentially equivalent (although more efficient) that
\cite{Bruscoli} one.

\begin{Theo}
  Under the three-valued logic the program completion of {\it
    CompNeg$(P)$} and {\it CIntNeg$(P)$} are equivalent logic
  formulas.
\end{Theo}

This allows to apply the fixpoint semantics to our programs. Let us
now discuss how to handle Herbrand constraints and universally
quantified goals. We will show that our implementation computes the
same answers that {\sc sld$^\forall$}.

% %%%%%%%%%%%%%%%%%%%%%%%%%%%%%%%%%%%%%%%%%%%%%%%%%%%%%%%%%%%%%%%%%%
% %%%%%%%%%%%%%%%%  IMPLEMENTATION ISSUES %%%%%%%%%%%%%%%%%%%%%%%%%%
% %%%%%%%%%%%%%%%%%%%%%%%%%%%%%%%%%%%%%%%%%%%%%%%%%%%%%%%%%%%%%%%%%%

% \section{Implementation Issues}
% \label{implementation}

% We introduce some implementation issues that we will use in the
% negated predicates: the disequality constraints and the universal
% quantification. Later we will talk about the implementation of the
% transformation of the input program.

% %%%%%%%% MANAGMENT OF DISEQUALITY CONSTRAINTS   %%%%%%%%%%%%%%%%%

% \subsection{Management of Disequality Constraints}
% \label{disequality}

% An instrumental step in order to manage negation in a more advanced
% way is to be able to handle disequalities between terms such as $t_1
% \neq t_2$.  Prolog implementations typically include only the built-in
% predicate $\backslash ~== ~/2$ % es \ pero no me funciona
%  which can only work with disequalities if both
% terms are ground and simply succeeds in the presence of free variables.
% A ``constructive'' behavior must allow the ``binding'' of a variable
% with a disequality. On the other hand, the
% negation of an equation $X = t(\overline{Y})$ produces the universal
% quantification of the free variables in the equation, unless a more
% external quantification affects them. The negation of such an equation
% is $\forall~ \overline{Y}.~X \neq t(\overline{Y})$.
% % He borrado la referencia a Carlsson
% As we explained in \cite{SusanaPADL2000}, the inclusion of
% disequalities and constrained answers has a very low cost. It
% incorporates negative normal form constraints instead of bindings and
% the decomposition step can produce disjunctions. More precisely, the
% normal form of constraints is:

% %\vspace{-20pt}
% \[ \underbrace{\bigwedge_i (X_i = t_i)}_{\mbox{positive information}} \wedge~~~~ (
% \underbrace{\bigwedge_j \forall~ \overline{Z}_j^1 . ~(Y_j^1 \neq s_j^1)
% \vee \ldots \vee \bigwedge_l \forall~ \overline{Z_l}^n . ~(Y_l^n
% \neq s_l^n) )}_{\mbox{negative information}} \]
% %\vspace{-20pt}

% \noindent
% where each $X_i$ appears only in $X_i = t_i$, none $s_k^r$ is $Y_k^r$
% and the universal quantification could be
% empty (leaving a simple disequality).

% Using some normalization rules we can obtain a normal form
% formula from any initial formula. It is easy to redefine
% the unification algorithm to manage constrained variables.

% This very compact way to represent a normal form was firstly presented in
% \cite{Moreno1} and differs from Chan's representation where
% only disjunctions are used\footnote{Chan treats the disjunctions
% by means of backtracking. The main advantage of our normal form is
% that the search space is drastically reduced.}.

% Therefore, in order to include disequalities into a Prolog compiler we
% need to reprogram unification. It is possible if the Prolog version
% allows attributed variables \cite{Carlsson} (e.g. in Sicstus Prolog,
% or in Eclipse where they are called meta-structures). These variables
% let us keep associated information with each variable during the
% unification what can be used to dynamically control the constraints.

% Attributed variables are variables with an associated attribute, which
% is a term. We will associate to each variable a data structure
% containing a normal form constraint. Basically, a list of list of
% pairs (variable, term) is used. They behave like ordinary variables,
% except that the programmer can supply code for unification, printing
% facilities and memory management.  In our case, the printing facility
% is used to show constrained answers. The main task is to provide a new
% unification code.

% Once the unification of a variable $X$ with a term $t$ is
% triggered, there are three possible cases (up to
% commutativity):

% %\vspace{-5pt}
% \begin{enumerate}
% %\addtolength{\itemsep}{-8pt}
% \item if $X$ is a free variable and $t$ is not a variable with a negative
% constraint, $X$ is just bound to $t$,
% \item if $X$ is a free variable or bound to a term $t'$ and
% $t$ is a variable $Y$ with a negative constraint, we need to check
% if $X$ (or, equivalently, $t'$) satisfies the constraint associated with $Y$.
% A conveniently defined predicate \emph{satisfy} is used for this purpose,
% \item if $X$ is bound to a term $t'$ and $t$ is a term (or a variable
% bound to a term), the classical unification algorithm can be used.
% \end{enumerate}
% %\vspace{-5pt}

% We have defined a predicate $=/=~ /2$ \cite{SusanaPADL2000}, used to
% check disequalities, in a similar way to explicit unification ($=$).
% Each constraint is a disjunction of conjunctions of disequalities that
% are implemented as a list of lists of terms as $T_1/T_2$ (that
% represents the disequality $T_1 \neq T_2$). When a universal
% quantification is used in a disequality (e.g., $\forall Y.~X \neq
% c(Y)$) the new constructor $fA/1$ is used (e.g., $X/c(fA(Y))$).

% The main list is used to represent disjunctions while the inside list
% represents the conjunction of disequalities.  We focus on the variable
% $X$ and we show in the Table \ref{tab:table2} an example of
% correspondence between constraints over X and the attributes that we
% manage. The predicate $forall/2$ is a the implementation of the
% universal quantification that we will see in the next section.

% \begin{table}[h]
% %\hspace{-0.5cm}
% \begin{center}
% \begin{tabular}{lll}
% SUBGOAL & ATTRIBUTE & CONSTRAINT \\
% \hline\hline
% \ \\
% {\tt cintneg(member(X,[1,2,3]))}   &  $[[X/1,X/2,X/3]]$  & $X \neq 1 \wedge X \neq 2 \wedge X \neq 3$\\
% {\tt member(X,[1,2,3]), X=/=2} &  $[[X/1,X/3]]$       & $X \neq 1 \wedge X \neq 3$\\
% {\tt member(X,[1]), X=/=1}     &  {\tt fail}          & $false$ \\
% {\tt X =/= 4}                   & $[[X/4]]$            & $X \neq 4$ \\
% {\tt X =/= 4; X=/=5}            & $[[X/4], [X/5]]$     & $X \neq 4 \vee X \neq 5$ \\
% {\tt X =/= 5; (X=/=6, X=/=Y)}   & $[[X/4],[X/6, X/Y]]$ & $X \neq 4 \vee (X \neq 6 \wedge X \neq Y)$\\
% {\tt forall(Y, X =/= s(Y)))}     & $[[X/s(fA(Y)]]$  & $\forall Y. X \neq s(Y)$ \\
% {\tt cintneg(natural(X))}     & $[[X/0,X/s(fA(Y)]]$            & $X \neq 0 \wedge \forall Y. X \neq s(Y)$

% \end{tabular}
% \vspace*{3mm}
% \caption{Attribute representation of constraints}
% \label{tab:table2}
% %\end{small}
% \end{center}
% \hspace{-2cm}
% \end{table}

% %%%%%%%% IMPLEMENTATION OF UNIVERSAL QUANTIFICATION %%%%%%%%%%%%%

\section{Universal Quantification}
\label{quantification}

The efficient implementation of universally quantified goals is not an
easy task. In fact, it is considered as an undecidable
problem. However, we were not interested in a complete implementation
but a implementation powerful enough to resolve the quantifications
that we obtain from the program transformation of the intensional
negation.

Our implementation is based on two ideas:
%\vspace{-5pt}
\begin{enumerate}
%\addtolength{\itemsep}{-8pt}
\item A universal quantification of the goal $Q$ over a
variable $X$ succeeds when $Q$ succeeds without
binding (or constraining) $X$.
\item A universal quantification of $Q$ over $X$ is true
if $Q$ is true for all possible values for the variable $X$.
\end{enumerate}

Instead of generating all possible values (which is not possible in
the presence of a constructor of arity greater than 0) we can generate
all the possible skeletons of values, using new variables.  The
simplest possibility is to include all the constants and all the
constructors applied to fresh variables. Now, the universal
quantification is tested for all this terms, using the new variables
in the quantification. Formally we can express this in the formula:
\[ \forall X.~Q(X) \equiv Q(Sk) \vee 
                 [ \forall \overline{X_1}.~Q(c_1(\overline{X_1})) 
                   \wedge ... \wedge
                   \forall \overline{X_n}.~Q(c_n(\overline{X_n})) ] 
\]
where $Sk$ is a constant of Skolem, that is a constant different from
any symbol of the Herbrand Universe of the problem and $c_1,...,c_n$
are the constructors of the domain (each of them with its
corresponding arity).

The second argument of the disjunction can be combined with the first
one in order to get an implementation. The first case will be true
when $Q$ succeeds without unifying $X$ to any nonvar term. The second
case evaluate all the posible unifications of $X$ with any of the $n$
constructors.


In order to formalize this concept, we need the notion of {\bf
covering}.

\begin{definition} [Covering]
A \emph{Covering of the Herbrand Universe} is any sequence of terms 
$\{t_1, \ldots, t_n\}$ such that:
%\vspace{-5pt}
\begin{itemize}
%\addtolength{\itemsep}{-8pt}
\item For every $i, j$ with $i \neq j$, $t_i$ and $t_j$ do
not superpose, i.e. there is no ground substitution $\sigma$
with $t_i\sigma = t_j\sigma$.
\item For all ground term $s$ of the Herbrand Universe there
exists $i$ and a ground substitution $\sigma$ with
$s = t_i\sigma$.
\end{itemize}
\end{definition}

The simplest covering is a variable $C_1=\{X\}$. If the program
uses only natural numbers, the following sets of terms are
coverings:
%\vspace{-5pt}
\begin{itemize}
%\addtolength{\itemsep}{-8pt}
\item $C_2=\{0, s(X)\}$
\item $C_3=\{0, s(0), s(s(X))\}$
\item $C_4=\{0, s(0), s(s(0)), s(s(s(X)))\}$
\end{itemize}

The example also gives us the hint about how to
incrementally generate coverings. We start from the simplest
covering $X$. From one covering we generate the next one
choosing one term and one variable of this term. The term is
removed and then we add all the terms obtained replacing the
variable by all the possible instances of that element. 

In order to fix a strategy to select the term and the variable we use
a Cantor's diagonalization\footnote{This is the method to enumerate
$\N^m$. It ensures that all elements are visited in a finite number of
steps.} to explore the domain of a set of variables. It is a breadth
first strategy to cover every element of the domain. The previous
concepts extend trivially in the case of tuple of elements of the
Herbrand Universe, i.e. several variables.

A covering $C_i=\{ \overline{t_1}, ..., \overline{t_m}\}$ can be
evaluated for the goal $Q(\overline{X})$. This means that for each
element $\overline{t_j}$ in the covering $(j \in \{1..m\})$ we execute
$Q(\overline{t_j})$. For simplicity we will make the following
definitions for one only varible because it can be extended for a
tuple. Therefore we will evaluate the goal $Q(X)$ for any of the
possible values of $X$ that are all the values of a covering.

\begin{definition}[Next Covering]
  Let a covering $C=\{ t_1, ..., t_m \}$, the \emph{next covering}
  to $C$ over the variable $X$, $Next(C,X)=\{ t_1,...,t_{j-1},
  t_{j+1},..., t_m, {t_j}_1,...,{t_j}_n \}$ where ${t_j}_k=t_j
  \sigma_k, ~ j \in \{1..m\},~ k \in \{1..n\}$ and the sustitution
  $\sigma_k=\{X \rightarrow c_k(\overline{X_k}\}$ if $c_1,...,c_n$ are the
  constructors of the domain (each of them with its corresponding
  arity, the number of variables of $\overline{X_k}$). The $j$ is
  chosen following a Cantor's diagonalization to obtain a fair
  selection rule.
\end{definition}

\begin{definition}[Sequence of Coverings]
  $\mathcal{C}$ is a \emph{sequence of coverings} if $\mathcal{C}=
  \{C_1,...,C_n\}$ where for each $C_i$ with $i \in \{1..n-1\}$,
  $C_{i+1}=Next(C_i,X_i)$, and $X_i$ is the first variable that we
  find if we examine the terms of $C_i$ from the left to the right.

%  If there is no variables in $C_i$ then $C_{i+1}=C_i$
\end{definition}

\begin{definition}[Evaluation of a Covering]
The \emph{evaluation of a covering} $C$ for the goal $Q(X)$ is the
conjunction of the results of evaluate the goal $Q(X)$ with all the
possible values of $X$ that are the elements of a covering $C$. That
is
\[ \xi(C,Q(X)) = Q(t_1) \wedge ... \wedge Q(t_m)\]
where $C=\{ t_1, ..., t_m\}$
\end{definition}

\begin{definition}[Meaning of the Evaluation of a Covering]
  The \emph{meaning of the evaluation of a covering} $C_i=\{ t_1, ...,
  t_m\}$ of a sequence of coverings $\mathcal{C}$ for a goal $Q(X)$
  over the variable $X$ is one of these cases:
%\vspace{-5pt}
\begin{enumerate}
%\addtolength{\itemsep}{-8pt}

    \item if $Q(t_i)= \true$ for all $i \in \{1..m\}$
without any bind of the variables introduced by the covering, then the
meaning of the evaluation of the covering
succeeds. \[ \|\xi(C_i,Q(X))\| = \true \]

    \item otherwise if $Q(t_i)= \false~$ for any $i \in
\{1..m\}$ such that $t_i$ is ground, then the meaning of the
evaluation of the covering fails. \[ \|\xi(C_i,Q(X))\| =
\false~ \]

\item otherwise if $Q(t_i)= \false~$ for any $i \in \{1..m\}$ such
  that $t_i$ is not ground. In this case we have tried to unify one of
  the variables of $t_i$ and to obtain the meaning of the evaluation
  of the covering we must obtain the meaning of the evaluation of the
  next covering of $\mathcal{C}$.
\[ \|\xi(C_i,Q(X))\| = \|\xi(C_{i+1},Q(X))\|~ \]

    \item otherwise if $Q(t_i)= \underline{\mathrm{u}}~$ for any $i \in
  \{1..m\}$ such that $t_i$ is ground, then the meaning of the
  evaluation of the covering is unknown. \[ \|\xi(C_i,Q(X))\| =
  \underline{\mathrm{u}}~ \]

\end{enumerate}
The selection of the elements $t_i$ of the covering for its evaluation
$Q(t_i)$ is made using a fair selection rule that assures us that if
for any ot the terms of the covering $Q(t_i)$ fails, then that term
will be selected in a finite time.
\end{definition}

\begin{lemma}
  If the meaning of an evaluation of a covering $C_i$ in the sequence
  $\mathcal{C}$ is unknown for the goal $Q(X)$ then the meaning of the
  evaluation of the following covering $Next(C_i,X)=C_{i+1}$ for the
  same goal is also unknown.
\[ \|\xi(C_i,Q(X))\| = \underline{\mathrm{u}} ~ \rightarrow ~
   \|\xi(C_{i+1},Q(X))\|= \underline{\mathrm{u}} \]
\end{lemma}

\begin{proof}
  When it is chosen a term of the covering such that $Q(t_i)$ is
  unknown it is because for any other term $t_j$ of the covering
  $Q(t_j)$ fails due to the use of a fair selection rule it would have
  been selected before.

  If the meaning of the evaluation of a covering is unknown, it is
  because it is unknown for at least one of the terms of the covering
  and true or unknown for the rest of them.  
  
  For a universe with n different constructors $c_1,...,c_n$, we can
  obtain the next covering expanding one term $t$ over the variable
  $X$ in $t_1,...,t_n$ such that $t_k=t \sigma_k, ~ k \in \{1..m\}$
  and the unification $\sigma_k=\{X \rightarrow c_k(\overline{X_k}\}$. 
  
  $$\|Q(t)\|= \underline{\mathrm{u}} \rightarrow 
  \forall k \in {1..m}.~
  \|Q(t_k)\|= \underline{\mathrm{u}} $$
  
  $$\|Q(t)\|= \true \rightarrow 
  \forall k \in {1..m}.~
  \|Q(t_k)\|= \true $$

  When we the next covering $C_{i+1}$ is generated from a covering
  $C_i$ such that its meaning is unknown, the meaning of the obtained
  covering will be unknown for the definition of the selection rule.
  Formally

$$ (\|\xi(C_i,Q(X))\| = \underline{\mathrm{u}} ~ \equiv $$
$$   \forall t_j \in C_i.~ (\|Q(t_i)\|= \underline{\mathrm{u}} \vee 
                         \|Q(t_i)\|= \true) \wedge
   \exists t_k \in C_i.~ \|Q(t_k)\|= \underline{\mathrm{u}}) \rightarrow ~ $$
$$   \|\xi(C_{i+1},Q(X))\|= \underline{\mathrm{u}} $$

\end{proof}


\subsection{Implementation Issues}
\subsubsection{Disequality Constraints}
\subsubsection{Implementation of Universal Quantification}

We implement the universal quantification by means of the predicate%\linebreak 
$forall([X1,\ldots,Xn],~ Q)$, where $X1,\ldots,Xn$ are the
universal quantified variables and $Q$ is the goal to quantify. We
start with the initial covering $\{(X_1,\ldots,X_n)\}$ of depth 1.

There are two important details that optimize the execution.
The first one is that in order to check if there are bindings
in the covering variables, it is better to replace them by
new constants that do not appear in the program. In other words,
we are using ``Skolem constants''.

The second optimization is much more useful. Notice that the
coverings grow up incrementally, so we only need to check the
most recently included terms. The other ones have been checked
before and there is no reason to do it again.

As an example, consider a program which uses only natural
numbers: the sequence of coverings for the
goal $\forall~ X,Y,Z. ~ p(X,Y,Z)$ will be the following
 (where $Sk(i)$, with $i$ a number, represents the ith
Skolem constant).
%\vspace{-3pt}
\begin{small}
\begin{mytabbing}
$C_1 = [(Sk(1),Sk(2),Sk(3))]$ \\
$C_2 = [\underline{(0,Sk(1),Sk(2))}, \underline{(s(Sk(1)),Sk(2),Sk(3))}]$ \\
$C_3 = [\underline{(0,0,Sk(1))}, \underline{(0,s(Sk(1)),Sk(2))},
        (s(Sk(1)),Sk(2),Sk(3))]$ \\
$C_4 = [\underline{(0,0,0)}, \underline{(0,0,s(Sk(1)))},
        (0,s(Sk(1)),Sk(2)),(s(Sk(1)),Sk(2),Sk(3))]$ \\
$C_5 = [(0,0,0), \underline{(0,0,s(0))}, \underline{(0,0,s(s(Sk(1))))},
        (0,s(Sk(1)),Sk(2)),$ \\
~~~~~~~~$        (s(Sk(1)),Sk(2),Sk(3))]$\\
$C_6 = \ldots$
\end{mytabbing}
\end{small}
%\vspace{-3pt}

In each step, only two elements need to be checked, those that appear
underlined. The rest are part of the previous covering and they
do not need to be checked again. 
%Again, the authors can supply
%details of the code (or see \cite{SusanaTFC}).

Let us show some examples of the use of the \emph{forall} predicate,
indicating the covering found to get the solution. We are still
working only with natural numbers:
% and we are going to consider a
%maximal depth of 5:

%\vspace{-8pt}
\begin{tt}
\begin{mytabbing}
~~~~\=$|$ ?- \= forall([X], even(X)). \\
    \>       \> no 
\end{mytabbing}
\end{tt}
%\vspace{-8pt}

\noindent
with the covering of depth 3 $\{0, \underline{s(0)}, s(s(Sk(1))\}$.

%\vspace{-8pt}
\begin{tt}
\begin{mytabbing}
~~~~\=$|$ ?- \= forall([X], X =/= a). \\
    \>       \> yes
\end{mytabbing}
\end{tt}
%\vspace{-8pt}

\noindent
with the covering of depth 1 $\{Sk(1)\}$.

%\vspace{-8pt}
\begin{tt}
\begin{mytabbing}
~~~~\=$|$ ?- \=forall([X], less(0, X) -$>$ less(X, Y)). \\
    \>       \>Y = s(s(\_A))
\end{mytabbing}
\end{tt}
%\vspace{-8pt}

\noindent
with the covering of depth 2 $\{0, s(Sk(1))\}$.
%\bigskip
%\vspace{-8pt}
\begin{tt}
\begin{mytabbing}
~~~~\=$|$ ?- \=forall([X], (Y=X ; lesser(Y, X))). \\
    \>       \>Y = 0
\end{mytabbing}
\end{tt}
%\vspace{-8pt}

\noindent
with the covering of depth 2 $\{0, s(Sk(1))\}$.
%\bigskip


% The general solution does not guarantee completeness of the query
% evaluation process. There are some cases when the generation of
% coverings does not find one which is correct or incorrect.
% Nevertheless, this solution fails to work properly in very particular
% cases.  Remember that we are not interested in giving the user an
% universal quantification operator, but just to implement the code
% coming from the transformation of a negated predicate. In order to
% have a practical use of the method, we restrict the depth of the
% coverings to some (user defined) constant $d$. If the {\em forall/4}
% predicate is not able to achieve a solution at this covering
% generation depth, the predicate informs that it is not possible to
% know the result of the computation at that depth $d$ ({\tt S =
%   unknown}).

% We can formally prove that in those cases when a result is obtained
% with the {\em forall/4} predicate it is equivalent to the one obtained
% by {\sc sld$^{\forall}$} (i.e. {\sc sld} with universal quantification).

% \begin{Theo}
% For any universally quantified goal in a program $P$ such that 

% \[ \mbox{\em forall} ([X_1, \ldots, X_n], p (t), n, V) ~\leadsto_{P, \sigma} ~ \Box \]

% \noindent
% for any $n$, if $\sigma(V) =$ {\tt success}, we have $~\forall X_1,
% \ldots, X_n. p (s) \sigma ~ \leadsto^{\forall}_{P} ~ \Box~ $; and if
% $\sigma(V) =$ {\tt fail}, we have that the goal $\forall X_1, \ldots,
% X_n. p (s)\sigma$ has no successful {\sc sld$^{\forall}$}
% derivations.
% \end{Theo}

As we have seen in the definition of the quantifier, this
implementation is correct and if we use a fair selection rule to chose
the order in which the terms of the covering are visited then it is
also complete. We can implement this because in Ciao Prolog is posible
to ensures AND-fairness by goal shuffling \cite{CIAO}. 

If we implement this in a Prolog compiler using depth first
SLD-resolution, the selection will make the process incomplete. When
we are evaluating a covering $C=\{ t_1, ..., t_m \}$ for a goal $Q(X)$
if we use the depth first strategy from the left to the right, we can
find that the evaluation of a $Q(t_j)$ is unknown and is there exists
one $k > j$ such that $t_k$ is ground and $Q(t_k)$ fails, then we
would obtain that $forall([X],Q(X))$ is unknown when indeed it is
false.


% %%%%%%%%%%%%%%%% RESULTING IMPLEMENTATION   %%%%%%%%%%%%%%%%%%%%%

% \subsection{Resulting Implementation}
% \label{resulting}


% We have implemented a transformation of the input program that adds
% explicit negation. It is possible to expand the code during the
% execution thanks to the packages system of Ciao
% \cite{ciao-modules-cl2000}. If a Prolog Program with explicit
% negations loads the package $intneg.pl$ the initial program will be
% enlarged with a new predicate $cintneg/2$ that is the implementation
% of the complement of the code of every predicate of the input program
% that can be negated at runtime.

% % So, a call to $cintneg(p(X,Y),S)$ is
% % equivalent to the call $cintneg(p(X,Y))$ but with an additional
% % argument that serves to return $true$ if the negation has a solution
% % (and the solution will be obtained too, of course), $fail$ if the
% % negation fails and $unknown$ if there is a universal quantification
% % in the execution of the negation and it does not finish the generation
% % of coverings in a finite number or steps.

% % With this simple mechanism we can use intensional negation to negate
% % any goal and the worst case would be that the result would be
% % $unknown$ and we would have to continue negating with another
% % technique (constructive negation). Due to the inefficiency of the
% % constructive negation for general goals and the efficiency of
% % intensional negation , the overhead of this useless execution 
% % is worthy.

% \begin{table}[h]
% %\begin{small} 
% \begin{center}
% \begin{tabular}{||c|c|c|c||}
% \hline %------------------------------------------------------------------
% \hline %-------------------------------------------------------------------
% {\bf $\neg G_1$} &~~ {\bf $\neg G_2$} ~~& ~~{\bf $\neg G_1 \vee \neg G_2$}~~ &~~ {\bf $\neg G_1 \wedge \neg G_2$}~~ \\ 

% \hline %--------------------------------------------------------------
% T & T & T & T \\
% F & F & F & F \\
% U & U & U & U \\
% T & F & T & F \\
% T & F & T & F \\
% T & U & T & U \\
% U & T & T & U \\
% F & U & U & F \\
% U & F & U & F \\

% \hline %------------------------------------------------------------
% \hline %------------------------------------------------------------

% \end{tabular}
% \vspace*{3mm}
% \caption{Three-valued connectives}
% \label{tab:table1}
% %\end{small}
% \end{center}
% \end{table}

% To combine the results of negated goals three-valued logic connectives
% must be used as presented in table \ref{tab:table1}.  Obviously, these
% equivalences are satisfied:

% $$ \neg (G_1 \wedge G_2) \equiv \neg  G_1 \vee \neg G_2 $$
% $$ \neg (G_1 \vee G_2) \equiv \neg  G_1 \wedge \neg G_2 $$
% $$ \neg \neg G_1 \equiv G_1 $$

\subsubsection{Efficiency Evaluation}

\begin{table}[t]
%\begin{small} 
\begin{tabular}{||l|r|r|r|r||}
\hline %------------------------------------------------------------------
\hline %-------------------------------------------------------------------
{\bf goals} &~~ {\bf Goal} ~~& ~~{\bf naf(Goal) }~~ &~~ {\bf cneg(Goal)}~~ &~~ {\bf ratio}~~ \\ 

\hline %--------------------------------------------------
boole(1)                     &  2049      &  2099    &  2069   &   0.98   \\ 
\hline %--------------------------------------------------
boole(8)                     &  2070      &  2170    &  2590   &   1.19   \\ 
\hline %--------------------------------------------------
boole(X)                     &  2080      &  -       &  3109   &          \\ 
\hline %--------------------------------------------------
positive(s(s(s(s(s(s(0))))))~~~ &  2079      &  1600    &  2159   &   1.3    \\ 
\hline %--------------------------------------------------
positive(s(s(s(s(s(0))))))   &  2079      &  2139    &  2060   &   0.96   \\ 
\hline %--------------------------------------------------
positive(X)                  &  2020      &  -       &  7189   &          \\ 
\hline %--------------------------------------------------
greater(s(s(s(0))),s(0))     &  2110      &  2099    &  2100   &   1.00   \\ 
\hline %--------------------------------------------------
greater(s(0),s(s(s(0))))     &  2119      &  2129    &  2089   &   0.98   \\ 
\hline %--------------------------------------------------
greater(s(s(s(0))),X)        &  2099      &  -       &  6990   &          \\ 
\hline %--------------------------------------------------
greater(X,Y)                 &  7040      &  -       &  7519   &          \\ 
\hline %--------------------------------------------------
queens(s(s(0)),Qs)           &  6939      &  -       &  9119   &          \\ 

\hline %-------------------------------------------------------
\hline %-------------------------------------------------------
{\bf average}                &            &          &         &   1.06   \\ 
\hline %--------------------------------------------------------------------------------
\hline %--------------------------------------------------------------------------------
\end{tabular}
\vspace*{3mm}
\caption{Runtime comparation}
\label{table}
%\end{small}
\end{table}
 
 %% tab:table3

Let us show some experimental results very encouraging.  We present
three set of examples.  The first one collects some examples where it
is slightly worst or a little better to use intensional negation
instead of negation as failure. That is for implementation reasons for
sample goals.  They correspond to cases where the general strategy
will prefer negation as failure to execute the goal.

The second set contains examples in which negation as
failure cannot be used. Intensional negation is very efficient in the sense
that the execution time is similar as the time needed to execute 
the positive goal. 

% We have included an example of a goal that
% cannot be negated using intensional negation (constructive negation will
% be used instead in the general strategy).

The third set of examples is the most interesting because contains
more complex goals where negation as failure cannot be used and the
speed-up of intensional negation over constructive negation is over 5
times better.


% %%%%%%%%%%%%%%%%%%%%%%%%%%%%%%%%%%%%%%%%%%%%%%%%%%%%%%%%%%%%%%%%%%
% %%%%%%%%%%%%%%%%%%%%%%  EXAMPLES %%%%%%%%%%%%%%%%%%%%%%%%%%%%%%%%%
% %%%%%%%%%%%%%%%%%%%%%%%%%%%%%%%%%%%%%%%%%%%%%%%%%%%%%%%%%%%%%%%%%%

% \section{Examples}
% \label{examples}

% The code that is generated for the examples of the above section is:

% \begin{tt}
% \begin{mytabbing}
% ~~~~\=cintneg(even(X),true) :- X =/= 0, X =/= s(s(fA(Y))). \\
%     \> cintneg(even(s(s(X))),S) :- cintneg(even(X),S). \\
% \\
% ~~~~\=cintneg(less(W, Z),true) :- \= ( (W, Z)  =/= (0, s(fA(J))) ), \\
%     \>                           \> ( (W, Z)  =/= (s(fA(X)), s(fA(Y))) ). \\
% ~~~~\=cintneg(less(s(X), s(Y)),S) :- cintneg(less(X, Y),S). \\
% \\
% ~~~~\=cintneg(ancestor(X,Y),S) :- \= cintneg(parent(X,Y),S1), \\
%     \>                          \> forall([Z],cintneg((parent(Z,Y),ancestor(X,Z)),S2). \\
%     \>                          \> and(S1,S2,S).
% \end{mytabbing}
% \end{tt}

% Note that the $and/3$ is the three-valued version of conjunction
% (see Table \ref{tab:table1}). Next we have some examples coming from a
% running session that show the behavior at runtime:

% \begin{small}
% \begin{tt}
% \begin{minipage}[h]{10cm}
% \begin{mytabbing}
% \=$|$?- \=cintneg(even(s(s(0)))). \\
%     \>      \>no \\
% \>$|$?- cintneg(even(s(s(s(0))))). \\
%     \>      \>yes \\
% \>$|$?- cintneg(even(X)). \\
%     \>      \>X =/= 0,X=/=s(s(fA(\_A))) ?;\\
%     \>      \>X=s(s(Y)),Y=/=0, \\
%     \>      \>Y=/=s(s(fA(\_A))) ?; \\
%     \>      \> \dots \\ %\vdots \\
% \>$|$?- cintneg(less(0, s(X))). \\
%     \>      \>no \\
% \>$|$?- cintneg(less(s(X), 0)). \\
%     \>      \>yes 
% \end{mytabbing}
% \end{minipage}
% \begin{minipage}[h]{7.3cm}
% \begin{mytabbing}
% \=$|$?- \=cintneg(less(s(X),X)). \\
%     \>      \>... \\
% \>$|$?- \>cintneg(ancestor(mary,peter)).\\
%     \>      \>yes \\
% \>$|$?- cintneg(ancestor(john,X), S). \\
%     \>      \>S = true, X = john \\
% \>$|$?- cintneg(ancestor(peter,X), S). \\
%     \>      \>S = true, X =/= susan ? ; \\
%     \>      \>S = fail, X = susan  \\
% \>$|$?- cintneg((parent(X,mary), \\
%     \>      \>~~~~~~~parent(X,peter)), S). \\
%     \>      \>S = true, X =/= john ? ; \\
%     \>      \>S = fail, X = john 
% \end{mytabbing}
% \end{minipage}
% \end{tt}
% \end{small}

% %\smallskip
% The divergence of the goal $cintneg(less(s(X), X))$ is of the same
% nature of the divergence of $less(X, s(X))$ and is related to the
% incompleteness of Prolog implementations. Furtheremore, our
% implementation provides only sound results.

% % although
% % there are cases where we cannot provide any result.

% % In order to provide some heuristics to guide the computation
% % of the negation process we will use some information provided
% % by a global analysis of the source program.

%%%%%%%%%%%%%%%%%%%%%%%%%%%%%%%%%%%%%%%%%%%%%%%%%%%%%%%%%%%%%%%%%%
%%%%%%%% INTENSIONAL NEGATION IN THE COMPILER STRATEGY %%%%%%%%%%%
%%%%%%%%%%%%%%%%%%%%%%%%%%%%%%%%%%%%%%%%%%%%%%%%%%%%%%%%%%%%%%%%%%

\section{Conclusion and Future Work}
\label{strategy}

Intensional Negation is just a part of a more general project to
implement negation where several other approaches are used: negation
as failure possibly combined with dynamical goal reordering,
intensional negation, and constructive negation. The decision of what
approach can be used is fixed by the information of different program
analyses: naf in case of groundness of statically detected goal
reordering, finite constructive negation in case of finiteness of the
number of solutions, etc.  See \cite{SusanaLPAR01} for details.
Notice that the strategy also ensures the soundness of the method: if
the analysis is correct, the precondition to apply a technique is
ensured, so the results are sound.  

% Given a (sub)goal of the form
% $\neg G(\overline{X})$ the compiler produces one of the following
% codes: \emph{naf} ($G (\overline{X})$), \emph{cintneg} ($G
% (\overline{X}))$, \emph{cnegf} ($G (\overline{X}))$ or \emph{cneg} ($G
% (\overline{X}))$ in this preference order. Completeness is ensured by
% the completeness of constructive negation, the last technique to be
% used.  

When the transformation of our intensional negation will be
integrated into the general strategy we would optimize the
implementation by introducing in the negation of the rhs of the
predicates the general negation $neg/1$ or any other of the techniques
(for example $naf/1$ if the analysis recommends it).  We come back to
the family example at the end of the Section \ref{transformation}. In
case the analysis recommend constructive negation considering that
$parent(X,Y)$ has a finite number of solutions. The optimized code
will be:


\begin{tt}
\begin{mytabbing}
~~~~\=cintneg(ancestor(X,Y),S) :- \= cnegf(parent(X,Y)), \\
    \>                      \> forall([Z],(\= cnegf(parent(Z,Y)); \\
    \>                      \>             \> cintneg(ancestor(X,Z),S2)),S2), \\
    \>                      \> and(S1,S2,S).
\end{mytabbing}
\end{tt}

In the case we can infer that a call to $\neg parent(Z, Y)$ is ground,
and \emph{naf} can be used instead of \emph{cnegf}. The transformed
predicate ($cintneg(G (\overline(X))$) will be used in case of
recursive calls.

% We have to find a sufficient condition over the goal $G
% (\overline(X))$ that ensures the success of $cintneg(G
% (\overline(X)))$. While we are working on static analyses to answer
% the question we are using a dynamic approach.

%%%%%%%%%%%%%%%%%%%%%%%%%%%%%%%%%%%%%%%%%%%%%%%%%%%%%%%%%%%%%%%%%%
%%%%%%%%%%%%%%%%%%%%%%  FUTURE WORK %%%%%%%%%%%%%%%%%%%%%%%%%%%%%%
%%%%%%%%%%%%%%%%%%%%%%%%%%%%%%%%%%%%%%%%%%%%%%%%%%%%%%%%%%%%%%%%%%


Prolog can be more widely used for knowledge representation based
applications if negated goals can be included in programs.  We have
presented a collection of techniques \cite{SusanaPADL2000}, more or
less well known in logic programming, that can be used together in
order to produce a system that can handle negation efficiently.  In
our approach, intensional negation plays a significant role and a
deeper study of it have been presented.  To our knowledge it is one of
the first serious attempts to include such proposals as implmentations
into a Prolog compiler.

The main novelties of our constructive intensional negation approach is:
\begin{itemize}
\item The formulation in terms of disequality constraints, solving
  some of the problems of intensional negation \cite{Barbuti2} in a
  more efficient way than \cite{Bruscoli}.
\item The computation of universally quantified goals, that was
  sketched in \cite{Barbuti2}, and now we have discuss its soundness
  and completeness in this paper.
\end{itemize}

As a future work, we plan to include this implementation into a
compiler in order to produce a version of Ciao Prolog with negation.
It will give us a real measure of what important is the information of
the analyzers to help our strategy. 

% On the other hand, there are still
% some unsolved problems. The most important is the detection of the
% cases where the universal quantification does not work in order to
% avoid the overhead of the \emph{forall} predicate when it cannot find
% a solution.  

Another field of work is the optimization of the implementation of the
\emph{forall} using different search techniques and more specialized
ways of generating the coverings of the Herbrand Universe of our
negation subsystem.


%%%%%%%%%%%%%%%%%%%%%%%%%%%%%%%%%%%%%%%%%%%%%%%%%%%%%%%%%%%%%%%%%
%%%%%%%%%%%%%%%%%%%%%% BIBLIOGRAPHY %%%%%%%%%%%%%%%%%%%%%%%%%%%%%
%%%%%%%%%%%%%%%%%%%%%%%%%%%%%%%%%%%%%%%%%%%%%%%%%%%%%%%%%%%%%%%%%

\begin{small}

%\linespread{0.9}
  \bibliographystyle{plain} \bibliography{bibliography}

\end{small}


%%%%%%%%%%%%%%%%%%%%%%%%%%%%%%%%%%%%%%%%%%%%%%%%%%%%%%%%%%%%%%%%%


\end{document}

%%%%%%%%%%%%%%%%%%%%%%%%%%%%%%%%%%%%%%%%%%%%%%%%%%%%%%%%%%%%%%%%%
%%%%%%%%%%%%%%%%%%%%%%%  THE END  %%%%%%%%%%%%%%%%%%%%%%%%%%%%%%
%%%%%%%%%%%%%%%%%%%%%%%%%%%%%%%%%%%%%%%%%%%%%%%%%%%%%%%%%%%%%%%%%












