%%%%%%%%%%%%%%%%%%%%%%%%%%%%%%%%%%%%%%%%%%%%%%%%%%%%%%%%%%%
% Implementation of Constructive Negation for Prolog 
% (con el ingles corregido)
%%%%%%%%%%%%%%%%%%%%%%%%%%%%%%%%%%%%%%%%%%%%%%%%%%%%%%%%%%%

\documentclass{llncs}

%% \newenvironment{mytabbing}
%%    {\vspace{0.3em}\begin{small}\begin{tabbing}}
%%    {\end{tabbing}\end{small}\vspace{0.3em}}

%%\newenvironment{mytabbing}
%%   {\begin{tabbing}}
%%   {\end{tabbing}}

\usepackage{pst-node}

\usepackage{amsmath} %% for functions defined using different parts
\usepackage{amssymb}
\usepackage{latexsym}
\usepackage{times}
\usepackage{mathptmx}
\usepackage{theorem}


\newcommand{\naf}{{\em naf}}\newcommand{\viejo}[1]{}
\newcommand{\ciao}{Ciao}


\newcommand{\tab}{\hspace{2em}}
\newcommand{\ra}{$\rightarrow~$}
\newcommand{\Ra}{\Rightarrow~}
\newcommand{\HINT}{{\cal H\!-\!INT}}
%\newcommand{\cts}{\mid}
\newcommand{\cts}{~[\!]~}
\newcommand{\N}{I\!\!N}

\newcommand{\ToDo}[1]{
  \begin{center}
      \begin{minipage}{0.75\textwidth}
        \hrule
        \textbf{To do:}\\
        {\em #1}
        \hrule
      \end{minipage}\\
  \end{center}
}

%%%%%%%%%%%%%%%%%%%%%%%%%%%%
\begin{document}

%%%%%%%%%%%%%%%%%%%%%%%%%%%%%%

\title{A Real Implementation for \\
       Constructive Negation}

%\author{~~Susana Mu\~{n}oz~~~~~~~~~ Juan Jos\'{e} Moreno-Navarro \\
%         \email{susana@fi.upm.es}~~~~~~~~~~ \email{jjmoreno@fi.upm.es}}
\author{~~~~~~~~~~~~ Susana Mu\~{n}oz ~~~~ Juan Jos\'{e} Moreno-Navarro \\
 \email{susana@fi.upm.es}~~~~~ \email{jjmoreno@fi.upm.es}}

\institute{ 
LSIIS, Facultad de Inform\'{a}tica \\
Universidad Polit\'{e}cnica de  Madrid \\ 
Campus de Montegancedo s/n Boadilla del Monte\\
28660 Madrid, Spain \footnote{This work was partly supported by the
Spanish MCYT project TIC2000-1632.} \\
 }

\maketitle

%%%%%%%%%%%%%%%%%%%%%%%%
\vspace{-12pt}

\paragraph{\bf Keywords}
Constructive Negation, Negation in Logic Programming, Constraint Logic
Programming, Implementations of Logic Programming.


%%%%%%%%%%%%%%%%%%%%%%%%%%%%%%%%%%%%%%%%%%%%%%%%%%%%%%%%%%%%%%%%
%%%%%%%%%%%%%%%%%%%%%%%% INTRODUCTION %%%%%%%%%%%%%%%%%%%%%%%%%%
%%%%%%%%%%%%%%%%%%%%%%%%%%%%%%%%%%%%%%%%%%%%%%%%%%%%%%%%%%%%%%%%

\subsubsection{Introduction}
Logic Programming has been advocated as a language for system
  specification, especially for those involving logical behaviours,
  rules and knowledge. However, modelling problems involving negation,
  which is quite natural in many cases, is somewhat limitated if
  Prolog is used as the specification/implementation language. These
  restrictions are not related to theory viewpoint, where users can
  find many different models with their respective semantics; they
  concern practical implementation issues.  The negation capabilities
  supported by current Prolog systems are rather limited, and there is
  no correct and complete implementation. Of all the proposals,
  constructive negation \cite{Chan1,Chan2} is probably the most
  promising because it has been proven to be sound and complete
  \cite{Stuckey95}, and its semantics is fully compatible with
  Prolog's. Constructive negation was, in fact, announced in early
  versions of the Eclipse Prolog compiler, but was removed from the
  latest releases.  The reasons seem to be related to some technical
  problems with the use of coroutining (risk of floundering) and the
  management of constrained solutions.


%The  goal of this paper 
Our goal is to give an algorithmic description of
constructive negation, i.e. explicitly stating the details needed for
an implementation. We also intend to discuss the pragmatic ideas
needed to provide a concrete and real implementation. Early results
for a concrete implementation extending the \ciao\ Prolog compiler are
presented.  We assume some familiarity with constructive negation
techniques and Chan's papers.

%%%%%%%%%%%%%%%%%%%%%%%%%%%%%%%%%%%%%%%%%%%%%%%%%%%%%%%%%%%%%%%%%%
%%%%%%%%%%%%%%%   CONSTRUCTIVE NEGATION   %%%%%%%%%%%%%%%%%%%%%%%%
%%%%%%%%%%%%%%%%%%%%%%%%%%%%%%%%%%%%%%%%%%%%%%%%%%%%%%%%%%%%%%%%%%

\subsubsection{Constructive Negation}
\label{constructive}

When we tried to implement constructive negation algorithm we came
across several problems, including the management of constrained
answers and floundering. It
is our befief that these problems cannot be easily and efficiently
overcome. Therefore, we decided to design an implementation from
scratch.  One of our additional requirements is that we want to use a
standard Prolog implementation (to be able to reuse thousands of
existing Prolog lines and maintain their efficiency), so we will avoid
implementation-level manipulations.

Intuitively, the constructive negation of a goal, $cneg(G)$, is the
negation of the frontier $Frontier(G) \equiv C_1 \vee ... \vee
C_N$ (formal definition in \cite{Stuckey95}) of the goal $G$.

The solutions of $cneg(G)$ are the solutions of the combination
(conjunction) of one solution of each of the negations of the $N$
conjunctions $C_i$ of the frontier.
%Now we are going to explain how to negate 
The negation of a single conjunction $C_i$ is done in two phases:
\emph{Preparation} and \emph{Negation of the formula}. We describe in
detail both phases including unclear steps of the algorithm. 
 
We provide an additional step of {\bf simplification of the
conjunction}. If one of the terms of $C_i$ is trivially equivalent to
$true$ (e.g. $X=X$, $\forall X. s(X) \neq 0$), we can eliminate this
term from $C_i$. Symmetrically,if one of the terms is trivially
$fail$ing (e.g. $X \neq X$, $\forall X. X \neq Y$, $\forall X. X \neq
Y$, $\forall X. X = Y$), we can simplify $C_i \equiv fail$. The
simplification phase can be carried out during the generation of
frontier terms. We should take into account terms with universally
quantified variables (that were not taken into account in
\cite{Chan1,Chan2}) because without simplifying them it is impossible
to obtain results.

We also provide a variant in the {\bf negation of $\overline{D}_{exp}
           \wedge \overline{R}_{exp}$} that is the step where the
           disequalities with free variables, $\overline{D}_{exp}$ and
           the rest of terms of $C_i$ with free variables,
           $\overline{R}_{exp}$ are negated. This conjunction cannot
           be disclosed because of the negation of $ \exists~
           \overline{V}_{exp}.~ \overline{D}_{exp} \wedge
           \overline{R}_{exp}$, where $\overline{V}_{exp}$ gives
           universal quantifications: $\forall~ \overline{V}_{exp}.~
           cneg(\overline{D}_{exp} \wedge \overline{R}_{exp})$. The
           entire constructive negation algorithm must be applied
           again. Variables of $\overline{V}_{exp}$ are considered as
           free variables. When solutions of $cneg(\overline{D}_{exp}
           \wedge \overline{R}_{exp})$ are obtained some can be
           rejected: solutions with equalities with variables in
           $\overline{V}_{exp}$. If there is a disequality with any of
           these variables, e.g. $V$, the variable will be universally
           quantified in the disequality.  This is the way to negate
           the negation of a goal, but there is a detail that was not
           considered in former approaches and that is necessary to
           get a sound implementation: the existence of universally
           quantified variables in $\overline{D}_{exp} \wedge
           \overline{R}_{exp}$ by the iterative application of the
           method.  So, what we are really negating is a subgoal of
           the form: $ \exists.~ \overline{V}_{exp}~
           \overline{D}_{exp} \wedge \overline{R}_{exp}$. Its negation
           is $\forall ~ \overline{V}_{exp}.~ \neg~(\overline{D}_{exp}
           \wedge \overline{R}_{exp})$.

An instrumental step for managing negation is to be able to handle
disequalities between terms such as $t_1 \neq t_2$.  The typical
Prolog resources for handling these disequalities are limited to the
built-in predicate {\tt /== /2}, which needs both terms to be ground
because it always succeeds in the presence of free variables.  It is
clear that a variable needs to be bound with a disequality to achieve
a ``constructive'' behaviour.  Moreover, when an equation $X =
t(\overline{Y})$ is negated, the free variables in the equation must
be universaly quantified, unless affected by a more external
quantification, i.e. $\forall~ \overline{Y}.~X \neq t(\overline{Y})$ is
the correct negation.  As we explained in \cite{SusanaPADL2000}, the
inclusion of disequalities and constrained answers has a very low
cost.
%%%%%%%% OPTIMIZATION  %%%%%%%%%%%%%%%%%%%%%%%%%%%%%%%%%

\subsubsection{Optimizing the algorithm and the implementation}
\label{optimization}

Our constructive negation algorithm and the implementation techniques
admit some additional optimizations that can improve the runtime
behaviour of the system. Basically, the optimizations rely on the
compact representation of information, as well as the early detection
of successful or failing branches.

\noindent
- {\bf Compact information}. In our system, negative information is
represented quite compactly, providing fewer solutions from the
negation of $\overline{I}$. The advantage is twofold. On the one hand
constraints contain more information and failing branches can be
detected earlier (i.e. the search space could be smaller). On the
other hand, if we ask for all solutions using backtracking, we are
cutting the search tree by offering all the solutions together in a
single answer.

\noindent
- {\bf Pruning subgoals}. The frontiers generation search tree can be
cut with a double action over the ground subgoals: removing the
subgoals whose failure we are able to detect early on, and simplifying the
subgoals that can be reduced to true.

\noindent
- {\bf Constraint simplification}. During the whole process for negating
a goal,the frontier variables are constrained. In cases where the
constraints are satisfiable, they can be eliminated and where the
constraints can be reduced to fail, the evaluation can be stopped with
result \emph{true}.
 
%%%%%%%%%%%%%%%%%%%%%%%% RESULTS %%%%%%%%%%%%%%%%%%%%%%%%%%%%%%%%%

\subsubsection{Experimental results}
We have firstly measured the execution times in milliseconds for the
above examples when using negation as failure ($naf/1$) and
constructive negation ($cneg/1$). All measurements were made using \ciao\
Prolog.

Using {\bf naf} instead of {\bf cneg} results in small ratios around
1.06 on average for ground calls with few recursive calls. So, the
possible slow-down for constructive negation is not so high as we
might expect for these examples. Furthermore, the results are rather
similar. But the same goals with data that involve many recursive
calls yield ratios near 14.69 on average w.r.t {\bf naf},
increasing exponentially with the number of recursive calls. There
are, of course, many goals that cannot be negated using the \naf\
technique and that are solved using constructive negation.

%\begin{table}[t]
%\begin{small} 
\begin{tabular}{||l|r|r|r|r||}
\hline %------------------------------------------------------------------
\hline %-------------------------------------------------------------------
{\bf goals} &~~ {\bf Goal} ~~& ~~{\bf naf(Goal) }~~ &~~ {\bf cneg(Goal)}~~ &~~ {\bf ratio}~~ \\ 

\hline %--------------------------------------------------
boole(1)                     &  2049      &  2099    &  2069   &   0.98   \\ 
\hline %--------------------------------------------------
boole(8)                     &  2070      &  2170    &  2590   &   1.19   \\ 
\hline %--------------------------------------------------
boole(X)                     &  2080      &  -       &  3109   &          \\ 
\hline %--------------------------------------------------
positive(s(s(s(s(s(s(0))))))~~~ &  2079      &  1600    &  2159   &   1.3    \\ 
\hline %--------------------------------------------------
positive(s(s(s(s(s(0))))))   &  2079      &  2139    &  2060   &   0.96   \\ 
\hline %--------------------------------------------------
positive(X)                  &  2020      &  -       &  7189   &          \\ 
\hline %--------------------------------------------------
greater(s(s(s(0))),s(0))     &  2110      &  2099    &  2100   &   1.00   \\ 
\hline %--------------------------------------------------
greater(s(0),s(s(s(0))))     &  2119      &  2129    &  2089   &   0.98   \\ 
\hline %--------------------------------------------------
greater(s(s(s(0))),X)        &  2099      &  -       &  6990   &          \\ 
\hline %--------------------------------------------------
greater(X,Y)                 &  7040      &  -       &  7519   &          \\ 
\hline %--------------------------------------------------
queens(s(s(0)),Qs)           &  6939      &  -       &  9119   &          \\ 

\hline %-------------------------------------------------------
\hline %-------------------------------------------------------
{\bf average}                &            &          &         &   1.06   \\ 
\hline %--------------------------------------------------------------------------------
\hline %--------------------------------------------------------------------------------
\end{tabular}
\vspace*{3mm}
\caption{Runtime comparation}
\label{table}
%\end{small}
\end{table}
 
 

%%%%%%%%%%%%%%%%%%%%%%%%%%%%%%%%%%%%%%%%%%%%%%%%%%%%%%%%%%%%%%%%%%
%%%%%%%%%%%%%%%%%%%%%%%% CONCLUSIONS %%%%%%%%%%%%%%%%%%%%%%%%%%%%%
%%%%%%%%%%%%%%%%%%%%%%%%%%%%%%%%%%%%%%%%%%%%%%%%%%%%%%%%%%%%%%%%%%

\vspace{-1em}
\subsubsection{Conclusion and Future Work}
\label{conclusion}
\vspace{-1em}
After running some preliminary experiments with the constructive 
negation technique  following Chan's description, we realized that the
algorithm needed some additional explanations and modifications.

Having given a detailed specification of algorithm in a detailed way
we proceed to provide a real, complete and consistent
implementation. The result, we have reported are very encouraging,
because we have proved that it is possible to extend Prolog with a
constructive negation module relatively inexpensively. 
%Nevertheless,
%it is quite important to address possible optimizations, and 
Nevertheless, we are
working to improve the efficiency of the implementation. This include
a more accurate selection of the frontier based on the demanded form.
%of argument in the vein of \cite{Moreno2}). 
Other future work is to
incorporate our algorithm at the WAM machine level.

We are testing the implementation and trying to improve the code, and
our intention is to include it in the next version of Ciao Prolog
\footnote{http://www.clip.dia.fi.upm.es/Software}.
  

%%%%%%%%%%%%%%%%%%%%%%%%%%%%%%%%%%%%%%%%%%%%%%%%%%%%%%%%%%%%%%%%%
%%%%%%%%%%%%%%%%%%%%%% BIBLIOGRAPHY %%%%%%%%%%%%%%%%%%%%%%%%%%%%%
%%%%%%%%%%%%%%%%%%%%%%%%%%%%%%%%%%%%%%%%%%%%%%%%%%%%%%%%%%%%%%%%%

 \begin{small}

% \linespread{0.80}
    \bibliographystyle{plain} 
    \bibliography{bibliography}

 \end{small}


%%%%%%%%%%%%%%%%%%%%%%%%%%%%%%%%%%%%%%%%%%%%%%%%%%%%%%%%%%%%%%%%%


\end{document}


%%%%%%%%%%%%%%%%%%%%%%%%%%%%%%%%%%%%%%%%%%%%%%%%%%%%%%%%%%%%%%%%%
%%%%%%%%%%%%%%%%%%%%%%%  THE END  %%%%%%%%%%%%%%%%%%%%%%%%%%%%%%
%%%%%%%%%%%%%%%%%%%%%%%%%%%%%%%%%%%%%%%%%%%%%%%%%%%%%%%%%%%%%%%%%

