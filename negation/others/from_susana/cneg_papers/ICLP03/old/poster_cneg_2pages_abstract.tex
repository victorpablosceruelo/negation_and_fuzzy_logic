\documentclass{llncs}

\usepackage{pst-node}
\usepackage{amsmath} 
\usepackage{amssymb}
\usepackage{latexsym}
\usepackage{times}
\usepackage{mathptmx}
\usepackage{theorem}


\newcommand{\naf}{{\em naf}}\newcommand{\viejo}[1]{}
\newcommand{\ciao}{Ciao}

%%%%%%%%%%%%%%%%%%%%%%%%%%%%%%%%%%%%%%%%%%%%%%%%%%%%%%%%%%%%%%%%%

\begin{document}

\title{A Real Implementation for 
       Constructive Negation}

\author{Susana Mu\~{n}oz ~~~~ Juan Jos\'{e} Moreno-Navarro}

\institute{ 
LSIIS, Facultad de Inform\'{a}tica \\
Universidad Polit\'{e}cnica de  Madrid \\ 
Campus de Montegancedo s/n Boadilla del Monte\\
28660 Madrid, Spain \footnote{This work was partly supported by the
Spanish MCYT project TIC2000-1632.} \\
\{ susana $|$ jjmoreno\}@fi.upm.es  \\
}

\maketitle

\paragraph{\bf Keywords}
Constructive Negation, Negation in Logic Programming, Constraint Logic
Programming, Implementations of Logic Programming.


\vspace{0.3cm}

%%%%%%%%%%%%%%%%%%%%%%%%%%%%%%%%%%%%%%%%%%%%%%%%%%%%%%%%%%%%%%%%
%%%%%%%%%%%%%%%%%%%%%%%% INTRODUCTION %%%%%%%%%%%%%%%%%%%%%%%%%%
%%%%%%%%%%%%%%%%%%%%%%%%%%%%%%%%%%%%%%%%%%%%%%%%%%%%%%%%%%%%%%%%

%\subsubsection{Introduction}
\begin{abstract}
Logic Programming has been advocated as a language for system
  specification, especially for logical behaviours,
  rules and knowledge. However, modeling problems involving negation,
  which is quite natural in many cases, is somewhat restricted if
  Prolog is used as the specification/implementation language. These
  constraints are not related to theory viewpoint, where users can
  find many different models with their respective semantics; they
  concern practical implementation issues.  The negation capabilities
  supported by current Prolog systems are rather limited, and a
  correct and complete implementation there is not available. Of all
  the proposals, constructive negation \cite{Chan1,Chan2} is probably
  the most promising because it has been proven to be sound and
  complete \cite{Stuckey95}, and its semantics is fully compatible
  with Prolog's.

Intuitively, the constructive negation of a goal, $cneg(G)$, is the
negation of the frontier $Frontier(G) \equiv C_1 \vee ... \vee C_N$
(formal definition in \cite{Stuckey95}) of the goal $G$. After running
some preliminary experiments with the constructive negation technique
following Chan's description, we realized that the algorithm needed
some additional explanations and modifications.


%The  goal of this paper 
Our goal is to give an algorithmic description of constructive
negation, i.e. explicitly stating the details and discussing the
pragmatic ideas needed to provide a real implementation. Early results
for a concrete implementation extending the \ciao\ Prolog compiler are
also presented.

%%%%%%%%%%%%%%%%%%%%%%%%%%%%%%%%%%%%%%%%%%%%%%%%%%%%%%%%%%%%%%%%%%
%%%%%%%%%%%%%%%   CONSTRUCTIVE NEGATION   %%%%%%%%%%%%%%%%%%%%%%%%
%%%%%%%%%%%%%%%%%%%%%%%%%%%%%%%%%%%%%%%%%%%%%%%%%%%%%%%%%%%%%%%%%%

%\subsubsection{Constructive Negation}

Constructive negation was, in fact, announced in early versions of the
  Eclipse Prolog compiler, but was removed from the latest releases.
  The reasons seem to be related to some technical problems with the
  use of coroutining (risk of floundering) and the management of
  constrained solutions. It is our belief
  that these problems cannot be easily and efficiently
  overcome. Therefore, we decided to design an implementation from
  scratch.  One of our additional requirements is that we want to use
  a standard Prolog implementation (to be able to reuse thousands of
  existing Prolog lines and maintain their efficiency), so we will
  avoid implementation-level manipulations.


We provide an additional step of {\bf simplification} during the
generation of frontier terms. We should take into account terms with
universally quantified variables (that were not taken into account in
\cite{Chan1,Chan2}) because without simplifying them it is impossible
to obtain results. We also provide a variant in the {\bf negation of
terms with free variables} that entails universal
quantifications. There is a detail that was not considered in former
approaches and that is necessary to get a sound implementation: the
existence of universally quantified variables by the iterative
application of the method.

An instrumental step for managing negation is to be able to handle
disequalities between terms with a ``constructive'' behaviour.
Moreover, when an equation $ \exists ~ \overline{Y}. ~ X =
t(\overline{Y})$ is negated, the free variables in the equation must
be universally quantified, unless affected by a more external
quantification, i.e. $\forall~ \overline{Y}.~X \neq t(\overline{Y})$
is the correct negation.  As we explained in \cite{SusanaPADL2000},
the inclusion of disequalities and constrained answers has a very low
cost.

%%%%%%%%%%%%%%%%%%%%%%%%%%%%%%%%%%%%%%%%%%%%%%%%%%%%%%%%%%%%%%%%%%
%%%%%%%%%%%%%%%%%%%    OPTIMIZATION    %%%%%%%%%%%%%%%%%%%%%%%%%%%
%%%%%%%%%%%%%%%%%%%%%%%%%%%%%%%%%%%%%%%%%%%%%%%%%%%%%%%%%%%%%%%%%%

%\subsubsection{Optimizing the algorithm and the implementation}

Our constructive negation algorithm and the implementation techniques
admit some additional optimizations that can improve the system runtime
behaviour.
% Basically, the optimizations rely on the
%compact representation of information, as well as the early detection
%of successful or failing branches.

\noindent
- {\bf Compact representation of the information}. The advantage is
twofold. On the one hand constraints contain more information and
failing branches can be detected earlier (i.e. the search space could
be smaller). On the other hand, if we ask for all solutions instead of
using backtracking, we are cutting the search tree by offering all the
solutions in a single answer.

\noindent
- {\bf Pruning subgoals}. The frontiers generation search tree can be
cut with a double action over the ground subgoals: removing the
subgoals whose failure we are able to detect early on, and simplifying the
subgoals that can be reduced to true.

\noindent
- {\bf Constraint simplification}. During the whole process for negating
a goal,the frontier variables are constrained. In cases where the
constraints are satisfiable, they can be eliminated and where the
constraints can be reduced to fail, the evaluation can be stopped with
result \emph{true}.
 
%%%%%%%%%%%%%%%%%%%%%%%%%%%%%%%%%%%%%%%%%%%%%%%%%%%%%%%%%%%%%%%%%%
%%%%%%%%%%%%%%%%%%%%%%%% CONCLUSIONS %%%%%%%%%%%%%%%%%%%%%%%%%%%%%
%%%%%%%%%%%%%%%%%%%%%%%%%%%%%%%%%%%%%%%%%%%%%%%%%%%%%%%%%%%%%%%%%%

%\subsubsection{Conclusion and Future Work}

Having given a detailed specification of algorithm in a detailed way
we proceed to provide a real, complete and consistent
implementation. The results that we have reported are very
encouraging, because we have proved that it is possible to extend
Prolog with a constructive negation module relatively inexpensively
and we have provided experimental results. Nevertheless, we are
working to improve the efficiency of the implementation. This include
a more accurate selection of the frontier based on the demanded
form. Other future work is to incorporate our algorithm at the WAM
machine level. We are testing the implementation and trying to improve
the code, and our intention is to include it in the next version of
Ciao Prolog
\footnote{http://www.clip.dia.fi.upm.es/Software}.
  
\end{abstract}

%%%%%%%%%%%%%%%%%%%%%%%%%%%%%%%%%%%%%%%%%%%%%%%%%%%%%%%%%%%%%%%%%
%%%%%%%%%%%%%%%%%%%%%% BIBLIOGRAPHY %%%%%%%%%%%%%%%%%%%%%%%%%%%%%
%%%%%%%%%%%%%%%%%%%%%%%%%%%%%%%%%%%%%%%%%%%%%%%%%%%%%%%%%%%%%%%%%

\begin{small}

    \bibliographystyle{plain} 
    \bibliography{bibliography}

\end{small}


%%%%%%%%%%%%%%%%%%%%%%%%%%%%%%%%%%%%%%%%%%%%%%%%%%%%%%%%%%%%%%%%%


\end{document}


%%%%%%%%%%%%%%%%%%%%%%%%%%%%%%%%%%%%%%%%%%%%%%%%%%%%%%%%%%%%%%%%%
%%%%%%%%%%%%%%%%%%%%%%%  THE END  %%%%%%%%%%%%%%%%%%%%%%%%%%%%%%
%%%%%%%%%%%%%%%%%%%%%%%%%%%%%%%%%%%%%%%%%%%%%%%%%%%%%%%%%%%%%%%%%

